\documentclass[10pt]{article}
\usepackage[utf8]{inputenc}
\usepackage[T1]{fontenc}
\usepackage{amsmath}
\usepackage{amsfonts}
\usepackage{amssymb}
\usepackage[version=4]{mhchem}
\usepackage{stmaryrd}
\usepackage{graphicx}
\usepackage[export]{adjustbox}
\graphicspath{ {./images/} }
\usepackage{caption}
\usepackage{mathrsfs}
\usepackage{bbold}
\usepackage{fvextra, csquotes}

%New command to display footnote whose markers will always be hidden
\let\svthefootnote\thefootnote
\newcommand\blfootnotetext[1]{%
  \let\thefootnote\relax\footnote{#1}%
  \addtocounter{footnote}{-1}%
  \let\thefootnote\svthefootnote%
}

%Overriding the \footnotetext command to hide the marker if its value is `0`
\let\svfootnotetext\footnotetext
\renewcommand\footnotetext[2][?]{%
  \if\relax#1\relax%
    \ifnum\value{footnote}=0\blfootnotetext{#2}\else\svfootnotetext{#2}\fi%
  \else%
    \if?#1\ifnum\value{footnote}=0\blfootnotetext{#2}\else\svfootnotetext{#2}\fi%
    \else\svfootnotetext[#1]{#2}\fi%
  \fi
}

\DeclareUnicodeCharacter{21D2}{\ifmmode\Rightarrow\else{$\Rightarrow$}\fi}
\DeclareUnicodeCharacter{21D4}{\ifmmode\Leftrightarrow\else{$\Leftrightarrow$}\fi}

\begin{document}
\captionsetup{singlelinecheck=false}
\section*{Chapter 2}
\section*{Topological Spaces and Continuous Functions}
The concept of topological space grew out of the study of the real line and euclidean space and the study of continuous functions on these spaces. In this chapter, we define what a topological space is, and we study a number of ways of constructing a topology on a set so as to make it into a topological space. We also consider some of the elementary concepts associated with topological spaces. Open and closed sets, limit points, and continuous functions are introduced as natural generalizations of the corresponding ideas for the real line and euclidean space.

\section*{§12 Topological Spaces}
The definition of a topological space that is now standard was a long time in being formulated. Various mathematicians-Fréchet, Hausdorff, and others-proposed different definitions over a period of years during the first decades of the twentieth century, but it took quite a while before mathematicians settled on the one that seemed most suitable. They wanted, of course, a definition that was as broad as possible, so that it would include as special cases all the various examples that were useful in mathematics-euclidean space, infinite-dimensional euclidean space, and function spaces among them-but they also wanted the definition to be narrow enough that the standard theorems about these familiar spaces would hold for topological spaces in\\
general. This is always the problem when one is trying to formulate a new mathematical concept, to decide how general its definition should be. The definition finally settled on may seem a bit abstract, but as you work through the various ways of constructing topological spaces, you will get a better feeling for what the concept means.

Definition. A topology on a set $X$ is a collection $\mathcal{T}$ of subsets of $X$ having the following properties:\\
(1) $\varnothing$ and $X$ are in $\mathcal{T}$.\\
(2) The union of the elements of any subcollection of $\mathcal{T}$ is in $\mathcal{T}$.\\
(3) The intersection of the elements of any finite subcollection of $\mathcal{T}$ is in $\mathcal{T}$.

A set $X$ for which a topology $\mathcal{T}$ has been specified is called a topological space.\\
Properly speaking, a topological space is an ordered pair $(X, \mathcal{T})$ consisting of a set $X$ and a topology $\mathcal{T}$ on $X$, but we often omit specific mention of $\mathcal{T}$ if no confusion will arise.

If $X$ is a topological space with topology $\mathcal{T}$, we say that a subset $U$ of $X$ is an open set of $X$ if $U$ belongs to the collection $\mathcal{T}$. Using this terminology, one can say that a topological space is a set $X$ together with a collection of subsets of $X$, called open sets, such that $\varnothing$ and $X$ are both open, and such that arbitrary unions and finite intersections of open sets are open.

Example 1. Let $X$ be a three-element set, $X=\{a, b, c\}$. There are many possible topologies on $X$, some of which are indicated schematically in Figure 12.1. The diagram in the upper right-hand corner indicates the topology in which the open sets are $X, \varnothing$, $\{a, b\},\{b\}$, and $\{b, c\}$. The topology in the upper left-hand corner contains only $X$ and $\varnothing$, while the topology in the lower right-hand corner contains every subset of $X$. You can get other topologies on $X$ by permuting $a, b$, and $c$.

\begin{figure}[h]
\begin{center}
  \includegraphics[width=\textwidth]{2025_11_21_a8ffce9f36674b61ee7eg-002}
\captionsetup{labelformat=empty}
\caption{Figure 12.1}
\end{center}
\end{figure}

From this example, you can see that even a three-element set has many different topologies. But not every collection of subsets of $X$ is a topology on $X$. Neither of the collections indicated in Figure 12.2 is a topology, for instance.

\begin{figure}[h]
\begin{center}
  \includegraphics[width=\textwidth]{2025_11_21_a8ffce9f36674b61ee7eg-003}
\captionsetup{labelformat=empty}
\caption{Figure 12.2}
\end{center}
\end{figure}

EXAMPLE 2. If $X$ is any set, the collection of all subsets of $X$ is a topology on $X$; it is called the discrete topology. The collection consisting of $X$ and $\varnothing$ only is also a topology on $X$; we shall call it the indiscrete topology, or the trivial topology.

EXAMPLE 3. Let $X$ be a set; let $\mathcal{T}_{f}$ be the collection of all subsets $U$ of $X$ such that $X-U$ either is finite or is all of $X$. Then $\mathcal{T}_{f}$ is a topology on $X$, called the finite complement topology. Both $X$ and $\varnothing$ are in $\mathcal{T}_{f}$, since $X-X$ is finite and $X-\varnothing$ is all of $X$. If $\left\{U_{\alpha}\right\}$ is an indexed family of nonempty elements of $\mathcal{T}_{f}$, to show that $\bigcup U_{\alpha}$ is in $\mathcal{T}_{f}$, we compute

$$
X-\bigcup U_{\alpha}=\bigcap\left(X-U_{\alpha}\right) .
$$

The latter set is finite because each set $X-U_{\alpha}$ is finite. If $U_{1}, \ldots, U_{n}$ are nonempty elements of $\mathcal{T}_{f}$, to show that $\bigcap U_{i}$ is in $\mathcal{T}_{f}$, we compute

$$
X-\bigcap_{i=1}^{n} U_{i}=\bigcup_{i=1}^{n}\left(X-U_{i}\right) .
$$

The latter set is a finite union of finite sets and, therefore, finite.\\
EXAMPLE 4. Let $X$ be a set; let $\mathcal{T}_{c}$ be the collection of all subsets $U$ of $X$ such that $X-U$ either is countable or is all of $X$. Then $\mathcal{T}_{c}$ is a topology on $X$, as you can check.

Definition. Suppose that $\mathcal{T}$ and $\mathcal{T}^{\prime}$ are two topologies on a given set $X$. If $\mathcal{T}^{\prime} \supset \mathcal{T}$, we say that $\mathcal{T}^{\prime}$ is finer than $\mathcal{T}$; if $\mathcal{T}^{\prime}$ properly contains $\mathcal{T}$, we say that $\mathcal{T}^{\prime}$ is strictly finer than $\mathcal{T}$. We also say that $\mathcal{T}$ is coarser than $\mathcal{T}^{\prime}$, or strictly coarser, in these two respective situations. We say $\mathcal{T}$ is comparable with $\mathcal{T}^{\prime}$ if either $\mathcal{T}^{\prime} \supset \mathcal{T}$ or $\mathcal{T} \supset \mathcal{T}^{\prime}$.

This terminology is suggested by thinking of a topological space as being something like a truckload full of gravel-the pebbles and all unions of collections of pebbles being the open sets. If now we smash the pebbles into smaller ones, the collection of open sets has been enlarged, and the topology, like the gravel, is said to have been made finer by the operation.

Two topologies on $X$ need not be comparable, of course. In Figure 12.1 preceding, the topology in the upper right-hand corner is strictly finer than each of the three topologies in the first column and strictly coarser than each of the other topologies in the third column. But it is not comparable with any of the topologies in the second column.

Other terminology is sometimes used for this concept. If $\mathcal{T}^{\prime} \supset \mathcal{T}$, some mathematicians would say that $\mathcal{T}^{\prime}$ is larger than $\mathcal{T}$, and $\mathcal{T}$ is smaller than $\mathcal{T}^{\prime}$. This is certainly acceptable terminology, if not as vivid as the words "finer" and "coarser."

Many mathematicians use the words "weaker" and "stronger" in this context. Unfortunately, some of them (particularly analysts) are apt to say that $\mathcal{T}^{\prime}$ is stronger than $\mathcal{T}$ if $\mathcal{T}^{\prime} \supset \mathcal{T}$, while others (particularly topologists) are apt to say that $\mathcal{T}^{\prime}$ is weaker than $\mathcal{T}$ in the same situation! If you run across the terms "strong topology" or "weak topology" in some book, you will have to decide from the context which inclusion is meant. We shall not use these terms in this book.

\section*{§13 Basis for a Topology}
For each of the examples in the preceding section, we were able to specify the topology by describing the entire collection $\mathcal{T}$ of open sets. Usually this is too difficult. In most cases, one specifies instead a smaller collection of subsets of $X$ and defines the topology in terms of that.

Definition. If $X$ is a set, abasis for a topology on $X$ is a collection $B$ of subsets of $X$ (called basis elements) such that\\
(1) For each $x \in X$, there is at least one basis element $B$ containing $x$.\\
(2) If $x$ belongs to the intersection of two basis elements $B_{1}$ and $B_{2}$, then there is a basis element $B_{3}$ containing $x$ such that $B_{3} \subset B_{1} \cap B_{2}$.\\
If $B$ satisfies these two conditions, then we define the topology $\mathcal{T}$ generated by $\mathscr{B}$ as follows: $A$ subset $U$ of $X$ is said to be open in $X$ (that is, to be an element of $\mathcal{T}$ ) if for each $x \in U$, there is a basis element $B \in \mathscr{B}$ such that $x \in B$ and $B \subset U$. Note that each basis element is itself an element of $\mathcal{T}$.

We will check shortly that the collection $\mathcal{T}$ is indeed a topology on $X$. But first let us consider some examples.

EXAMPLE 1. Let $\mathfrak{B}$ be the collection of all circular regions (interiors of circles) in the plane. Then $\mathscr{B}$ satisfies both conditions for a basis. The second condition is illustrated in Figure 13.1. In the topology generated by $B$, a subset $U$ of the plane is open if every $x$ in $U$ lies in some circular region contained in $U$.

\begin{figure}[h]
\begin{center}
  \includegraphics[width=\textwidth]{2025_11_21_a8ffce9f36674b61ee7eg-004(1)}
\captionsetup{labelformat=empty}
\caption{Figure 13.1}
\end{center}
\end{figure}

\begin{figure}[h]
\begin{center}
  \includegraphics[width=\textwidth]{2025_11_21_a8ffce9f36674b61ee7eg-004}
\captionsetup{labelformat=empty}
\caption{Figure 13.2}
\end{center}
\end{figure}

EXAMPLE 2. Let $\mathscr{B}^{\prime}$ be the collection of all rectangular regions (interiors of rectangles) in the plane, where the rectangles have sides parallel to the coordinate axes. Then $\mathcal{B}^{\prime}$ satisfies both conditions for a basis. The second condition is illustrated in Figure 13.2; in this case, the condition is trivial, because the intersection of any two basis elements is itself a basis element (or empty). As we shall see later, the basis $\mathscr{B}^{\prime}$ generates the same topology on the plane as the basis $\mathscr{B}$ given in the preceding example.

EXAMPLE 3. If $X$ is any set, the collection of all one-point subsets of $X$ is a basis for the discrete topology on $X$.

Let us check now that the collection $\mathcal{T}$ generated by the basis $\mathscr{B}$ is, in fact, a topology on $X$. If $U$ is the empty set, it satisfies the defining condition of openness vacuously. Likewise, $X$ is in $\mathcal{T}$, since for each $x \in X$ there is some basis element $B$ containing $x$ and contained in $X$. Now let us take an indexed family $\left\{U_{\alpha}\right\}_{\alpha \in J}$, of elements of $\mathcal{T}$ and show that

$$
U=\bigcup_{\alpha \in J} U_{\alpha}
$$

belongs to $\mathcal{T}$. Given $x \in U$, there is an index $\alpha$ such that $x \in U_{\alpha}$. Since $U_{\alpha}$ is open, there is a basis element $B$ such that $x \in B \subset U_{\alpha}$. Then $x \in B$ and $B \subset U$, so that $U$ is open, by definition.

Now let us take two elements $U_{1}$ and $U_{2}$ of $\mathcal{T}$ and show that $U_{1} \cap U_{2}$ belongs to $\mathcal{T}$. Given $x \in U_{1} \cap U_{2}$, choose a basis element $B_{1}$ containing $x$ such that $B_{1} \subset U_{1}$; choose also a basis element $B_{2}$ containing $x$ such that $B_{2} \subset U_{2}$. The second condition for a basis enables us to choose a basis element $B_{3}$ containing $x$ such that $B_{3} \subset B_{1} \cap B_{2}$. See Figure 13.3. Then $x \in B_{3}$ and $B_{3} \subset U_{1} \cap U_{2}$, so $U_{1} \cap U_{2}$ belongs to $\mathcal{T}$, by definition.

\begin{figure}[h]
\begin{center}
  \includegraphics[width=\textwidth]{2025_11_21_a8ffce9f36674b61ee7eg-005}
\captionsetup{labelformat=empty}
\caption{Figure 13.3}
\end{center}
\end{figure}

Finally, we show by induction that any finite intersection $U_{1} \cap \cdots \cap U_{n}$ of elements of $\mathcal{T}$ is in $\mathcal{T}$. This fact is trivial for $n=1$; we suppose it true for $n-1$ and prove it for $n$. Now

$$
\left(U_{1} \cap \cdots \cap U_{n}\right)=\left(U_{1} \cap \cdots \cap U_{n-1}\right) \cap U_{n}
$$

By hypothesis, $U_{1} \cap \cdots \cap U_{n-1}$ belongs to $\mathcal{T}$; by the result just proved, the intersection of $U_{1} \cap \cdots \cap U_{n-1}$ and $U_{n}$ also belongs to $\mathcal{T}$.

Thus we have checked that collection of open sets generated by a basis $\mathscr{B}$ is, in fact, a topology.

Another way of describing the topology generated by a basis is given in the following lemma:

Lemma 13.1. Let $X$ be a set; let $\mathcal{B}$ be a basis for a topology $\mathcal{T}$ on $X$. Then $\mathcal{T}$ equals the collection of all unions of elements of $\mathfrak{B}$.

Proof. Given a collection of elements of $\mathfrak{B}$, they are also elements of $\mathcal{T}$. Because $\mathcal{T}$ is a topology, their union is in $\mathcal{T}$. Conversely, given $U \in \mathcal{T}$, choose for each $x \in U$ an element $B_{x}$ of $\mathscr{B}$ such that $x \in B_{x} \subset U$. Then $U=\bigcup_{x \in U} B_{x}$, so $U$ equals a union of elements of $\mathfrak{B}$.

This lemma states that every open set $U$ in $X$ can be expressed as a union of basis elements. This expression for $U$ is not, however, unique. Thus the use of the term "basis" in topology differs drastically from its use in linear algebra, where the equation expressing a given vector as a linear combination of basis vectors is unique.

We have described in two different ways how to go from a basis to the topology it generates. Sometimes we need to go in the reverse direction, from a topology to a basis generating it. Here is one way of obtaining a basis for a given topology; we shall use it frequently.

Lemma 13.2. Let $X$ be a topological space. Suppose that $\mathcal{C}$ is a collection of open sets of $X$ such that for each open set $U$ of $X$ and each $x$ in $U$, there is an element $C$ of $\mathcal{C}$ such that $x \in C \subset U$. Then $\mathcal{C}$ is a basis for the topology of $X$.

Proof. We must show that $\mathcal{C}$ is a basis. The first condition for a basis is easy: Given $x \in X$, since $X$ is itself an open set, there is by hypothesis an element $C$ of $\mathcal{C}$ such that $x \in C \subset X$. To check the second condition, let $x$ belong to $C_{1} \cap C_{2}$, where $C_{1}$ and $C_{2}$ are elements of $\mathcal{C}$. Since $C_{1}$ and $C_{2}$ are open, so is $C_{1} \cap C_{2}$. Therefore, there exists by hypothesis an element $C_{3}$ in $\mathcal{C}$ such that $x \in C_{3} \subset C_{1} \cap C_{2}$.

Let $\mathcal{T}$ be the collection of open sets of $X$; we must show that the topology $\mathcal{T}^{\prime}$ generated by $\mathcal{C}$ equals the topology $\mathcal{T}$. First, note that if $U$ belongs to $\mathcal{T}$ and if $x \in U$, then there is by hypothesis an element $C$ of $\mathcal{C}$ such that $x \in C \subset U$. It follows that $U$ belongs to the topology $\mathcal{T}^{\prime}$, by definition. Conversely, if $W$ belongs to the topology $\mathcal{T}^{\prime}$, then $W$ equals a union of elements of $\mathcal{C}$, by the preceding lemma. Since each element of $\mathcal{C}$ belongs to $\mathcal{T}$ and $\mathcal{T}$ is a topology, $W$ also belongs to $\mathcal{T}$.

When topologies are given by bases, it is useful to have a criterion in terms of the bases for determining whether one topology is finer than another. One such criterion is the following:

Lemma 13.3. Let $\mathfrak{B}$ and $\mathscr{B}^{\prime}$ be bases for the topologies $\mathcal{T}$ and $\mathcal{T}^{\prime}$, respectively, on $X$. Then the following are equivalent:\\
(1) $\mathcal{T}^{\prime}$ is finer than $\mathcal{T}$.\\
(2) For each $x \in X$ and each basis element $B \in B$ containing $x$, there is a basis element $B^{\prime} \in \mathscr{B}^{\prime}$ such that $x \in B^{\prime} \subset B$.

Proof. (2) ⇒ (1). Given an element $U$ of $\mathcal{T}$, we wish to show that $U \in \mathcal{T}^{\prime}$. Let $x \in U$. Since $\mathscr{B}$ generates $\mathcal{T}$, there is an element $B \in \mathscr{B}$ such that $x \in B \subset U$. Condition (2) tells us there exists an element $B^{\prime} \in B^{\prime}$ such that $x \in B^{\prime} \subset B$. Then $x \in B^{\prime} \subset U$, so $U \in \mathcal{T}^{\prime}$, by definition.\\
(1) ⇒ (2). We are given $x \in X$ and $B \in B$, with $x \in B$. Now $B$ belongs to $\mathcal{T}$ by definition and $\mathcal{T} \subset \mathcal{T}^{\prime}$ by condition (1); therefore, $B \in \mathcal{T}^{\prime}$. Since $\mathcal{T}^{\prime}$ is generated by $\mathscr{B}^{\prime}$, there is an element $B^{\prime} \in \mathscr{B}^{\prime}$ such that $x \in B^{\prime} \subset B$.

Some students find this condition hard to remember. "Which way does the inclusion go?" they ask. It may be easier to remember if you recall the analogy between a topological space and a truckload full of gravel. Think of the pebbles as the basis elements of the topology; after the pebbles are smashed to dust, the dust particles are the basis elements of the new topology. The new topology is finer than the old one, and each dust particle was contained inside a pebble, as the criterion states.

EXAMPLE 4. One can now see that the collection $\mathscr{B}$ of all circular regions in the plane generates the same topology as the collection $\mathscr{B}^{\prime}$ of all rectangular regions; Figure 13.4 illustrates the proof. We shall treat this example more formally when we study metric spaces.

\begin{figure}[h]
\begin{center}
  \includegraphics[width=\textwidth]{2025_11_21_a8ffce9f36674b61ee7eg-007}
\captionsetup{labelformat=empty}
\caption{Figure 13.4}
\end{center}
\end{figure}

We now define three topologies on the real line $\mathbb{R}$, all of which are of interest.

Definition. If $B$ is the collection of all open intervals in the real line,

$$
(a, b)=\{x \mid a<x<b\},
$$

the topology generated by $\mathscr{B}$ is called the standard topology on the real line. Whenever we consider $\mathbb{R}$, we shall suppose it is given this topology unless we specifically state otherwise. If $\mathscr{B}^{\prime}$ is the collection of all half-open intervals of the form

$$
[a, b)=\{x \mid a \leq x<b\},
$$

where $a<b$, the topology generated by $\mathscr{B}^{\prime}$ is called the lower limit topology on $\mathbb{R}$. When $\mathbb{R}$ is given the lower limit topology, we denote it by $\mathbb{R}_{\ell}$. Finally let $K$ denote the set of all numbers of the form $1 / n$, for $n \in \mathbb{Z}_{+}$, and let $\mathfrak{B}^{\prime \prime}$ be the collection of all open intervals $(a, b)$, along with all sets of the form $(a, b)-K$. The topology generated by $\mathfrak{B}^{\prime \prime}$ will be called the K-topology on $\mathbb{R}$. When $\mathbb{R}$ is given this topology, we denote it by $\mathbb{R}_{K}$.

It is easy to see that all three of these collections are bases; in each case, the intersection of two basis elements is either another basis element or is empty. The relation between these topologies is the following:

Lemma 13.4. The topologies of $\mathbb{R}_{\ell}$ and $\mathbb{R}_{K}$ are strictly finer than the standard topology on $\mathbb{R}$, but are not comparable with one another.

Proof. Let $\mathcal{T}, \mathcal{T}^{\prime}$, and $\mathcal{T}^{\prime \prime}$ be the topologies of $\mathbb{R}, \mathbb{R}_{\ell}$, and $\mathbb{R}_{K}$, respectively. Given a basis element $(a, b)$ for $\mathcal{T}$ and a point $x$ of $(a, b)$, the basis element $[x, b)$ for $\mathcal{T}^{\prime}$ contains $x$ and lies in ( $a, b$ ). On the other hand, given the basis element $[x, d)$ for $\mathcal{T}^{\prime}$, there is no open interval ( $a, b$ ) that contains $x$ and lies in $[x, d)$. Thus $\mathcal{T}^{\prime}$ is strictly finer than $\mathcal{T}$.

A similar argument applies to $\mathbb{R}_{K}$. Given a basis element $(a, b)$ for $\mathcal{T}$ and a point $x$ of ( $a, b$ ), this same interval is a basis element for $\mathcal{T}^{\prime \prime}$ that contains $x$. On the other hand, given the basis element $B=(-1,1)-K$ for $\mathcal{T}^{\prime \prime}$ and the point 0 of $B$, there is no open interval that contains 0 and lies in $B$.

We leave it to you to show that the topologies of $\mathbb{R}_{\ell}$ and $\mathbb{R}_{K}$ are not comparable.

A question may occur to you at this point. Since the topology generated by a basis $\mathscr{B}$ may be described as the collection of arbitrary unions of elements of $\mathscr{B}$, what happens if you start with a given collection of sets and take finite intersections of them as well as arbitrary unions? This question leads to the notion of a subbasis for a topology.

Definition. A subbasis $\mathcal{S}$ for a topology on $X$ is a collection of subsets of $X$ whose union equals $X$. The topology generated by the subbasis $S$ is defined to be the collection $\mathcal{T}$ of all unions of finite intersections of elements of $S$.

We must of course check that $\mathcal{T}$ is a topology. For this purpose it will suffice to show that the collection $\mathscr{B}$ of all finite intersections of elements of $\mathcal{S}$ is a basis, for then the collection $\mathcal{T}$ of all unions of elements of $\mathcal{B}$ is a topology, by Lemma 13.1. Given $x \in X$, it belongs to an element of $S$ and hence to an element of $\mathfrak{B}$; this is the first condition for a basis. To check the second condition, let

$$
B_{1}=S_{1} \cap \cdots \cap S_{m} \quad \text { and } \quad B_{2}=S_{1}^{\prime} \cap \cdots \cap S_{n}^{\prime}
$$

be two elements of $\mathcal{B}$. Their intersection

$$
B_{1} \cap B_{2}=\left(S_{1} \cap \cdots \cap S_{m}\right) \cap\left(S_{1}^{\prime} \cap \cdots \cap S_{n}^{\prime}\right)
$$

is also a finite intersection of elements of $S$, so it belongs to $\mathscr{B}$.

\section*{Exercises}
\begin{enumerate}
  \item Let $X$ be a topological space; let $A$ be a subset of $X$. Suppose that for each $x \in A$ there is an open set $U$ containing $x$ such that $U \subset A$. Show that $A$ is open in $X$.
  \item Consider the nine topologies on the set $X=\{a, b, c\}$ indicated in Example 1 of §12. Compare them; that is, for each pair of topologies, determine whether they are comparable, and if so, which is the finer.
  \item Show that the collection $\mathcal{T}_{c}$ given in Example 4 of $\S 12$ is a topology on the set $X$. Is the collection
\end{enumerate}

$$
\mathcal{T}_{\infty}=\{U \mid X-U \text { is infinite or empty or all of } X\}
$$

a topology on $X$ ?\\
4. (a) If $\left\{\mathcal{T}_{\alpha}\right\}$ is a family of topologies on $X$, show that $\bigcap \mathcal{T}_{\alpha}$ is a topology on $X$. Is $\bigcup \mathcal{T}_{\alpha}$ a topology on $X$ ?\\
(b) Let $\left\{\mathcal{T}_{\alpha}\right\}$ be a family of topologies on $X$. Show that there is a unique smallest topology on $X$ containing all the collections $\mathcal{T}_{\alpha}$, and a unique largest topology contained in all $\mathcal{T}_{\alpha}$.\\
(c) If $X=\{a, b, c\}$, let

$$
\mathcal{T}_{1}=\{\varnothing, X,\{a\},\{a, b\}\} \quad \text { and } \quad \mathcal{T}_{2}=\{\varnothing, X,\{a\},\{b, c\}\}
$$

Find the smallest topology containing $\mathcal{T}_{1}$ and $\mathcal{T}_{2}$, and the largest topology contained in $\mathcal{T}_{1}$ and $\mathcal{T}_{2}$.\\
5. Show that if $\mathcal{A}$ is a basis for a topology on $X$, then the topology generated by $\mathscr{A}$ equals the intersection of all topologies on $X$ that contain $\mathcal{A}$. Prove the same if $\mathcal{A}$ is a subbasis.\\
6. Show that the topologies of $\mathbb{R}_{\ell}$ and $\mathbb{R}_{K}$ are not comparable.\\
7. Consider the following topologies on $\mathbb{R}$ :

$$
\begin{aligned}
& \mathcal{T}_{1}=\text { the standard topology, } \\
& \mathcal{T}_{2}=\text { the topology of } \mathbb{R}_{K}, \\
& \mathcal{T}_{3}=\text { the finite complement topology, } \\
& \mathcal{T}_{4}=\text { the upper limit topology, having all sets }(a, b] \text { as basis, } \\
& \mathcal{T}_{5}=\text { the topology having all sets }(-\infty, a)=\{x \mid x<a\} \text { as basis. }
\end{aligned}
$$

Determine, for each of these topologies, which of the others it contains.\\
8. (a) Apply Lemma 13.2 to show that the countable collection

$$
\mathscr{B}=\{(a, b) \mid a<b, a \text { and } b \text { rational }\}
$$

is a basis that generates the standard topology on $\mathbb{R}$.\\
(b) Show that the collection

$$
\mathcal{C}=\{[a, b) \mid a<b, a \text { and } b \text { rational }\}
$$

is a basis that generates a topology different from the lower limit topology on $\mathbb{R}$.

\section*{§14 The Order Topology}
If $X$ is a simply ordered set, there is a standard topology for $X$, defined using the order relation. It is called the order topology; in this section, we consider it and study some of its properties.

Suppose that $X$ is a set having a simple order relation $<$. Given elements $a$ and $b$ of $X$ such that $a<b$, there are four subsets of $X$ that are called the intervals determined by $a$ and $b$. They are the following:

$$
\begin{aligned}
& (a, b)=\{x \mid a<x<b\}, \\
& (a, b]=\{x \mid a<x \leq b\}, \\
& {[a, b)=\{x \mid a \leq x<b\},} \\
& {[a, b]=\{x \mid a \leq x \leq b\} .}
\end{aligned}
$$

The notation used here is familiar to you already in the case where $X$ is the real line, but these are intervals in an arbitrary ordered set. A set of the first type is called an open interval in $X$, a set of the last type is called a closed interval in $X$, and sets of the second and third types are called half-open intervals. The use of the term "open" in this connection suggests that open intervals in $X$ should turn out to be open sets when we put a topology on $X$. And so they will.

Definition. Let $X$ be a set with a simple order relation; assume $X$ has more than one element. Let $\mathscr{B}$ be the collection of all sets of the following types:\\
(1) All open intervals $(a, b)$ in $X$.\\
(2) All intervals of the form $\left[a_{0}, b\right)$, where $a_{0}$ is the smallest element (if any) of $X$.\\
(3) All intervals of the form $\left(a, b_{0}\right]$, where $b_{0}$ is the largest element (if any) of $X$. The collection $\mathfrak{B}$ is a basis for a topology on $X$, which is called the order topology.

If $X$ has no smallest element, there are no sets of type (2), and if $X$ has no largest element, there are no sets of type (3).

One has to check that $\mathscr{B}$ satisfies the requirements for a basis. First, note that every element $x$ of $X$ lies in at least one element of $B$ : The smallest element (if any) lies in all sets of type (2), the largest element (if any) lies in all sets of type (3), and every other element lies in a set of type (1). Second, note that the intersection of any two sets of the preceding types is again a set of one of these types, or is empty. Several cases need to be checked; we leave it to you.

EXAMPLE 1. The standard topology on $\mathbb{R}$, as defined in the preceding section, is just the order topology derived from the usual order on $\mathbb{R}$.

EXAMPLE 2. Consider the set $\mathbb{R} \times \mathbb{R}$ in the dictionary order; we shall denote the general element of $\mathbb{R} \times \mathbb{R}$ by $x \times y$, to avoid difficulty with notation. The set $\mathbb{R} \times \mathbb{R}$ has neither a largest nor a smallest element, so the order topology on $\mathbb{R} \times \mathbb{R}$ has as basis the collection of all open intervals of the form $(a \times b, c \times d)$ for $a<c$, and for $a=c$ and $b<d$. These two types of intervals are indicated in Figure 14.1. The subcollection consisting of only intervals of the second type is also a basis for the order topology on $\mathbb{R} \times \mathbb{R}$, as you can check.

\begin{figure}[h]
\begin{center}
  \includegraphics[width=\textwidth]{2025_11_21_a8ffce9f36674b61ee7eg-011}
\captionsetup{labelformat=empty}
\caption{Figure 14.1}
\end{center}
\end{figure}

EXAMPLE 3. The positive integers $\mathbb{Z}_{+}$form an ordered set with a smallest element. The order topology on $\mathbb{Z}_{+}$is the discrete topology, for every one-point set is open: If $n>1$, then the one-point set $\{n\}=(n-1, n+1)$ is a basis element; and if $n=1$, the one-point set $\{1\}=[1,2)$ is a basis element.

EXAMPLE 4. The set $X=\{1,2\} \times \mathbb{Z}_{+}$in the dictionary order is another example of an ordered set with a smallest element. Denoting $1 \times n$ by $a_{n}$ and $2 \times n$ by $b_{n}$, we can represent $X$ by

$$
a_{1}, a_{2}, \ldots ; b_{1}, b_{2}, \ldots .
$$

The order topology on $X$ is not the discrete topology. Most one-point sets are open, but there is an exception-the one-point set $\left\{b_{1}\right\}$. Any open set containing $b_{1}$ must contain a basis element about $b_{1}$ (by definition), and any basis element containing $b_{1}$ contains points of the $a_{i}$ sequence.

Definition. If $X$ is an ordered set, and $a$ is an element of $X$, there are four subsets of $X$ that are called the rays determined by $a$. They are the following:

$$
\begin{aligned}
& (a,+\infty)=\{x \mid x>a\}, \\
& (-\infty, a)=\{x \mid x<a\}, \\
& {[a,+\infty)=\{x \mid x \geq a\},} \\
& (-\infty, a]=\{x \mid x \leq a\} .
\end{aligned}
$$

Sets of the first two types are called open rays, and sets of the last two types are called closed rays.

The use of the term "open" suggests that open rays in $X$ are open sets in the order topology. And so they are. Consider, for example, the ray ( $a,+\infty$ ). If $X$ has a largest element $b_{0}$, then ( $a,+\infty$ ) equals the basis element ( $a, b_{0}$ ]. If $X$ has no largest element, then ( $a,+\infty$ ) equals the union of all basis elements of the form ( $a, x$ ), for $x>a$. In either case, $(a,+\infty)$ is open. A similar argument applies to the ray $(-\infty, a)$.

The open rays, in fact, form a subbasis for the order topology on $X$, as we now show. Because the open rays are open in the order topology, the topology they generate is contained in the order topology. On the other hand, every basis element for the order topology equals a finite intersection of open rays; the interval ( $a, b$ ) equals the intersection of $(-\infty, b)$ and $(a,+\infty)$, while $\left[a_{0}, b\right)$ and $\left(a, b_{0}\right]$, if they exist, are themselves open rays. Hence the topology generated by the open rays contains the order topology.

\section*{§15 The Product Topology on $X \times Y$}
If $X$ and $Y$ are topological spaces, there is a standard way of defining a topology on the cartesian product $X \times Y$. We consider this topology now and study some of its properties.

Definition. Let $X$ and $Y$ be topological spaces. The product topology on $X \times Y$ is the topology having as basis the collection $\mathscr{B}$ of all sets of the form $U \times V$, where $U$ is an open subset of $X$ and $V$ is an open subset of $Y$.

Let us check that $\mathscr{B}$ is a basis. The first condition is trivial, since $X \times Y$ is itself a basis element. The second condition is almost as easy, since the intersection of any two basis elements $U_{1} \times V_{1}$ and $U_{2} \times V_{2}$ is another basis element. For

$$
\left(U_{1} \times V_{1}\right) \cap\left(U_{2} \times V_{2}\right)=\left(U_{1} \cap U_{2}\right) \times\left(V_{1} \cap V_{2}\right)
$$

and the latter set is a basis element because $U_{1} \cap U_{2}$ and $V_{1} \cap V_{2}$ are open in $X$ and $Y$, respectively. See Figure 15.1.

Note that the collection $\mathscr{B}$ is not a topology on $X \times Y$. The union of the two rectangles pictured in Figure 15.1, for instance, is not a product of two sets, so it cannot belong to $\mathscr{B}$; however, it is open in $X \times Y$.

Each time we introduce a new concept, we shall try to relate it to the concepts that have been previously introduced. In the present case, we ask: What can one say if the topologies on $X$ and $Y$ are given by bases? The answer is as follows:

Theorem 15.1. If $B$ is a basis for the topology of $X$ and $\mathcal{C}$ is a basis for the topology of $Y$, then the collection

$$
\mathcal{D}=\{B \times C \mid B \in \mathscr{B} \text { and } C \in \mathcal{C}\}
$$

is a basis for the topology of $X \times Y$.

\begin{figure}[h]
\begin{center}
  \includegraphics[width=\textwidth]{2025_11_21_a8ffce9f36674b61ee7eg-013}
\captionsetup{labelformat=empty}
\caption{Figure 15.1}
\end{center}
\end{figure}

Proof. We apply Lemma 13.2. Given an open set $W$ of $X \times Y$ and a point $x \times y$ of $W$, by definition of the product topology there is a basis element $U \times V$ such that $x \times y \in U \times V \subset W$. Because $\mathscr{B}$ and $\mathcal{C}$ are bases for $X$ and $Y$, respectively, we can choose an element $B$ of $\mathscr{B}$ such that $x \in B \subset U$, and an element $C$ of $\mathcal{C}$ such that $y \in C \subset V$. Then $x \times y \in B \times C \subset W$. Thus the collection $D$ meets the criterion of Lemma 13.2, so $\mathcal{D}$ is a basis for $X \times Y$.

EXAMPLE 1. We have a standard topology on $\mathbb{R}$ : the order topology. The product of this topology with itself is called the standard topology on $\mathbb{R} \times \mathbb{R}=\mathbb{R}^{2}$. It has as basis the collection of all products of open sets of $\mathbb{R}$, but the theorem just proved tells us that the much smaller collection of all products $(a, b) \times(c, d)$ of open intervals in $\mathbb{R}$ will also serve as a basis for the topology of $\mathbb{R}^{2}$. Each such set can be pictured as the interior of a rectangle in $\mathbb{R}^{2}$. Thus the standard topology on $\mathbb{R}^{2}$ is just the one we considered in Example 2 of §13.

It is sometimes useful to express the product topology in terms of a subbasis. To do this, we first define certain functions called projections.

Definition. Let $\pi_{1}: X \times Y \rightarrow X$ be defined by the equation

$$
\pi_{1}(x, y)=x
$$

let $\pi_{2}: X \times Y \rightarrow Y$ be defined by the equation

$$
\pi_{2}(x, y)=y .
$$

The maps $\pi_{1}$ and $\pi_{2}$ are called the projections of $X \times Y$ onto its first and second factors, respectively.

We use the word "onto" because $\pi_{1}$ and $\pi_{2}$ are surjective (unless one of the spaces $X$ or $Y$ happens to be empty, in which case $X \times Y$ is empty and our whole discussion is empty as well!).

If $U$ is an open subset of $X$, then the set $\pi_{1}^{-1}(U)$ is precisely the set $U \times Y$, which is open in $X \times Y$. Similarly, if $V$ is open in $Y$, then

$$
\pi_{2}^{-1}(V)=X \times V,
$$

which is also open in $X \times Y$. The intersection of these two sets is the set $U \times V$, as indicated in Figure 15.2. This fact leads to the following theorem:

Theorem 15.2. The collection

$$
S=\left\{\pi_{1}^{-1}(U) \mid U \text { open in } X\right\} \cup\left\{\pi_{2}^{-1}(V) \mid V \text { open in } Y\right\}
$$

is a subbasis for the product topology on $X \times Y$.

\begin{figure}[h]
\begin{center}
  \includegraphics[width=\textwidth]{2025_11_21_a8ffce9f36674b61ee7eg-014}
\captionsetup{labelformat=empty}
\caption{Figure 15.2}
\end{center}
\end{figure}

Proof. Let $\mathcal{T}$ denote the product topology on $X \times Y$; let $\mathcal{T}^{\prime}$ be the topology generated by $S$. Because every element of $S$ belongs to $\mathcal{T}$, so do arbitrary unions of finite intersections of elements of $S$. Thus $\mathcal{T}^{\prime} \subset \mathcal{T}$. On the other hand, every basis element $U \times V$ for the topology $\mathcal{T}$ is a finite intersection of elements of $S$, since

$$
U \times V=\pi_{1}^{-1}(U) \cap \pi_{2}^{-1}(V) .
$$

Therefore, $U \times V$ belongs to $\mathcal{T}^{\prime}$, so that $\mathcal{T} \subset \mathcal{T}^{\prime}$ as well. \(\square\)

\section*{§16 The Subspace Topology}
Definition. Let $X$ be a topological space with topology $\mathcal{T}$. If $Y$ is a subset of $X$, the collection

$$
\mathcal{T}_{Y}=\{Y \cap U \mid U \in \mathcal{T}\}
$$

is a topology on $Y$, called the subspace topology. With this topology, $Y$ is called a subspace of $X$; its open sets consist of all intersections of open sets of $X$ with $Y$.

It is easy to see that $\mathcal{T}_{Y}$ is a topology. It contains $\varnothing$ and $Y$ because

$$
\varnothing=Y \cap \varnothing \quad \text { and } \quad Y=Y \cap X,
$$

where $\varnothing$ and $X$ are elements of $\mathcal{T}$. The fact that it is closed under finite intersections and arbitrary unions follows from the equations

$$
\begin{aligned}
\left(U_{1} \cap Y\right) \cap \cdots \cap\left(U_{n} \cap Y\right) & =\left(U_{1} \cap \cdots \cap U_{n}\right) \cap Y, \\
\bigcup_{\alpha \in J}\left(U_{\alpha} \cap Y\right) & =\left(\bigcup_{\alpha \in J} U_{\alpha}\right) \cap Y .
\end{aligned}
$$

Lemma 16.1. If $B$ is a basis for the topology of $X$ then the collection

$$
\mathscr{B}_{Y}=\{B \cap Y \mid B \in \mathscr{B}\}
$$

is a basis for the subspace topology on $Y$.\\
Proof. Given $U$ open in $X$ and given $y \in U \cap Y$, we can choose an element $B$ of $\mathscr{B}$ such that $y \in B \subset U$. Then $y \in B \cap Y \subset U \cap Y$. It follows from Lemma 13.2 that $\mathcal{B}_{Y}$ is a basis for the subspace topology on $Y$.

When dealing with a space $X$ and a subspace $Y$, one needs to be careful when one uses the term "open set". Does one mean an element of the topology of $Y$ or an element of the topology of $X$ ? We make the following definition : If $Y$ is a subspace of $X$, we say that a set $U$ is open in $Y$ (or open relative to $Y$ ) if it belongs to the topology of $Y$; this implies in particular that it is a subset of $Y$. We say that $U$ is open in $X$ if it belongs to the topology of $X$.

There is a special situation in which every set open in $Y$ is also open in $X$ :

Lemma 16.2. Let $Y$ be a subspace of $X$. If $U$ is open in $Y$ and $Y$ is open in $X$, then $U$ is open in $X$.

Proof. Since $U$ is open in $Y, U=Y \cap V$ for some set $V$ open in $X$. Since $Y$ and $V$ are both open in $X$, so is $Y \cap V$.

Now let us explore the relation between the subspace topology and the order and product topologies. For product topologies, the result is what one might expect; for order topologies, it is not.

Theorem 16.3. If $A$ is a subspace of $X$ and $B$ is a subspace of $Y$, then the product topology on $A \times B$ is the same as the topology $A \times B$ inherits as a subspace of $X \times Y$.

Proof. The set $U \times V$ is the general basis element for $X \times Y$, where $U$ is open in $X$ and $V$ is open in $Y$. Therefore, $(U \times V) \cap(A \times B)$ is the general basis element for the subspace topology on $A \times B$. Now

$$
(U \times V) \cap(A \times B)=(U \cap A) \times(V \cap B) .
$$

Since $U \cap A$ and $V \cap B$ are the general open sets for the subspace topologies on $A$ and $B$, respectively, the set ( $U \cap A$ ) $\times(V \cap B)$ is the general basis element for the product topology on $A \times B$.

The conclusion we draw is that the bases for the subspace topology on $A \times B$ and for the product topology on $A \times B$ are the same. Hence the topologies are the same. \(\square\)

Now let $X$ be an ordered set in the order topology, and let $Y$ be a subset of $X$. The order relation on $X$, when restricted to $Y$, makes $Y$ into an ordered set. However, the resulting order topology on $Y$ need not be the same as the topology that $Y$ inherits as a subspace of $X$. We give one example where the subspace and order topologies on $Y$ agree, and two examples where they do not.

Example 1. Consider the subset $Y=[0,1]$ of the real line $\mathbb{R}$, in the subspace topology. The subspace topology has as basis all sets of the form $(a, b) \cap Y$, where $(a, b)$ is an open interval in $\mathbb{R}$. Such a set is of one of the following types:

$$
(a, b) \cap Y=\left\{\begin{array}{ll}
(a, b) & \text { if } a \text { and } b \text { are in } Y, \\
{[0, b)} & \text { if only } b \text { is in } Y, \\
(a, 1] & \text { if only } a \text { is in } Y, \\
Y \text { or } \varnothing & \text { if neither } a \text { nor } b \text { is in } Y .
\end{array} .\right.
$$

By definition, each of these sets is open in $Y$. But sets of the second and third types are not open in the larger space $\mathbb{R}$.

Note that these sets form a basis for the order topology on $Y$. Thus, we see that in the case of the set $Y=[0,1]$, its subspace topology (as a subspace of $\mathbb{R}$ ) and its order topology are the same.

EXAMPLE 2. Let $Y$ be the subset $[0,1) \cup\{2\}$ of $\mathbb{R}$. In the subspace topology on $Y$ the one-point set $\{2\}$ is open, because it is the intersection of the open set $\left(\frac{3}{2}, \frac{5}{2}\right)$ with $Y$. But in the order topology on $Y$, the set $\{2\}$ is not open. Any basis element for the order topology on $Y$ that contains 2 is of the form

$$
\{x \mid x \in Y \text { and } a<x \leq 2\}
$$

for some $a \in Y$; such a set necessarily contains points of $Y$ less than 2 .\\
Example 3. Let $I=[0,1]$. The dictionary order on $I \times I$ is just the restriction to $I \times I$ of the dictionary order on the plane $\mathbb{R} \times \mathbb{R}$. However, the dictionary order topology on $I \times I$ is not the same as the subspace topology on $I \times I$ obtained from the dictionary order topology on $\mathbb{R} \times \mathbb{R}$ ! For example, the set $\{1 / 2\} \times(1 / 2,1]$ is open in $I \times I$ in the subspace topology, but not in the order topology, as you can check. See Figure 16.1.

The set $I \times I$ in the dictionary order topology will be called the ordered square, and denoted by $I_{o}^{2}$.

The anomaly illustrated in Examples 2 and 3 does not occur for intervals or rays in an ordered set $X$. This we now prove.

Given an ordered set $X$, let us say that a subset $Y$ of $X$ is convex in $X$ if for each pair of points $a<b$ of $Y$, the entire interval $(a, b)$ of points of $X$ lies in $Y$. Note that intervals and rays in $X$ are convex in $X$.

\begin{figure}[h]
\begin{center}
  \includegraphics[width=\textwidth]{2025_11_21_a8ffce9f36674b61ee7eg-017}
\captionsetup{labelformat=empty}
\caption{Figure 16.1}
\end{center}
\end{figure}

Theorem 16.4. Let $X$ be an ordered set in the order topology; let $Y$ be a subset of $X$ that is convex in $X$. Then the order topology on $Y$ is the same as the topology $Y$ inherits as a subspace of $X$.\\
Proof. Consider the ray ( $a,+\infty$ ) in $X$. What is its intersection with $Y$ ? If $a \in Y$, then

$$
(a,+\infty) \cap Y=\{x \mid x \in Y \text { and } x>a\} ;
$$

this is an open ray of the ordered set $Y$. If $a \notin Y$, then $a$ is either a lower bound on $Y$ or an upper bound on $Y$, since $Y$ is convex. In the former case, the set $(a,+\infty) \cap Y$ equals all of $Y$; in the latter case, it is empty.

A similar remark shows that the intersection of the ray $(-\infty, a)$ with $Y$ is either an open ray of $Y$, or $Y$ itself, or empty. Since the sets $(a,+\infty) \cap Y$ and $(-\infty, a) \cap Y$ form a subbasis for the subspace topology on $Y$, and since each is open in the order topology, the order topology contains the subspace topology.

To prove the reverse, note that any open ray of $Y$ equals the intersection of an open ray of $X$ with $Y$, so it is open in the subspace topology on $Y$. Since the open rays of $Y$ are a subbasis for the order topology on $Y$, this topology is contained in the subspace topology.

To avoid ambiguity, let us agree that whenever $X$ is an ordered set in the order topology and $Y$ is a subset of $X$, we shall assume that $Y$ is given the subspace topology unless we specifically state otherwise. If $Y$ is convex in $X$, this is the same as the order topology on $Y$; otherwise, it may not be.

\section*{Exercises}
\begin{enumerate}
  \item Show that if $Y$ is a subspace of $X$, and $A$ is a subset of $Y$, then the topology $A$\\
inherits as a subspace of $Y$ is the same as the topology it inherits as a subspace of $X$.
  \item If $\mathcal{T}$ and $\mathcal{T}^{\prime}$ are topologies on $X$ and $\mathcal{T}^{\prime}$ is strictly finer than $\mathcal{T}$, what can you say about the corresponding subspace topologies on the subset $Y$ of $X$ ?
  \item Consider the set $Y=[-1,1]$ as a subspace of $\mathbb{R}$. Which of the following sets are open in $Y$ ? Which are open in $\mathbb{R}$ ?
\end{enumerate}

$$
\begin{aligned}
& A=\left\{x\left|\frac{1}{2}<|x|<1\right\}\right. \\
& B=\left\{x\left|\frac{1}{2}<|x| \leq 1\right\}\right. \\
& C=\left\{x\left|\frac{1}{2} \leq|x|<1\right\}\right. \\
& D=\left\{x\left|\frac{1}{2} \leq|x| \leq 1\right\}\right. \\
& E=\left\{x\left|0<|x|<1 \text { and } 1 / x \notin \mathbb{Z}_{+}\right\}\right.
\end{aligned}
$$

\begin{enumerate}
  \setcounter{enumi}{3}
  \item A map $f: X \rightarrow Y$ is said to be an open map if for every open set $U$ of $X$, the set $f(U)$ is open in $Y$. Show that $\pi_{1}: X \times Y \rightarrow X$ and $\pi_{2}: X \times Y \rightarrow Y$ are open maps.
  \item Let $X$ and $X^{\prime}$ denote a single set in the topologies $\mathcal{T}$ and $\mathcal{T}^{\prime}$, respectively; let $Y$ and $Y^{\prime}$ denote a single set in the topologies $\mathcal{U}$ and $\mathcal{U}^{\prime}$, respectively. Assume these sets are nonempty.\\
(a) Show that if $\mathcal{T}^{\prime} \supset \mathcal{T}$ and $\mathcal{U}^{\prime} \supset \mathcal{U}$, then the product topology on $X^{\prime} \times Y^{\prime}$ is finer than the product topology on $X \times Y$.\\
(b) Does the converse of (a) hold? Justify your answer.
  \item Show that the countable collection
\end{enumerate}

$$
\{(a, b) \times(c, d) \mid a<b \text { and } c<d, \text { and } a, b, c, d \text { are rational }\}
$$

is a basis for $\mathbb{R}^{2}$.\\
7. Let $X$ be an ordered set. If $Y$ is a proper subset of $X$ that is convex in $X$, does it follow that $Y$ is an interval or a ray in $X$ ?\\
8. If $L$ is a straight line in the plane, describe the topology $L$ inherits as a subspace of $\mathbb{R}_{\ell} \times \mathbb{R}$ and as a subspace of $\mathbb{R}_{\ell} \times \mathbb{R}_{\ell}$. In each case it is a familiar topology.\\
9. Show that the dictionary order topology on the set $\mathbb{R} \times \mathbb{R}$ is the same as the product topology $\mathbb{R}_{d} \times \mathbb{R}$, where $\mathbb{R}_{d}$ denotes $\mathbb{R}$ in the discrete topology. Compare this topology with the standard topology on $\mathbb{R}^{2}$.\\
10. Let $I=[0,1]$. Compare the product topology on $I \times I$, the dictionary order topology on $I \times I$, and the topology $I \times I$ inherits as a subspace of $\mathbb{R} \times \mathbb{R}$ in the dictionary order topology.

\section*{§17 Closed Sets and Limit Points}
Now that we have a few examples at hand, we can introduce some of the basic concepts associated with topological spaces. In this section, we treat the notions of closed set,\\
closure of a set, and limit point. These lead naturally to consideration of a certain axiom for topological spaces called the Hausdorff axiom.

\section*{Closed Sets}
A subset $A$ of a topological space $X$ is said to be closed if the set $X-A$ is open.\\
EXAMPLE 1. The subset $[a, b]$ of $\mathbb{R}$ is closed because its complement

$$
\mathbb{R}-[a, b]=(-\infty, a) \cup(b,+\infty)
$$

is open. Similarly, $[a,+\infty)$ is closed, because its complement $(-\infty, a)$ is open. These facts justify our use of the terms "closed interval" and "closed ray." The subset $[a, b)$ of $\mathbb{R}$ is neither open nor closed.

Example 2. In the plane $\mathbb{R}^{2}$, the set

$$
\{x \times y \mid x \geq 0 \text { and } y \geq 0\}
$$

is closed, because its complement is the union of the two sets

$$
(-\infty, 0) \times \mathbb{R} \quad \text { and } \quad \mathbb{R} \times(-\infty, 0)
$$

each of which is a product of open sets of $\mathbb{R}$ and is, therefore, open in $\mathbb{R}^{2}$.

EXAMPLE 3. In the finite complement topology on a set $X$, the closed sets consist of $X$ itself and all finite subsets of $X$.

Example 4. In the discrete topology on the set $X$, every set is open; it follows that every set is closed as well.

Example 5. Consider the following subset of the real line:

$$
Y=[0,1] \cup(2,3),
$$

in the subspace topology. In this space, the set $[0,1]$ is open, since it is the intersection of the open set $\left(-\frac{1}{2}, \frac{3}{2}\right)$ of $\mathbb{R}$ with $Y$. Similarly, $(2,3)$ is open as a subset of $Y$; it is even open as a subset of $\mathbb{R}$. Since $[0,1]$ and $(2,3)$ are complements in $Y$ of each other, we conclude that both $[0,1]$ and $(2,3)$ are closed as subsets of $Y$.

These examples suggest that an answer to the mathematician's riddle: "How is a set different from a door?" should be: "A door must be either open or closed, and cannot be both, while a set can be open, or closed, or both, or neither!"

The collection of closed subsets of a space $X$ has properties similar to those satisfied by the collection of open subsets of $X$ :

Theorem 17.1. Let $X$ be a topological space. Then the following conditions hold:\\
(1) $\varnothing$ and $X$ are closed.\\
(2) Arbitrary intersections of closed sets are closed.\\
(3) Finite unions of closed sets are closed.

Proof. (1) $\varnothing$ and $X$ are closed because they are the complements of the open sets $X$ and $\varnothing$, respectively.\\
(2) Given a collection of closed sets $\left\{A_{\alpha}\right\}_{\alpha \in J}$, we apply DeMorgan's law,

$$
X-\bigcap_{\alpha \in J} A_{\alpha}=\bigcup_{\alpha \in J}\left(X-A_{\alpha}\right) .
$$

Since the sets $X-A_{\alpha}$ are open by definition, the right side of this equation represents an arbitrary union of open sets, and is thus open. Therefore, $\bigcap A_{\alpha}$ is closed.\\
(3) Similarly, if $A_{i}$ is closed for $i=1, \ldots, n$, consider the equation

$$
X-\bigcup_{i=1}^{n} A_{i}=\bigcap_{i=1}^{n}\left(X-A_{i}\right) .
$$

The set on the right side of this equation is a finite intersection of open sets and is therefore open. Hence $\bigcup A_{i}$ is closed. \(\square\)

Instead of using open sets, one could just as well specify a topology on a space by giving a collection of sets (to be called "closed sets") satisfying the three properties of this theorem. One could then define open sets as the complements of closed sets and proceed just as before. This procedure has no particular advantage over the one we have adopted, and most mathematicians prefer to use open sets to define topologies.

Now when dealing with subspaces, one needs to be careful in using the term "closed set." If $Y$ is a subspace of $X$, we say that a set $A$ is closed in $Y$ if $A$ is a subset of $Y$ and if $A$ is closed in the subspace topology of $Y$ (that is, if $Y-A$ is open in $Y$ ). We have the following theorem:

Theorem 17.2. Let $Y$ be a subspace of $X$. Then a set $A$ is closed in $Y$ if and only if it equals the intersection of a closed set of $X$ with $Y$.

Proof. Assume that $A=C \cap Y$, where $C$ is closed in $X$. (See Figure 17.1.) Then $X-C$ is open in $X$, so that $(X-C) \cap Y$ is open in $Y$, by definition of the subspace topology. But $(X-C) \cap Y=Y-A$. Hence $Y-A$ is open in $Y$, so that $A$ is closed in $Y$. Conversely, assume that $A$ is closed in $Y$. (See Figure 17.2.) Then $Y-A$ is open in $Y$, so that by definition it equals the intersection of an open set $U$ of $X$ with $Y$. The set $X-U$ is closed in $X$, and $A=Y \cap(X-U)$, so that $A$ equals the intersection of a closed set of $X$ with $Y$, as desired. \(\square\)

A set $A$ that is closed in the subspace $Y$ may or may not be closed in the larger space $X$. As was the case with open sets, there is a criterion for $A$ to be closed in $X$; we leave the proof to you:

\begin{figure}[h]
\begin{center}
  \includegraphics[width=\textwidth]{2025_11_21_a8ffce9f36674b61ee7eg-021(1)}
\captionsetup{labelformat=empty}
\caption{Figure 17.1}
\end{center}
\end{figure}

\begin{figure}[h]
\begin{center}
  \includegraphics[width=\textwidth]{2025_11_21_a8ffce9f36674b61ee7eg-021}
\captionsetup{labelformat=empty}
\caption{Figure 17.2}
\end{center}
\end{figure}

Theorem 17.3. Let $Y$ be a subspace of $X$. If $A$ is closed in $Y$ and $Y$ is closed in $X$, then $A$ is closed in $X$.

\section*{Closure and Interior of a Set}
Given a subset $A$ of a topological space $X$, the interior of $A$ is defined as the union of all open sets contained in $A$, and the closure of $A$ is defined as the intersection of all closed sets containing $A$.

The interior of $A$ is denoted by $\operatorname{Int} A$ and the closure of $A$ is denoted by $\mathrm{Cl} A$ or by $\bar{A}$. Obviously $\operatorname{Int} A$ is an open set and $\bar{A}$ is a closed set; furthermore,

$$
\operatorname{Int} A \subset A \subset \bar{A} .
$$

If $A$ is open, $A=\operatorname{Int} A$; while if $A$ is closed, $A=\bar{A}$.\\
We shall not make much use of the interior of a set, but the closure of a set will be quite important.

When dealing with a topological space $X$ and a subspace $Y$, one needs to exercise care in taking closures of sets. If $A$ is a subset of $Y$, the closure of $A$ in $Y$ and the closure of $A$ in $X$ will in general be different. In such a situation, we reserve the notation $\bar{A}$ to stand for the closure of $A$ in $X$. The closure of $A$ in $Y$ can be expressed in terms of $\bar{A}$, as the following theorem shows:

Theorem 17.4. Let $Y$ be a subspace of $X$; let $A$ be a subset of $Y$; let $\bar{A}$ denote the closure of $A$ in $X$. Then the closure of $A$ in $Y$ equals $\bar{A} \cap Y$.

Proof. Let $B$ denote the closure of $A$ in $Y$. The set $\bar{A}$ is closed in $X$, so $\bar{A} \cap Y$ is closed in $Y$ by Theorem 17.2. Since $\bar{A} \cap Y$ contains $A$, and since by definition $B$ equals the intersection of all closed subsets of $Y$ containing $A$, we must have $B \subset(\bar{A} \cap Y)$.

On the other hand, we know that $B$ is closed in $Y$. Hence by Theorem 17.2, $B=C \cap Y$ for some set $C$ closed in $X$. Then $C$ is a closed set of $X$ containing $A$; because $\bar{A}$ is the intersection of all such closed sets, we conclude that $\bar{A} \subset C$. Then $(\bar{A} \cap Y) \subset(C \cap Y)=B$.

The definition of the closure of a set does not give us a convenient way for actually finding the closures of specific sets, since the collection of all closed sets in $X$, like the collection of all open sets, is usually much too big to work with. Another way of describing the closure of a set, useful because it involves only a basis for the topology of $X$, is given in the following theorem.

First let us introduce some convenient terminology. We shall say that a set $A$ intersects a set $B$ if the intersection $A \cap B$ is not empty.

Theorem 17.5. Let $A$ be a subset of the topological space $X$.\\
(a) Then $x \in \bar{A}$ if and only if every open set $U$ containing $x$ intersects $A$.\\
(b) Supposing the topology of $X$ is given by a basis, then $x \in \bar{A}$ if and only if every basis element $B$ containing $x$ intersects $A$.

Proof. Consider the statement in (a). It is a statement of the form $P \Leftrightarrow Q$. Let us transform each implication to its contrapositive, thereby obtaining the logically equivalent statement (not $P$ ) ⇔ (not $Q$ ). Written out, it is the following:\\
$x \notin \bar{A} \Longleftrightarrow$ there exists an open set $U$ containing $x$ that does not intersect $A$.\\
In this form, our theorem is easy to prove. If $x$ is not in $\bar{A}$, the set $U=X-\bar{A}$ is an open set containing $x$ that does not intersect $A$, as desired. Conversely, if there exists an open set $U$ containing $x$ which does not intersect $A$, then $X-U$ is a closed set containing $A$. By definition of the closure $\bar{A}$, the set $X-U$ must contain $\bar{A}$; therefore, $x$ cannot be in $\bar{A}$.

Statement (b) follows readily. If every open set containing $x$ intersects $A$, so does every basis element $B$ containing $x$, because $B$ is an open set. Conversely, if every basis element containing $x$ intersects $A$, so does every open set $U$ containing $x$, because $U$ contains a basis element that contains $x$.

Mathematicians often use some special terminology here. They shorten the statement " $U$ is an open set containing $x$ " to the phrase

\begin{displayquote}
" $U$ is a neighborhood of $x$."
\end{displayquote}

Using this terminology, one can write the first half of the preceding theorem as follows:\\
If $A$ is a subset of the topological space $X$, then $x \in \bar{A}$ if and only if every neighborhood of $x$ intersects $A$.

Example 6. Let $X$ be the real line $\mathbb{R}$. If $A=(0,1]$, then $\bar{A}=[0,1]$, for every neighborhood of 0 intersects $A$, while every point outside $[0,1]$ has a neighborhood disjoint from $A$. Similar arguments apply to the following subsets of $X$ :

If $B=\left\{1 / n \mid n \in \mathbb{Z}_{+}\right\}$, then $\bar{B}=\{0\} \cup B$. If $C=\{0\} \cup(1,2)$, then $\bar{C}=\{0\} \cup[1,2]$. If $\mathbb{Q}$ is the set of rational numbers, then $\mathbb{Q}=\mathbb{R}$. If $\mathbb{Z}_{+}$is the set of positive integers, then $\overline{\mathbb{Z}}_{+}=\mathbb{Z}_{+}$. If $\mathbb{R}_{+}$is the set of positive reals, then the closure of $\mathbb{R}_{+}$is the set $\mathbb{R}_{+} \cup\{0\}$. (This is the reason we introduced the notation $\overline{\mathbb{R}}_{+}$for the set $\mathbb{R}_{+} \cup\{0\}$, back in §2.)

EXAMPLE 7. Consider the subspace $Y=(0,1]$ of the real line $\mathbb{R}$. The set $A=\left(0, \frac{1}{2}\right)$ is a subset of $Y$; its closure in $\mathbb{R}$ is the set $\left[0, \frac{1}{2}\right]$, and its closure in $Y$ is the set $\left[0, \frac{1}{2}\right] \cap Y=$ ( $0, \frac{1}{2}$ ].

Some mathematicians use the term "neighborhood" differently. They say that $A$ is a neighborhood of $x$ if $A$ merely contains an open set containing $x$. We shall not follow this practice.

\section*{Limit Points}
There is yet another way of describing the closure of a set, a way that involves the important concept of limit point, which we consider now.

If $A$ is a subset of the topological space $X$ and if $x$ is a point of $X$, we say that $x$ is a limit point (or "cluster point," or "point of accumulation") of $A$ if every neighborhood of $x$ intersects $A$ in some point other than $x$ itself. Said differently, $x$ is a limit point of $A$ if it belongs to the closure of $A-\{x\}$. The point $x$ may lie in $A$ or not; for this definition it does not matter.

Example 8. Consider the real line $\mathbb{R}$. If $A=(0,1]$, then the point 0 is a limit point of $A$ and so is the point $\frac{1}{2}$. In fact, every point of the interval $[0,1]$ is a limit point of $A$, but no other point of $\mathbb{R}$ is a limit point of $A$.

If $B=\left\{1 / n \mid n \in \mathbb{Z}_{+}\right\}$, then 0 is the only limit point of $B$. Every other point $x$ of $\mathbb{R}$ has a neighborhood that either does not intersect $B$ at all, or it intersects $B$ only in the point $x$ itself. If $C=\{0\} \cup(1,2)$, then the limit points of $C$ are the points of the interval $[1,2]$. If $\mathbb{Q}$ is the set of rational numbers, every point of $\mathbb{R}$ is a limit point of $\mathbb{Q}$. If $\mathbb{Z}_{+}$is the set of positive integers, no point of $\mathbb{R}$ is a limit point of $\mathbb{Z}_{+}$. If $\mathbb{R}_{+}$is the set of positive reals, then every point of $\{0\} \cup \mathbb{R}_{+}$is a limit point of $\mathbb{R}_{+}$.

Comparison of Examples 6 and 8 suggests a relationship between the closure of a set and the limit points of a set. That relationship is given in the following theorem:

Theorem 17.6. Let $A$ be a subset of the topological space $X$; let $A^{\prime}$ be the set of all limit points of $A$. Then

$$
\bar{A}=A \cup A^{\prime} .
$$

Proof. If $x$ is in $A^{\prime}$, every neighborhood of $x$ intersects $A$ (in a point different from $x$ ). Therefore, by Theorem 17.5, $x$ belongs to $\bar{A}$. Hence $A^{\prime} \subset \bar{A}$. Since by definition $A \subset \bar{A}$, it follows that $A \cup A^{\prime} \subset \bar{A}$.

To demonstrate the reverse inclusion, we let $x$ be a point of $\bar{A}$ and show that $x \in A \cup A^{\prime}$. If $x$ happens to lie in $A$, it is trivial that $x \in A \cup A^{\prime}$; suppose that $x$ does not lie in $A$. Since $x \in \bar{A}$, we know that every neighborhood $U$ of $x$ intersects $A$; because $x \notin A$, the set $U$ must intersect $A$ in a point different from $x$. Then $x \in A^{\prime}$, so that $x \in A \cup A^{\prime}$, as desired.

Corollary 17.7. A subset of a topological space is closed if and only if it contains all its limit points.

Proof. The set $A$ is closed if and only if $A=\bar{A}$, and the latter holds if and only if $A^{\prime} \subset A$.

\section*{Hausdorff Spaces}
One's experience with open and closed sets and limit points in the real line and the plane can be misleading when one considers more general topological spaces. For example, in the spaces $\mathbb{R}$ and $\mathbb{R}^{2}$, each one-point set $\left\{x_{0}\right\}$ is closed. This fact is easily proved; every point different from $x_{0}$ has a neighborhood not intersecting $\left\{x_{0}\right\}$, so that $\left\{x_{0}\right\}$ is its own closure. But this fact is not true for arbitrary topological spaces. Consider the topology on the three-point set $\{a, b, c\}$ indicated in Figure 17.3. In this space, the one-point set $\{b\}$ is not closed, for its complement is not open.

\begin{figure}[h]
\begin{center}
  \includegraphics[width=\textwidth]{2025_11_21_a8ffce9f36674b61ee7eg-024}
\captionsetup{labelformat=empty}
\caption{Figure 17.3}
\end{center}
\end{figure}

Similarly, one's experience with the properties of convergent sequences in $\mathbb{R}$ and $\mathbb{R}^{2}$ can be misleading when one deals with more general topological spaces. In an arbitrary topological space, one says that a sequence $x_{1}, x_{2}, \ldots$ of points of the space $X$ converges to the point $x$ of $X$ provided that, corresponding to each neighborhood $U$ of $x$, there is a positive integer $N$ such that $x_{n} \in U$ for all $n \geq N$. In $\mathbb{R}$ and $\mathbb{R}^{2}$, a sequence cannot converge to more than one point, but in an arbitrary space, it can. In the space indicated in Figure 17.3, for example, the sequence defined by setting $x_{n}=b$ for all $n$ converges not only to the point $b$, but also to the point $a$ and to the point $c$ !

Topologies in which one-point sets are not closed, or in which sequences can converge to more than one point, are considered by many mathematicians to be somewhat strange. They are not really very interesting, for they seldom occur in other branches of mathematics. And the theorems that one can prove about topological spaces are rather limited if such examples are allowed. Therefore, one often imposes an additional condition that will rule out examples like this one, bringing the class of spaces under consideration closer to those to which one's geometric intuition applies. The condition was suggested by the mathematician Felix Hausdorff, so mathematicians have come to call it by his name.

Definition. A topological space $X$ is called a Hausdorff space if for each pair $x_{1}, x_{2}$ of distinct points of $X$, there exist neighborhoods $U_{1}$, and $U_{2}$ of $x_{1}$ and $x_{2}$, respectively, that are disjoint.

Theorem 17.8. Every finite point set in a Hausdorff space $X$ is closed.\\
Proof. It suffices to show that every one-point set $\left\{x_{0}\right\}$ is closed. If $x$ is a point of $X$ different from $x_{0}$, then $x$ and $x_{0}$ have disjoint neighborhoods $U$ and $V$, respectively. Since $U$ does not intersect $\left\{x_{0}\right\}$, the point $x$ cannot belong to the closure of the set $\left\{x_{0}\right\}$. As a result, the closure of the set $\left\{x_{0}\right\}$ is $\left\{x_{0}\right\}$ itself, so that it is closed.

The condition that finite point sets be closed is in fact weaker than the Hausdorff condition. For example, the real line $\mathbb{R}$ in the finite complement topology is not a Hausdorff space, but it is a space in which finite point sets are closed. The condition that finite point sets be closed has been given a name of its own: it is called the $\boldsymbol{T}_{\mathbf{1}} \boldsymbol{a x}$ iom. (We shall explain the reason for this strange terminology in Chapter 4.) The $T_{1}$ axiom will appear in this book in a few exercises, and in just one theorem, which is the following:

Theorem 17.9. Let $X$ be a space satisfying the $T_{1}$ axiom; let $A$ be a subset of $X$. Then the point $x$ is a limit point of $A$ if and only if every neighborhood of $x$ contains infinitely many points of $A$.

Proof. If every neighborhood of $x$ intersects $A$ in infinitely many points, it certainly intersects $A$ in some point other than $x$ itself, so that $x$ is a limit point of $A$.

Conversely, suppose that $x$ is a limit point of $A$, and suppose some neighborhood $U$ of $x$ intersects $A$ in only finitely many points. Then $U$ also intersects $A-\{x\}$ in finitely many points; let $\left\{x_{1}, \ldots, x_{m}\right\}$ be the points of $U \cap(A-\{x\})$. The set $X-\left\{x_{1}, \ldots, x_{m}\right\}$ is an open set of $X$, since the finite point set $\left\{x_{1}, \ldots, x_{m}\right\}$ is closed; then

$$
U \cap\left(X-\left\{x_{1}, \ldots, x_{m}\right\}\right)
$$

is a neighborhood of $x$ that intersects the set $A-\{x\}$ not at all. This contradicts the assumption that $x$ is a limit point of $A$.

One reason for our lack of interest in the $T_{1}$ axiom is the fact that many of the interesting theorems of topology require not just that axiom, but the full strength of the Hausdorff axiom. Furthermore, most of the spaces that are important to mathematicians are Hausdorff spaces. The following two theorems give some substance to these remarks.

Theorem 17.10. If $X$ is a Hausdorff space, then a sequence of points of $X$ converges to at most one point of $X$.

Proof. Suppose that $x_{n}$ is a sequence of points of $X$ that converges to $x$. If $y \neq x$, let $U$ and $V$ be disjoint neighborhoods of $x$ and $y$, respectively. Since $U$ contains $x_{n}$ for all but finitely many values of $n$, the set $V$ cannot. Therefore, $x_{n}$ cannot converge to $y$.

If the sequence $x_{n}$ of points of the Hausdorff space $X$ converges to the point $x$ of $X$, we often write $x_{n} \rightarrow x$, and we say that $x$ is the limit of the sequence $x_{n}$.

The proof of the following result is left to the exercises.

Theorem 17.11. Every simply ordered set is a Hausdorff space in the order topology. The product of two Hausdorff spaces is a Hausdorff space. A subspace of a Hausdorff space is a Hausdorff space.

The Hausdorff condition is generally considered to be a very mild extra condition to impose on a topological space. Indeed, in a first course in topology some mathematicians go so far as to impose this condition at the outset, refusing to consider spaces that are not Hausdorff spaces. We shall not go this far, but we shall certainly assume the Hausdorff condition whenever it is needed in a proof without having any qualms about limiting seriously the range of applications of the results.

The Hausdorff condition is one of a number of extra conditions one can impose on a topological space. Each time one imposes such a condition, one can prove stronger theorems, but one limits the class of spaces to which the theorems apply. Much of the research that has been done in topology since its beginnings has centered on the problem of finding conditions that will be strong enough to enable one to prove interesting theorems about spaces satisfying those conditions, and yet not so strong that they limit severely the range of applications of the results.

We shall study a number of such conditions in the next two chapters. The Hausdorff condition and the $T_{1}$ axiom are but two of a collection of conditions similar to one another that are called collectively the separation axioms. Other conditions include the countability axioms, and various compactness and connectedness conditions. Some of these are quite stringent requirements, as you will see.

\section*{Exercises}
\begin{enumerate}
  \item Let $\mathcal{C}$ be a collection of subsets of the set $X$. Suppose that $\varnothing$ and $X$ are in $\mathcal{C}$, and that finite unions and arbitrary intersections of elements of $\mathcal{C}$ are in $\mathcal{C}$. Show that the collection
\end{enumerate}

$$
\mathcal{T}=\{X-C \mid C \in \mathcal{C}\}
$$

is a topology on $X$.\\
2. Show that if $A$ is closed in $Y$ and $Y$ is closed in $X$, then $A$ is closed in $X$.\\
3. Show that if $A$ is closed in $X$ and $B$ is closed in $Y$, then $A \times B$ is closed in $X \times Y$.\\
4. Show that if $U$ is open in $X$ and $A$ is closed in $X$, then $U-A$ is open in $X$, and $A-U$ is closed in $X$.\\
5. Let $X$ be an ordered set in the order topology. Show that $\overline{(a, b)} \subset[a, b]$. Under what conditions does equality hold?\\
6. Let $A, B$, and $A_{\alpha}$ denote subsets of a space $X$. Prove the following:\\
(a) If $A \subset B$, then $\bar{A} \subset \bar{B}$.\\
(b) $\overline{A \cup B}=\bar{A} \cup \bar{B}$.\\
(c) $\overline{\bigcup A_{\alpha}} \supset \bigcup \bar{A}_{\alpha}$; give an example where equality fails.\\
7. Criticize the following "proof" that $\overline{\bigcup A_{\alpha}} \subset \bigcup \bar{A}_{\alpha}$ : if $\left\{A_{\alpha}\right\}$ is a collection of sets in $X$ and if $x \in \bigcup A_{\alpha}$, then every neighborhood $U$ of $x$ intersects $\bigcup A_{\alpha}$. Thus $U$ must intersect some $A_{\alpha}$, so that $x$ must belong to the closure of some $A_{\alpha}$. Therefore, $x \in \bigcup \bar{A}_{\alpha}$.\\
8. Let $A, B$, and $A_{\alpha}$ denote subsets of a space $X$. Determine whether the following equations hold; if an equality fails, determine whether one of the inclusions $>$ or $\subset$ holds.\\
(a) $\overline{A \cap B}=\bar{A} \cap \bar{B}$.\\
(b) $\overline{\bigcap A_{\alpha}}=\bigcap \bar{A}_{\alpha}$.\\
(c) $\overline{A-B}=\bar{A}-\bar{B}$.\\
9. Let $A \subset X$ and $B \subset Y$. Show that in the space $X \times Y$,

$$
\overline{A \times B}=\bar{A} \times \bar{B} .
$$

\begin{enumerate}
  \setcounter{enumi}{9}
  \item Show that every order topology is Hausdorff.
  \item Show that the product of two Hausdorff spaces is Hausdorff.
  \item Show that a subspace of a Hausdorff space is Hausdorff.
  \item Show that $X$ is Hausdorff if and only if the diagonal $\Delta=\{x \times x \mid x \in X\}$ is closed in $X \times X$.
  \item In the finite complement topology on $\mathbb{R}$, to what point or points does the sequence $x_{n}=1 / n$ converge?
  \item Show the $T_{1}$ axiom is equivalent to the condition that for each pair of points of $X$, each has a neighborhood not containing the other.
  \item Consider the five topologies on $\mathbb{R}$ given in Exercise 7 of §13.\\
(a) Determine the closure of the set $K=\left\{1 / n \mid n \in \mathbb{Z}_{+}\right\}$under each of these topologies.\\
(b) Which of these topologies satisfy the Hausdorff axiom? the $T_{1}$ axiom?
  \item Consider the lower limit topology on $\mathbb{R}$ and the topology given by the basis $\mathcal{C}$ of Exercise 8 of §13. Determine the closures of the intervals $A=(0, \sqrt{2})$ and $B=(\sqrt{2}, 3)$ in these two topologies.
  \item Determine the closures of the following subsets of the ordered square:
\end{enumerate}

$$
\begin{aligned}
& A=\left\{(1 / n) \times 0 \mid n \in \mathbb{Z}_{+}\right\}, \\
& B=\left\{\left.(1-1 / n) \times \frac{1}{2} \right\rvert\, n \in \mathbb{Z}_{+}\right\}, \\
& C=\{x \times 0 \mid 0<x<1\}, \\
& D=\left\{\left.x \times \frac{1}{2} \right\rvert\, 0<x<1\right\}, \\
& E=\left\{\left.\frac{1}{2} \times y \right\rvert\, 0<y<1\right\} .
\end{aligned}
$$

\begin{enumerate}
  \setcounter{enumi}{18}
  \item If $A \subset X$, we define the boundary of $A$ by the equation
\end{enumerate}

$$
\mathrm{Bd} A=\bar{A} \cap(\overline{X-A})
$$

(a) Show that $\operatorname{Int} A$ and $\mathrm{Bd} A$ are disjoint, and $\bar{A}=\operatorname{Int} A \cup \mathrm{Bd} A$.\\
(b) Show that $\operatorname{Bd} A=\varnothing \Leftrightarrow A$ is both open and closed.\\
(c) Show that $U$ is open $\Leftrightarrow \operatorname{Bd} U=\bar{U}-U$.\\
(d) If $U$ is open, is it true that $U=\operatorname{Int}(\bar{U})$ ? Justify your answer.\\
20. Find the boundary and the interior of each of the following subsets of $\mathbb{R}^{2}$ :\\
(a) $A=\{x \times y \mid y=0\}$\\
(b) $B=\{x \times y \mid x>0$ and $y \neq 0\}$\\
(c) $C=A \cup B$\\
(d) $D=\{x \times y \mid x$ is rational $\}$\\
(e) $E=\left\{x \times y \mid 0<x^{2}-y^{2} \leq 1\right\}$\\
(f) $F=\{x \times y \mid x \neq 0$ and $y \leq 1 / x\}$\\
*21. (Kuratowski) Consider the collection of all subsets $A$ of the topological space $X$. The operations of closure $A \rightarrow \bar{A}$ and complementation $A \rightarrow X-A$ are functions from this collection to itself.\\
(a) Show that starting with a given set $A$, one can form no more than 14 distinct sets by applying these two operations successively.\\
(b) Find a subset $A$ of $\mathbb{R}$ (in its usual topology) for which the maximum of 14 is obtained.

\section*{§18 Continuous Functions}
The concept of continuous function is basic to much of mathematics. Continuous functions on the real line appear in the first pages of any calculus book, and continuous functions in the plane and in space follow not far behind. More general kinds of continuous functions arise as one goes further in mathematics. In this section, we shall formulate a definition of continuity that will include all these as special cases, and we shall study various properties of continuous functions. Many of these properties are direct generalizations of things you learned about continuous functions in calculus and analysis.

\section*{Continuity of a Function}
Let $X$ and $Y$ be topological spaces. A function $f: X \rightarrow Y$ is said to be continuous if for each open subset $V$ of $Y$, the set $f^{-1}(V)$ is an open subset of $X$.

Recall that $f^{-1}(V)$ is the set of all points $x$ of $X$ for which $f(x) \in V$; it is empty if $V$ does not intersect the image set $f(X)$ of $f$.

Continuity of a function depends not only upon the function $f$ itself, but also on the topologies specified for its domain and range. If we wish to emphasize this fact, we can say that $f$ is continuous relative to specific topologies on $X$ and $Y$.

Let us note that if the topology of the range space $Y$ is given by a basis $B$, then to prove continuity of $f$ it suffices to show that the inverse image of every basis element is open: The arbitrary open set $V$ of $Y$ can be written as a union of basis elements

$$
V=\bigcup_{\alpha \in J} B_{\alpha} .
$$

Then

$$
f^{-1}(V)=\bigcup_{\alpha \in J} f^{-1}\left(B_{\alpha}\right),
$$

so that $f^{-1}(V)$ is open if each set $f^{-1}\left(B_{\alpha}\right)$ is open.\\
If the topology on $Y$ is given by a subbasis $S$, to prove continuity of $f$ it will even suffice to show that the inverse image of each subbasis element is open: The arbitrary basis element $B$ for $Y$ can be written as a finite intersection $S_{1} \cap \cdots \cap S_{n}$ of subbasis elements; it follows from the equation

$$
f^{-1}(B)=f^{-1}\left(S_{1}\right) \cap \cdots \cap f^{-1}\left(S_{n}\right)
$$

that the inverse image of every basis element is open.\\
EXAMPLE 1. Let us consider a function like those studied in analysis, a "real-valued function of a real variable,"

$$
f: \mathbb{R} \longrightarrow \mathbb{R}
$$

In analysis, one defines continuity of $f$ via the " $\epsilon-\delta$ definition," a bugaboo over the years for every student of mathematics. As one would expect, the $\epsilon-\delta$ definition and ours are equivalent. To prove that our definition implies the $\epsilon-\delta$ definition, for instance, we proceed as follows:

Given $x_{0}$ in $\mathbb{R}$, and given $\epsilon>0$, the interval $V=\left(f\left(x_{0}\right)-\epsilon, f\left(x_{0}\right)+\epsilon\right)$ is an open set of the range space $\mathbb{R}$. Therefore, $f^{-1}(V)$ is an open set in the domain space $\mathbb{R}$. Because $f^{-1}(V)$ contains the point $x_{0}$, it contains some basis element ( $a, b$ ) about $x_{0}$. We choose $\delta$ to be the smaller of the two numbers $x_{0}-a$ and $b-x_{0}$. Then if $\left|x-x_{0}\right|<\delta$, the point $x$ must be in $(a, b)$, so that $f(x) \in V$, and $\left|f(x)-f\left(x_{0}\right)\right|<\epsilon$, as desired.

Proving that the $\epsilon-\delta$ definition implies our definition is no harder; we leave it to you. We shall return to this example when we study metric spaces.

EXAMPLE 2. In calculus one considers the property of continuity for many kinds of functions. For example, one studies functions of the following types:

$$
\begin{array}{ll}
f: \mathbb{R} \longrightarrow \mathbb{R}^{2} & \text { (curves in the plane) } \\
f: \mathbb{R} \rightarrow \mathbb{R}^{3} & \text { (curves in space) } \\
f: \mathbb{R}^{2} \rightarrow \mathbb{R} & \text { (functions } f(x, y) \text { of two real variables) } \\
f: \mathbb{R}^{3} \rightarrow \mathbb{R} & \text { (functions } f(x, y, z) \text { of three real variables) } \\
f: \mathbb{R}^{2} \rightarrow \mathbb{R}^{2} & \text { (vector fields } \mathbf{v}(x, y) \text { in the plane). }
\end{array}
$$

Each of them has a notion of continuity defined for it. Our general definition of continuity includes all these as special cases; this fact will be a consequence of general theorems we shall prove concerning continuous functions on product spaces and on metric spaces.

EXAMPLE 3. Let $\mathbb{R}$ denote the set of real numbers in its usual topology, and let $\mathbb{R}_{\ell}$ denote the same set in the lower limit topology. Let

$$
f: \mathbb{R} \longrightarrow \mathbb{R}_{\ell}
$$

be the identity function; $f(x)=x$ for every real number $x$. Then $f$ is not a continuous function; the inverse image of the open set $[a, b)$ of $\mathbb{R}_{\ell}$ equals itself, which is not open in $\mathbb{R}$. On the other hand, the identity function

$$
g: \mathbb{R}_{\ell} \longrightarrow \mathbb{R}
$$

is continuous, because the inverse image of $(a, b)$ is itself, which is open in $\mathbb{R}_{\ell}$.\\
In analysis, one studies several different but equivalent ways of formulating the definition of continuity. Some of these generalize to arbitrary spaces, and they are considered in the theorems that follow. The familiar " $\epsilon-\delta$ " definition and the "convergent sequence definition" do not generalize to arbitrary spaces; they will be treated when we study metric spaces.

Theorem 18.1. Let $X$ and $Y$ be topological spaces; let $f: X \rightarrow Y$. Then the following are equivalent:\\
(1) $f$ is continuous.\\
(2) For every subset $A$ of $X$, one has $f(\bar{A}) \subset \overline{f(A)}$.\\
(3) For every closed set $B$ of $Y$, the set $f^{-1}(B)$ is closed in $X$.\\
(4) For each $x \in X$ and each neighborhood $V$ of $f(x)$, there is a neighborhood $U$ of $x$ such that $f(U) \subset V$.

If the condition in (4) holds for the point $x$ of $X$, we say that $f$ is continuous at the point $x$.\\
Proof. We show that (1) ⇒ (2) ⇒ (3) $\Rightarrow(1)$ and that (1) $\Rightarrow(4) \Rightarrow(1)$.\\
(1) ⇒ (2). Assume that $f$ is continuous. Let $A$ be a subset of $X$. We show that if $x \in \bar{A}$, then $f(x) \in \overline{f(A)}$. Let $V$ be a neighborhood of $f(x)$. Then $f^{-1}(V)$ is an open set of $X$ containing $x$; it must intersect $A$ in some point $y$. Then $V$ intersects $f(A)$ in the point $f(y)$, so that $f(x) \in \overline{f(A)}$, as desired.\\
(2) ⇒ (3). Let $B$ be closed in $Y$ and let $A=f^{-1}(B)$. We wish to prove that $A$ is closed in $X$; we show that $\bar{A}=A$. By elementary set theory, we have $f(A)= f\left(f^{-1}(B)\right) \subset B$. Therefore, if $x \in \bar{A}$,

$$
f(x) \in f(\bar{A}) \subset \overline{f(A)} \subset \bar{B}=B
$$

so that $x \in f^{-1}(B)=A$. Thus $\bar{A} \subset A$, so that $\bar{A}=A$, as desired.\\
(3) ⇒ (1). Let $V$ be an open set of $Y$. Set $B=Y-V$. Then

$$
f^{-1}(B)=f^{-1}(Y)-f^{-1}(V)=X-f^{-1}(V) .
$$

Now $B$ is a closed set of $Y$. Then $f^{-1}(B)$ is closed in $X$ by hypothesis, so that $f^{-1}(V)$ is open in $X$, as desired.\\
(1) ⇒ (4). Let $x \in X$ and let $V$ be a neighborhood of $f(x)$. Then the set $U=f^{-1}(V)$ is a neighborhood of $x$ such that $f(U) \subset V$.\\
(4) ⇒ (1). Let $V$ be an open set of $Y$; let $x$ be a point of $f^{-1}(V)$. Then $f(x) \in V$, so that by hypothesis there is a neighborhood $U_{x}$ of $x$ such that $f\left(U_{x}\right) \subset V$. Then $U_{x} \subset f^{-1}(V)$. It follows that $f^{-1}(V)$ can be written as the union of the open sets $U_{x}$, so that it is open.

\section*{Homeomorphisms}
Let $X$ and $Y$ be topological spaces; let $f: X \rightarrow Y$ be a bijection. If both the function $f$ and the inverse function

$$
f^{-1}: Y \rightarrow X
$$

are continuous, then $f$ is called a homeomorphism.\\
The condition that $f^{-1}$ be continuous says that for each open set $U$ of $X$, the inverse image of $U$ under the map $f^{-1}: Y \rightarrow X$ is open in $Y$. But the inverse image of $U$ under the map $f^{-1}$ is the same as the image of $U$ under the map $f$. See Figure 18.1. So another way to define a homeomorphism is to say that it is a bijective correspondence $f: X \rightarrow Y$ such that $f(U)$ is open if and only if $U$ is open.

\begin{figure}[h]
\begin{center}
  \includegraphics[width=\textwidth]{2025_11_21_a8ffce9f36674b61ee7eg-031}
\captionsetup{labelformat=empty}
\caption{Figure 18.1}
\end{center}
\end{figure}

This remark shows that a homeomorphism $f: X \rightarrow Y$ gives us a bijective correspondence not only between $X$ and $Y$ but between the collections of open sets of $X$ and of $Y$. As a result, any property of $X$ that is entirely expressed in terms of the topology of $X$ (that is, in terms of the open sets of $X$ ) yields, via the correspondence $f$, the corresponding property for the space $Y$. Such a property of $X$ is called a topological property of $X$.

You may have studied in modern algebra the notion of an isomorphism between algebraic objects such as groups or rings. An isomorphism is a bijective correspondence that preserves the algebraic structure involved. The analogous concept in topology is that of homeomorphism; it is a bijective correspondence that preserves the topological structure involved.

Now suppose that $f: X \rightarrow Y$ is an injective continuous map, where $X$ and $Y$ are topological spaces. Let $Z$ be the image set $f(X)$, considered as a subspace of $Y$; then the function $f^{\prime}: X \rightarrow Z$ obtained by restricting the range of $f$ is bijective. If $f^{\prime}$ happens to be a homeomorphism of $X$ with $Z$, we say that the map $f: X \rightarrow Y$ is a topological imbedding, or simply an imbedding, of $X$ in $Y$.

EXAMPLE 4. The function $f: \mathbb{R} \rightarrow \mathbb{R}$ given by $f(x)=3 x+1$ is a homeomorphism. See Figure 18.2. If we define $g: \mathbb{R} \rightarrow \mathbb{R}$ by the equation

$$
g(y)=\frac{1}{3}(y-1)
$$

then one can check easily that $f(g(y))=y$ and $g(f(x))=x$ for all real numbers $x$ and $y$. It follows that $f$ is bijective and that $g=f^{-1}$; the continuity of $f$ and $g$ is a familiar result from calculus.

EXAMPLE 5. The function $F:(-1,1) \rightarrow \mathbb{R}$ defined by

$$
F(x)=\frac{x}{1-x^{2}}
$$

is a homeomorphism. See Figure 18.3. We have already noted in Example 9 of $\S 3$ that $F$ is a bijective order-preserving correspondence; its inverse is the function $G$ defined by

$$
G(y)=\frac{2 y}{1+\left(1+4 y^{2}\right)^{1 / 2}}
$$

The fact that $F$ is a homeomorphism can be proved in two ways. One way is to note that because $F$ is order preserving and bijective, $F$ carries a basis element for the order topology in ( $-1,1$ ) onto a basis element for the order topology in $\mathbb{R}$ and vice versa. As a result, $F$ is automatically a homeomorphism of $(-1,1)$ with $\mathbb{R}$ (both in the order topology). Since the order topology on $(-1,1)$ and the usual (subspace) topology agree, $F$ is a homeomorphism of $(-1,1)$ with $\mathbb{R}$.

\begin{figure}[h]
\begin{center}
  \includegraphics[width=\textwidth]{2025_11_21_a8ffce9f36674b61ee7eg-032(1)}
\captionsetup{labelformat=empty}
\caption{Figure 18.2}
\end{center}
\end{figure}

\begin{figure}[h]
\begin{center}
  \includegraphics[width=\textwidth]{2025_11_21_a8ffce9f36674b61ee7eg-032}
\captionsetup{labelformat=empty}
\caption{Figure 18.3}
\end{center}
\end{figure}

A second way to show $F$ a homeomorphism is to use the continuity of the algebraic functions and the square-root function to show that both $F$ and $G$ are continuous. These are familiar facts from calculus.

EXAMPLE 6. A bijective function $f: X \rightarrow Y$ can be continuous without being a homeomorphism. One such function is the identity map $g: \mathbb{R}_{\ell} \rightarrow \mathbb{R}$ considered in Example 3. Another is the following: Let $S^{1}$ denote the unit circle,

$$
S^{1}=\left\{x \times y \mid x^{2}+y^{2}=1\right\},
$$

considered as a subspace of the plane $\mathbb{R}^{2}$, and let

$$
F:[0,1) \longrightarrow S^{1}
$$

be the map defined by $f(t)=(\cos 2 \pi t, \sin 2 \pi t)$. The fact that $f$ is bijective and continuous follows from familiar properties of the trigonometric functions. But the function $f^{-1}$ is not continuous. The image under $f$ of the open set $U=\left[0, \frac{1}{4}\right)$ of the domain, for instance, is not open in $S^{1}$, for the point $p=f(0)$ lies in no open set $V$ of $\mathbb{R}^{2}$ such that $V \cap S^{l} \subset f(U)$. See Figure 18.4.

\begin{figure}[h]
\begin{center}
  \includegraphics[width=\textwidth]{2025_11_21_a8ffce9f36674b61ee7eg-033}
\captionsetup{labelformat=empty}
\caption{Figure 18.4}
\end{center}
\end{figure}

Example 7. Consider the function

$$
g:[0,1) \longrightarrow \mathbb{R}^{2}
$$

obtained from the function $f$ of the preceding example by expanding the range. The map $g$ is an example of a continuous injective map that is not an imbedding.

\section*{Constructing Continuous Functions}
How does one go about constructing continuous functions from one topological space to another? There are a number of methods used in analysis, of which some generalize to arbitrary topological spaces and others do not. We study first some constructions that do hold for general topological spaces, deferring consideration of the others until later.

Theorem 18.2 (Rules for constructing continuous functions). Let $X, Y$, and $Z$ be topological spaces.\\
(a) (Constant function) If $f: X \rightarrow Y$ maps all of $X$ into the single point $y_{0}$ of $Y$, then $f$ is continuous.\\
(b) (Inclusion) If $A$ is a subspace of $X$, the inclusion function $j: A \rightarrow X$ is continuous.\\
(c) (Composites) If $f: X \rightarrow Y$ and $g: Y \rightarrow Z$ are continuous, then the map $g \circ f: X \rightarrow Z$ is continuous.\\
(d) (Restricting the domain) If $f: X \rightarrow Y$ is continuous, and if $A$ is a subspace of $X$, then the restricted function $f \mid A: A \rightarrow Y$ is continuous.\\
(e) (Restricting or expanding the range) Let $f: X \rightarrow Y$ be continuous. If $Z$ is a subspace of $Y$ containing the image set $f(X)$, then the function $g: X \rightarrow Z$ obtained by restricting the range of $f$ is continuous. If $Z$ is a space having $Y$ as a subspace, then the function $h: X \rightarrow Z$ obtained by expanding the range of $f$ is continuous.\\
(f) (Local formulation of continuity) The map $f: X \rightarrow Y$ is continuous if $X$ can be written as the union of open sets $U_{\alpha}$ such that $f \mid U_{\alpha}$ is continuous for each $\alpha$.\\
Proof. (a) Let $f(x)=y_{0}$ for every $x$ in $X$. Let $V$ be open in $Y$. The set $f^{-1}(V)$ equals $X$ or $\varnothing$, depending on whether $V$ contains $y_{0}$ or not. In either case, it is open.\\
(b) If $U$ is open in $X$, then $j^{-1}(U)=U \cap A$, which is open in $A$ by definition of the subspace topology.\\
(c) If $U$ is open in $Z$, then $g^{-1}(U)$ is open in $Y$ and $f^{-1}\left(g^{-1}(U)\right)$ is open in $X$. But

$$
f^{-1}\left(g^{-1}(U)\right)=(g \circ f)^{-1}(U)
$$

by elementary set theory.\\
(d) The function $f \mid A$ equals the composite of the inclusion map $j: A \rightarrow X$ and the map $f: X \rightarrow Y$, both of which are continuous.\\
(e) Let $f: X \rightarrow Y$ be continuous. If $f(X) \subset Z \subset Y$, we show that the function $g: X \rightarrow Z$ obtained from $f$ is continuous. Let $B$ be open in $Z$. Then $B=Z \cap U$ for some open set $U$ of $Y$. Because $Z$ contains the entire image set $f(X)$,

$$
f^{-1}(U)=g^{-1}(B)
$$

by elementary set theory. Since $f^{-1}(U)$ is open, so is $g^{-1}(B)$.\\
To show $h: X \rightarrow Z$ is continuous if $Z$ has $Y$ as a subspace, note that $h$ is the composite of the map $f: X \rightarrow Y$ and the inclusion map $j: Y \rightarrow Z$.\\
(f) By hypothesis, we can write $X$ as a union of open sets $U_{\alpha}$, such that $f \mid U_{\alpha}$ is continuous for each $\alpha$. Let $V$ be an open set in $Y$. Then

$$
f^{-1}(V) \cap U_{\alpha}=\left(f \mid U_{\alpha}\right)^{-1}(V)
$$

because both expressions represent the set of those points $x$ lying in $U_{\alpha}$ for which $f(x) \in V$. Since $f \mid U_{\alpha}$ is continuous, this set is open in $U_{\alpha}$, and hence open in $X$. But

$$
f^{-1}(V)=\bigcup_{\alpha}\left(f^{-1}(V) \cap U_{\alpha}\right)
$$

so that $f^{-1}(V)$ is also open in $X$.\\
Theorem 18.3 (The pasting lemma). Let $X=A \cup B$, where $A$ and $B$ are closed in $X$. Let $f: A \rightarrow Y$ and $g: B \rightarrow Y$ be continuous. If $f(x)=g(x)$ for every $x \in A \cap B$, then $f$ and $g$ combine to give a continuous function $h: X \rightarrow Y$, defined by setting $h(x)=f(x)$ if $x \in A$, and $h(x)=g(x)$ if $x \in B$.

Proof. Let $C$ be a closed subset of $Y$. Now

$$
h^{-1}(C)=f^{-1}(C) \cup g^{-1}(C),
$$

by elementary set theory. Since $f$ is continuous, $f^{-1}(C)$ is closed in $A$ and, therefore, closed in $X$. Similarly, $g^{-1}(C)$ is closed in $B$ and therefore closed in $X$. Their union $h^{-1}(C)$ is thus closed in $X$. \(\square\)

This theorem also holds if $A$ and $B$ are open sets in $X$; this is just a special case of the "local formulation of continuity" rule given in preceding theorem.

EXAMPLE 8. Let us define a function $h: \mathbb{R} \rightarrow \mathbb{R}$ by setting

$$
h(x)= \begin{cases}x & \text { for } x \leq 0, \\ x / 2 & \text { for } x \geq 0 .\end{cases}
$$

Each of the "pieces" of this definition is a continuous function, and they agree on the overlapping part of their domains, which is the one-point set $\{0\}$. Since their domains are closed in $\mathbb{R}$, the function $h$ is continuous. One needs the "pieces" of the function to agree on the overlapping part of their domains in order to have a function at all. The equations

$$
k(x)= \begin{cases}x-2 & \text { for } x \leq 0, \\ x+2 & \text { for } x \geq 0,\end{cases}
$$

for instance, do not define a function. On the other hand, one needs some limitations on the sets $A$ and $B$ to guarantee continuity. The equations

$$
l(x)= \begin{cases}x-2 & \text { for } x<0 \\ x+2 & \text { for } x \geq 0\end{cases}
$$

for instance, do define a function $l$ mapping $\mathbb{R}$ into $\mathbb{R}$, and both of the pieces are continuous. But $l$ is not continuous; the inverse image of the open set ( 1,3 ), for instance, is the nonopen set $[0,1)$. See Figure 18.5.

\begin{figure}[h]
\begin{center}
  \includegraphics[width=\textwidth]{2025_11_21_a8ffce9f36674b61ee7eg-035}
\captionsetup{labelformat=empty}
\caption{Figure 18.5}
\end{center}
\end{figure}

Theorem 18.4 (Maps into products). Let $f: A \rightarrow X \times Y$ be given by the equation

$$
f(a)=\left(f_{1}(a), f_{2}(a)\right)
$$

Then $f$ is continuous if and only if the functions

$$
f_{1}: A \longrightarrow X \quad \text { and } \quad f_{2}: A \longrightarrow Y
$$

are continuous.\\
The maps $f_{1}$ and $f_{2}$ are called the coordinate functions of $f$.\\
Proof. Let $\pi_{1}: X \times Y \rightarrow X$ and $\pi_{2}: X \times Y \rightarrow Y$ be projections onto the first and second factors, respectively. These maps are continuous. For $\pi_{1}^{-1}(U)=U \times Y$ and $\pi_{2}^{-1}(V)=X \times V$, and these sets are open if $U$ and $V$ are open. Note that for each $a \in A$,

$$
f_{1}(a)=\pi_{1}(f(a)) \quad \text { and } \quad f_{2}(a)=\pi_{2}(f(a)) .
$$

If the function $f$ is continuous, then $f_{1}$ and $f_{2}$ are composites of continuous functions and therefore continuous. Conversely, suppose that $f_{1}$ and $f_{2}$ are continuous. We show that for each basis element $U \times V$ for the topology of $X \times Y$, its inverse image $f^{-1}(U \times V)$ is open. A point $a$ is in $f^{-1}(U \times V)$ if and only if $f(a) \in U \times V$, that is, if and only if $f_{1}(a) \in U$ and $f_{2}(a) \in V$. Therefore,

$$
f^{-1}(U \times V)=f_{1}^{-1}(U) \cap f_{2}^{-1}(V)
$$

Since both of the sets $f_{1}^{-1}(U)$ and $f_{2}^{-1}(V)$ are open, so is their intersection.\\
There is no useful criterion for the continuity of a map $f: A \times B \rightarrow X$ whose domain is a product space. One might ronjecture that $f$ is continuous if it is continuous "in each variable separately," but this conjecture is not true. (See Exercise 12.)

EXAMPLE 9. In calculus, a parametrized curve in the plane is defined to be a continuous map $f:[a, b] \rightarrow \mathbb{R}^{2}$. It is often expressed in the form $f(t)=(x(t), y(t))$; and one frequently uses the fact that $f$ is a continuous function of $t$ if both $x$ and $y$ are. Similarly, a vector field in the plane

$$
\begin{aligned}
\mathbf{v}(x, y) & =P(x, y) \mathbf{i}+Q(x, y) \mathbf{j} \\
& =(P(x, y), Q(x, y))
\end{aligned}
$$

is said to be continuous if both $P$ and $Q$ are continuous functions, or equivalently, if $\mathbf{v}$ is continuous as a map of $\mathbb{R}^{2}$ into $\mathbb{R}^{2}$. Both of these statements are simply special cases of the preceding theorem.\\
One way of forming continuous functions that is used a great deal in analysis is to take sums, differences, products, or quotients of continuous real-valued functions. It is a standard theorem that if $f, g: X \rightarrow \mathbb{R}$ are continuous, then $f+g, f-g$, and $f \cdot g$ are continuous, and $f / g$ is continuous if $g(x) \neq 0$ for all $x$. We shall consider this theorem in §21.

Yet another method for constructing continuous functions that is familiar from analysis is to take the limit of an infinite sequence of functions. There is a theorem to the effect that if a sequence of continuous real-valued functions of a real variable converges uniformly to a limit function, then the limit function is necessarily continuous. This theorem is called the Uniform Limit Theorem. It is used, for instance, to demonstrate the continuity of the trigonometric functions, when one defines these functions rigorously using the infinite series definitions of the sine and cosine. This theorem generalizes to a theorem about maps of an arbitrary topological space $X$ into a metric space $Y$. We shall prove it in §21.

\section*{Exercises}
\begin{enumerate}
  \item Prove that for functions $f: \mathbb{R} \rightarrow \mathbb{R}$, the $\epsilon-\delta$ definition of continuity implies the open set definition.
  \item Suppose that $f: X \rightarrow Y$ is continuous. If $x$ is a limit point of the subset $A$ of $X$, is it necessarily true that $f(x)$ is a limit point of $f(A)$ ?
  \item Let $X$ and $X^{\prime}$ denote a single set in the two topologies $\mathcal{T}$ and $\mathcal{T}^{\prime}$, respectively. Let $i: X^{\prime} \rightarrow X$ be the identity function.\\
(a) Show that $i$ is continuous $\Leftrightarrow \mathcal{T}^{\prime}$ is finer than $\mathcal{T}$.\\
(b) Show that $i$ is a homeomorphism $\Leftrightarrow \mathcal{T}^{\prime}=\mathcal{T}$.
  \item Given $x_{0} \in X$ and $y_{0} \in Y$, show that the maps $f: X \rightarrow X \times Y$ and $g: Y \rightarrow X \times Y$ defined by
\end{enumerate}

$$
f(x)=x \times y_{0} \quad \text { and } \quad g(y)=x_{0} \times y
$$

are imbeddings.\\
5. Show that the subspace $(a, b)$ of $\mathbb{R}$ is homeomorphic with $(0,1)$ and the subspace $[a, b]$ of $\mathbb{R}$ is homeomorphic with $[0,1]$.\\
6. Find a function $f: \mathbb{R} \rightarrow \mathbb{R}$ that is continuous at precisely one point.\\
7. (a) Suppose that $f: \mathbb{R} \rightarrow \mathbb{R}$ is "continuous from the right," that is,

$$
\lim _{x \rightarrow a^{+}} f(x)=f(a),
$$

for each $a \in \mathbb{R}$. Show that $f$ is continuous when considered as a function from $\mathbb{R}_{\ell}$ to $\mathbb{R}$.\\
(b) Can you conjecture what functions $f: \mathbb{R} \rightarrow \mathbb{R}$ are continuous when considered as maps from $\mathbb{R}$ to $\mathbb{R}_{\ell}$ ? As maps from $\mathbb{R}_{\ell}$ to $\mathbb{R}_{\ell}$ ? We shall return to this question in Chapter 3.\\
8. Let $Y$ be an ordered set in the order topology. Let $f, g: X \rightarrow Y$ be continuous.\\
(a) Show that the set $\{x \mid f(x) \leq g(x)\}$ is closed in $X$.\\
(b) Let $h: X \rightarrow Y$ be the function

$$
h(x)=\min \{f(x), g(x)\}
$$

Show that $h$ is continuous. [Hint: Use the pasting lemma.]\\
9. Let $\left\{A_{\alpha}\right\}$ be a collection of subsets of $X$; let $X=\bigcup_{\alpha} A_{\alpha}$. Let $f: X \rightarrow Y$; suppose that $f \mid A_{\alpha}$, is continuous for each $\alpha$.\\
(a) Show that if the collection $\left\{A_{\alpha}\right\}$ is finite and each set $A_{\alpha}$ is closed, then $f$ is continuous.\\
(b) Find an example where the collection $\left\{A_{\alpha}\right\}$ is countable and each $A_{\alpha}$ is closed, but $f$ is not continuous.\\
(c) An indexed family of sets $\left\{A_{\alpha}\right\}$ is said to be locally finite if each point $x$ of $X$ has a neighborhood that intersects $A_{\alpha}$ for only finitely many values of $\alpha$. Show that if the family $\left\{A_{\alpha}\right\}$ is locally finite and each $A_{\alpha}$ is closed, then $f$ is continuous.\\
10. Let $f: A \rightarrow B$ and $g: C \rightarrow D$ be continuous functions. Let us define a map $f \times g: A \times C \rightarrow B \times D$ by the equation

$$
(f \times g)(a \times c)=f(a) \times g(c)
$$

Show that $f \times g$ is continuous.\\
11. Let $F: X \times Y \rightarrow Z$. We say that $F$ is continuous in each variable separately if for each $y_{0}$ in $Y$, the map $h: X \rightarrow Z$ defined by $h(x)=F\left(x \times y_{0}\right)$ is continuous, and for each $x_{0}$ in $X$, the map $k: Y \rightarrow Z$ defined by $k(y)=F\left(x_{0} \times y\right)$ is continuous. Show that if $F$ is continuous, then $F$ is continuous in each variable separately.\\
12. Let $F: \mathbb{R} \times \mathbb{R} \rightarrow \mathbb{R}$ be defined by the equation

$$
F(x \times y)= \begin{cases}x y /\left(x^{2}+y^{2}\right) & \text { if } x \times y \neq 0 \times 0 \\ 0 & \text { if } x \times y=0 \times 0\end{cases}
$$

(a) Show that $F$ is continuous in each variable separately.\\
(b) Compute the function $g: \mathbb{R} \rightarrow \mathbb{R}$ defined by $g(x)=F(x \times x)$.\\
(c) Show that $F$ is not continuous.\\
13. Let $A \subset X$; let $f: A \rightarrow Y$ be continuous; let $Y$ be Hausdorff. Show that if $f$ may be extended to a continuous function $g: \bar{A} \rightarrow Y$, then $g$ is uniquely determined by $f$.

\section*{§19 The Product Topology}
We now return, for the remainder of the chapter, to the consideration of various methods for imposing topologies on sets.

Previously, we defined a topology on the product $X \times Y$ of two topological spaces. In the present section, we generalize this definition to more general cartesian products.

So let us consider the cartesian products

$$
X_{1} \times \cdots \times X_{n} \quad \text { and } \quad X_{1} \times X_{2} \times \cdots,
$$

where each $X_{i}$ is a topological space. There are two possible ways to proceed. One way is to take as basis all sets of the form $U_{1} \times \cdots \times U_{n}$ in the first case, and of the form $U_{1} \times U_{2} \times \cdots$ in the second case, where $U_{i}$ is an open set of $X_{i}$ for each $i$. This procedure does indeed define a topology on the cartesian product; we shall call it the box topology.

Another way to proceed is to generalize the subbasis formulation of the definition, given in §15. In this case, we take as a subbasis all sets of the form $\pi_{i}^{-1}\left(U_{i}\right)$, where $i$ is any index and $U_{i}$ is an open set of $X_{i}$. We shall call this topology the product topology.

How do these topologies differ? Consider the typical basis element $B$ for the second topology. It is a finite intersection of subbasis elements $\pi_{i}^{-1}\left(U_{i}\right)$, say for $i= i_{1}, \ldots, i_{k}$. Then a point $\mathbf{x}$ belongs to $B$ if and only if $\pi_{i}(\mathbf{x})$ belongs to $U_{i}$ for $i= i_{1}, \ldots, i_{k}$; there is no restriction on $\pi_{i}(x)$ for other values of $i$.

It follows that these two topologies agree for the finite cartesian product and differ for the infinite product. What is not clear is why we seem to prefer the second topology. This is the question we shall explore in this section.

Before proceeding, however, we shall introduce a more general notion of cartesian product. So far, we have defined the cartesian product of an indexed family of sets only in the cases where the index set was the set $\{1, \ldots, n\}$ or the set $\mathbb{Z}_{+}$. Now we consider the case where the index set is completely arbitrary.

Definition. Let $J$ be an index set. Given a set $X$, we define a $J$-tuple of elements of $X$ to be a function $\mathbf{x}: J \rightarrow X$. If $\alpha$ is an element of $J$, we often denote the value of $\mathbf{x}$ at $\alpha$ by $x_{\alpha}$ rather than $\mathbf{x}(\alpha)$; we call it the $\alpha$ th coordinate of $\mathbf{x}$. And we often denote the function $\mathbf{x}$ itself by the symbol

$$
\left(x_{\alpha}\right)_{\alpha \in J}
$$

which is as close as we can come to a "tuple notation" for an arbitrary index set $J$. We denote the set of all $J$-tuples of elements of $X$ by $X^{J}$.

Definition. Let $\left\{A_{\alpha}\right\}_{\alpha \in J}$ be an indexed family of sets; let $X=\bigcup_{\alpha \in J} A_{\alpha}$. The cartesian product of this indexed family, denoted by

$$
\prod_{\alpha \in J} A_{\alpha},
$$

is defined to be the set of all $J$-tuples $\left(x_{\alpha}\right)_{\alpha \in J}$ of elements of $X$ such that $x_{\alpha} \in A_{\alpha}$ for each $\alpha \in J$. That is, it is the set of all functions

$$
\mathrm{x}: J \rightarrow \bigcup_{\alpha \in J} A_{\alpha}
$$

such that $\mathbf{x}(\alpha) \in A_{\alpha}$ for each $\alpha \in J$.

Occasionally we denote the product simply by $\prod A_{\alpha}$, and its general element by ( $x_{\alpha}$ ), if the index set is understood.

If all the sets $A_{\alpha}$ are equal to one set $X$, then the cartesian product $\prod_{\alpha \in J} A_{\alpha}$ is just the set $X^{J}$ of all $J$-tuples of elements of $X$. We sometimes use "tuple notation" for the elements of $X^{J}$, and sometimes we use functional notation, depending on which is more convenient.

Definition. Let $\left\{X_{\alpha}\right\}_{\alpha \in J}$ be an indexed family of topological spaces. Let us take as a basis for a topology on the product space

$$
\prod_{\alpha \in J} X_{\alpha}
$$

the collection of all sets of the form

$$
\prod_{\alpha \in J} U_{\alpha},
$$

where $U_{\alpha}$ is open in $X_{\alpha}$, for each $\alpha \in J$. The topology generated by this basis is called the box topology.

This collection satisfies the first condition for a basis because $\prod X_{\alpha}$ is itself a basis element; and it satisfies the second condition because the intersection of any two basis elements is another basis element:

$$
\left(\prod_{\alpha \in J} U_{\alpha}\right) \cap\left(\prod_{\alpha \in J} V_{\alpha}\right)=\prod_{\alpha \in J}\left(U_{\alpha} \cap V_{\alpha}\right) .
$$

Now we generalize the subbasis formulation of the definition. Let

$$
\pi_{\beta}: \prod_{\alpha \in J} X_{\alpha} \rightarrow X_{\beta}
$$

be the function assigning to each element of the product space its $\beta$ th coordinate,

$$
\pi_{\beta}\left(\left(x_{\alpha}\right)_{\alpha \in J}\right)=x_{\beta} ;
$$

it is called the projection mapping associated with the index $\beta$.\\
Definition. Let $S_{\beta}$ denote the collection

$$
S_{\beta}=\left\{\pi_{\beta}^{-1}\left(U_{\beta}\right) \mid U_{\beta} \text { open in } X_{\beta}\right\}
$$

and let $S$ denote the union of these collections,

$$
s=\bigcup_{\beta \in J} s_{\beta}
$$

The topology generated by the subbasis $S$ is called the product topology. In this topology $\prod_{\alpha \in J} X_{\alpha}$ is called a product space.

To compare these topologies, we consider the basis $\mathscr{B}$ that $S$ generates. The collection $\mathscr{B}$ consists of all finite intersections of elements of $S$. If we intersect elements belonging to the same one of the sets $S_{\beta}$, we do not get anything new, because

$$
\pi_{\beta}^{-1}\left(U_{\beta}\right) \cap \pi_{\beta}^{-1}\left(V_{\beta}\right)=\pi_{\beta}^{-1}\left(U_{\beta} \cap V_{\beta}\right) ;
$$

the intersection of two elements of $S_{\beta}$, or of finitely many such elements, is again an element of $S_{\beta}$. We get something new only when we intersect elements from different sets $S_{\beta}$. The typical element of the basis $\mathscr{B}$ can thus be described as follows: Let $\beta_{1}$, $\ldots, \beta_{n}$ be a finite set of distinct indices from the index set $J$, and let $U_{\beta_{i}}$ be an open set in $X_{\beta_{i}}$ for $i=1, \ldots, n$. Then

$$
B=\pi_{\beta_{1}}^{-1}\left(U_{\beta_{1}}\right) \cap \pi_{\beta_{2}}^{-1}\left(U_{\beta_{2}}\right) \cap \cdots \cap \pi_{\beta_{n}}^{-1}\left(U_{\beta_{n}}\right)
$$

is the typical element of $\mathscr{B}$.\\
Now a point $\mathbf{x}=\left(x_{\alpha}\right)$ is in $B$ if and only if its $\beta_{1}$ th coordinate is in $U_{\beta_{1}}$, its $\beta_{2}$ th coordinate is in $U_{\beta_{2}}$, and so on. There is no restriction whatever on the $\alpha$ th coordinate of $\mathbf{x}$ if $\alpha$ is not one of the indices $\beta_{1}, \ldots, \beta_{n}$. As a result, we can write $B$ as the product

$$
B=\prod_{\alpha \in J} U_{\alpha}
$$

where $U_{\alpha}$ denotes the entire space $X_{\alpha}$ if $\alpha \neq \beta_{1}, \ldots, \beta_{n}$.\\
All this is summarized in the following theorem:\\
Theorem 19.1 (Comparison of the box and product topologies). The box topology on $\prod X_{\alpha}$ has as basis all sets of the form $\prod U_{\alpha}$, where $U_{\alpha}$ is open in $X_{\alpha}$ for each $\alpha$. The product topology on $\prod X_{\alpha}$ has as basis all sets of the form $\prod U_{\alpha}$, where $U_{\alpha}$ is open in $X_{\alpha}$ for each $\alpha$ and $U_{\alpha}$ equals $X_{\alpha}$ except for finitely many values of $\alpha$.

Two things are immediately clear. First, for finite products $\prod_{\alpha=1}^{n} X_{\alpha}$ the two topologies are precisely the same. Second, the box topology is in general finer than the product topology.

What is not so clear is why we prefer the product topology to the box topology. The answer will appear as we continue our study of topology. We shall find that a number of important theorems about finite products will also hold for arbitrary products if we use the product topology, but not if we use the box topology. As a result, the product topology is extremely important in mathematics. The box topology is not so important; we shall use it primarily for constructing counterexamples. Therefore, we make the following convention:

Whenever we consider the product $\prod X_{\alpha}$, we shall assume it is given the product topology unless we specifically state otherwise.

Some of the theorems we proved for the product $X \times Y$ hold for the product $\Pi X_{\alpha}$ no matter which topology we use. We list them here; most of the proofs are left to the exercises.

Theorem 19.2. Suppose the topology on each space $X_{\alpha}$ is given by a basis $\mathscr{B}_{\alpha}$. The collection of all sets of the form

$$
\prod_{\alpha \in J} B_{\alpha},
$$

where $B_{\alpha} \in B_{\alpha}$ for each $\alpha$, will serve as a basis for the box topology on $\prod_{\alpha \in J} X_{\alpha}$.\\
The collection of all sets of the same form, where $B_{\alpha} \in B_{\alpha}$ for finitely many indices $\alpha$ and $B_{\alpha}=X_{\alpha}$ for all the remaining indices, will serve as a basis for the product topology $\prod_{\alpha \in J} X_{\alpha}$.

EXAMPLE 1. Consider euclidean $n$-space $\mathbb{R}^{n}$. A basis for $\mathbb{R}$ consists of all open intervals in $\mathbb{R}$; hence a basis for the topology of $\mathbb{R}^{n}$ consists of all products of the form

$$
\left(a_{1}, b_{1}\right) \times\left(a_{2}, b_{2}\right) \times \cdots \times\left(a_{n}, b_{n}\right) .
$$

Since $\mathbb{R}^{n}$ is a finite product, the box and product topologies agree. Whenever we consider $\mathbb{R}^{n}$, we will assume that it is given this topology, unless we specifically state otherwise.

Theorem 19.3. Let $A_{\alpha}$ be a subspace of $X_{\alpha}$, for each $\alpha \in J$. Then $\prod A_{\alpha}$ is a subspace of $\prod X_{\alpha}$ if both products are given the box topology, or if both products are given the product topology.

Theorem 19.4. If each space $X_{\alpha}$ is a Hausdorff space, then $\prod X_{\alpha}$ is a Hausdorff space in both the box and product topologies.

Theorem 19.5. Let $\left\{X_{\alpha}\right\}$ be an indexed family of spaces; let $A_{\alpha} \subset X_{\alpha}$ for each $\alpha$. If $\prod X_{\alpha}$ is given either the product or the box topology, then

$$
\prod \bar{A}_{\alpha}=\overline{\prod A_{\alpha}} .
$$

Proof. Let $\mathbf{x}=\left(x_{\alpha}\right)$ be a point of $\prod \bar{A}_{\alpha}$; we show that $\mathbf{x} \in \overline{\prod A_{\alpha}}$. Let $U=\prod U_{\alpha}$ be a basis element for either the box or product topology that contains $\mathbf{x}$. Since $x_{\alpha} \in \bar{A}_{\alpha}$, we can choose a point $y_{\alpha} \in U_{\alpha} \cap A_{\alpha}$ for each $\alpha$. Then $\mathbf{y}=\left(y_{\alpha}\right)$ belongs to both $U$ and $\prod A_{\alpha}$. Since $U$ is arbitrary, it follows that $\mathbf{x}$ belongs to the closure of $\prod A_{\alpha}$.

Conversely, suppose $\mathbf{x}=\left(x_{\alpha}\right)$ lies in the closure of $\prod A_{\alpha}$, in either topology. We show that for any given index $\beta$, we have $x_{\beta} \in \bar{A}_{\beta}$. Let $V_{\beta}$ be an arbitrary open set of $X_{\beta}$ containing $x_{\beta}$. Since $\pi_{\beta}^{-1}\left(V_{\beta}\right)$ is open in $\prod X_{\alpha}$ in either topology, it contains a point $\mathbf{y}=\left(y_{\alpha}\right)$ of $\prod A_{\alpha}$. Then $y_{\beta}$ belongs to $V_{\beta} \cap A_{\beta}$. It follows that $x_{\beta} \in \bar{A}_{\beta}$. \(\square\)

So far, no reason has appeared for preferring the product to the box topology. It is when we try to generalize our previous theorem about continuity of maps into product spaces that a difference first arises. Here is a theorem that does not hold if $\prod X_{\alpha}$ is given the box topology:

Theorem 19.6. Let $f: A \rightarrow \prod_{\alpha \in J} X_{\alpha}$ be given by the equation

$$
f(a)=\left(f_{\alpha}(a)\right)_{\alpha \in J}
$$

where $f_{\alpha}: A \rightarrow X_{\alpha}$ for each $\alpha$. Let $\prod X_{\alpha}$ have the product topology. Then the function $f$ is continuous if and only if each function $f_{\alpha}$ is continuous.

Proof. Let $\pi_{\beta}$ be the projection of the product onto its $\beta$ th factor. The function $\pi_{\beta}$ is continuous, for if $U_{\beta}$ is open in $X_{\beta}$, the set $\pi_{\beta}^{-1}\left(U_{\beta}\right)$ is a subbasis element for the product topology on $X_{\alpha}$. Now suppose that $f: A \rightarrow \prod X_{\alpha}$ is continuous. The function $f_{\beta}$ equals the composite $\pi_{\beta} \circ f$; being the composite of two continuous functions, it is continuous.

Conversely, suppose that each coordinate function $f_{\alpha}$ is continuous. To prove that $f$ is continuous, it suffices to prove that the inverse image under $f$ of each subbasis element is open in $A$; we remarked on this fact when we defined continuous functions. A typical subbasis element for the product topology on $\prod X_{\alpha}$ is a set of the form $\pi_{\beta}^{-1}\left(U_{\beta}\right)$, where $\beta$ is some index and $U_{\beta}$ is open in $X_{\beta}$. Now

$$
f^{-1}\left(\pi_{\beta}^{-1}\left(U_{\beta}\right)\right)=f_{\beta}^{-1}\left(U_{\beta}\right),
$$

because $f_{\beta}=\pi_{\beta} \circ f$. Since $f_{\beta}$ is continuous, this set is open in $A$, as desired.\\
Why does this theorem fail if we use the box topology? Probably the most convincing thing to do is to look at an example.

Example 2. Consider $\mathbb{R}^{\omega}$, the countably infinite product of $\mathbb{R}$ with itself. Recall that

$$
\mathbb{R}^{\omega}=\prod_{n \in \mathbb{Z}_{+}} X_{n},
$$

where $X_{n}=\mathbb{R}$ for each $n$. Let us define a function $f: \mathbb{R} \rightarrow \mathbb{R}^{\omega}$ by the equation

$$
f(t)=(t, t, t, \ldots) ;
$$

the $n$th coordinate function of $f$ is the function $f_{n}(t)=t$. Each of the coordinate functions $f_{n}: \mathbb{R} \rightarrow \mathbb{R}$ is continuous; therefore, the function $f$ is continuous if $\mathbb{R}^{\omega}$ is given the product topology. But $f$ is not continuous if $\mathbb{R}^{\omega}$ is given the box topology. Consider, for example, the basis element

$$
B=(-1,1) \times\left(-\frac{1}{2}, \frac{1}{2}\right) \times\left(-\frac{1}{3}, \frac{1}{3}\right) \times \cdots
$$

for the box topology. We assert that $f^{-1}(B)$ is not open in $\mathbb{R}$. If $f^{-1}(B)$ were open in $\mathbb{R}$, it would contain some interval $(-\delta, \delta)$ about the point 0 . This would mean that $f((-\delta, \delta)) \subset B$, so that, applying $\pi_{n}$ to both sides of the inclusion,

$$
f_{n}((-\delta, \delta))=(-\delta, \delta) \subset(-1 / n, 1 / n)
$$

for all $n$, a contradiction.

\section*{Exercises}
\begin{enumerate}
  \item Prove Theorem 19.2.
  \item Prove Theorem 19.3.
  \item Prove Theorem 19.4.
  \item Show that $\left(X_{1} \times \cdots \times X_{n-1}\right) \times X_{n}$ is homeomorphic with $X_{1} \times \cdots \times X_{n}$.
  \item One of the implications stated in Theorem 19.6 holds for the box topology. Which one?
  \item Let $\mathbf{x}_{1}, \mathbf{x}_{2}, \ldots$ be a sequence of the points of the product space $\prod X_{\alpha}$. Show that this sequence converges to the point $\mathbf{x}$ if and only if the sequence $\pi_{\alpha}\left(\mathbf{x}_{1}\right), \pi_{\alpha}\left(\mathbf{x}_{2}\right)$, ... converges to $\pi_{\alpha}(\mathbf{x})$ for each $\alpha$. Is this fact true if one uses the box topology instead of the product topology?
  \item Let $\mathbb{R}^{\infty}$ be the subset of $\mathbb{R}^{\omega}$ consisting of all sequences that are "eventually zero," that is, all sequences $\left(x_{1}, x_{2}, \ldots\right)$ such that $x_{i} \neq 0$ for only finitely many values of $i$. What is the closure of $\mathbb{R}^{\infty}$ in $\mathbb{R}^{\omega}$ in the box and product topologies? Justify your answer.
  \item Given sequences ( $a_{1}, a_{2}, \ldots$ ) and ( $b_{1}, b_{2}, \ldots$ ) of real numbers with $a_{i}>0$ for all $i$, define $h: \mathbb{R}^{\omega} \rightarrow \mathbb{R}^{\omega}$ by the equation
\end{enumerate}

$$
h\left(\left(x_{1}, x_{2}, \ldots\right)\right)=\left(a_{1} x_{1}+b_{1}, a_{2} x_{2}+b_{2}, \ldots\right) .
$$

Show that if $\mathbb{R}^{\omega}$ is given the product topology, $h$ is a homeomorphism of $\mathbb{R}^{\omega}$ with itself. What happens if $\mathbb{R}^{\omega}$ is given the box topology?\\
9. Show that the choice axiom is equivalent to the statement that for any indexed family $\left\{A_{\alpha}\right\}_{\alpha \in J}$ of nonempty sets, with $J \neq 0$, the cartesian product

$$
\prod_{\alpha \in J} A_{\alpha}
$$

is not empty.\\
10. Let $A$ be a set; let $\left\{X_{\alpha}\right\}_{\alpha \in J}$ be an indexed family of spaces; and let $\left\{f_{\alpha}\right\}_{\alpha \in J}$ be an indexed family of functions $f_{\alpha}: A \rightarrow X_{\alpha}$.\\
(a) Show there is a unique coarsest topology $\mathcal{T}$ on $A$ relative to which each of the functions $f_{\alpha}$ is continuous.\\
(b) Let

$$
S_{\beta}=\left\{f_{\beta}^{-1}\left(U_{\beta}\right) \mid U_{\beta} \text { is open in } X_{\beta}\right\},
$$

and let $S=\bigcup S_{\beta}$. Show that $S$ is a subbasis for $\mathcal{T}$.\\
(c) Show that a map $g: Y \rightarrow A$ is continuous relative to $\mathcal{T}$ if and only if each map $f_{\alpha} \circ g$ is continuous.\\
(d) Let $f: A \rightarrow \prod X_{\alpha}$ be defined by the equation

$$
f(a)=\left(f_{\alpha}(a)\right)_{\alpha \in J} ;
$$

let $Z$ denote the subspace $f(A)$ of the product space $\prod X_{\alpha}$. Show that the image under $f$ of each element of $\mathcal{T}$ is an open set of $Z$.

\section*{§20 The Metric Topology}
One of the most important and frequently used ways of imposing a topology on a set is to define the topology in terms of a metric on the set. Topologies given in this way lie at the heart of modern analysis, for example. In this section, we shall define the metric topology and shall give a number of examples. In the next section, we shall consider some of the properties that metric topologies satisfy.

Definition. A metric on a set $X$ is a function

$$
d: X \times X \longrightarrow R
$$

having the following properties:\\
(1) $d(x, y) \geq 0$ for all $x, y \in X$; equality holds if and only if $x=y$.\\
(2) $d(x, y)=d(y, x)$ for all $x, y \in X$.\\
(3) (Triangle inequality) $d(x, y)+d(y, z) \geq d(x, z)$, for all $x, y, z \in X$.

Given a metric $d$ on $X$, the number $d(x, y)$ is often called the distance between $x$ and $y$ in the metric $d$. Given $\epsilon>0$, consider the set

$$
B_{d}(x, \epsilon)=\{y \mid d(x, y)<\epsilon\}
$$

of all points $y$ whose distance from $x$ is less than $\epsilon$. It is called the $\epsilon$-ball centered at $x$. Sometimes we omit the metric $d$ from the notation and write this ball simply as $B(x, \epsilon)$, when no confusion will arise.

Definition. If $d$ is a metric on the set $X$, then the collection of all $\epsilon$-balls $B_{d}(x, \epsilon)$, for $x \in X$ and $\epsilon>0$, is a basis for a topology on $X$, called the metric topology induced by $d$.

The first condition for a basis is trivial, since $x \in B(x, \epsilon)$ for any $\epsilon>0$. Before checking the second condition for a basis, we show that if $y$ is a point of the basis element $B(x, \epsilon)$, then there is a basis element $B(y, \delta)$ centered at $y$ that is contained in $B(x, \epsilon)$. Define $\delta$ to be the positive number $\epsilon-d(x, y)$. Then $B(y, \delta) \subset B(x, \epsilon)$, for if $z \in B(y, \delta)$, then $d(y, z)<\epsilon-d(x, y)$, from which we conclude that

$$
d(x, z) \leq d(x, y)+d(y, z)<\epsilon
$$

See Figure 20.1.\\
Now to check the second condition for a basis, let $B_{1}$ and $B_{2}$ be two basis elements and let $y \in B_{1} \cap B_{2}$. We have just shown that we can choose positive numbers $\delta_{1}$ and $\delta_{2}$ so that $B\left(y, \delta_{1}\right) \subset B_{1}$ and $B\left(y, \delta_{2}\right) \subset B_{2}$. Letting $\delta$ be the smaller of $\delta_{1}$ and $\delta_{2}$, we conclude that $B(y, \delta) \subset B_{1} \cap B_{2}$.

Using what we have just proved, we can rephrase the definition of the metric topology as follows:

\begin{figure}[h]
\begin{center}
  \includegraphics[width=\textwidth]{2025_11_21_a8ffce9f36674b61ee7eg-046}
\captionsetup{labelformat=empty}
\caption{Figure 20.1}
\end{center}
\end{figure}

A set $U$ is open in the metric topology induced by $d$ if and only if for each $y \in U$, there is a $\delta>0$ such that $B_{d}(y, \delta) \subset U$.

Clearly this condition implies that $U$ is open. Conversely, if $U$ is open, it contains a basis element $B=B_{d}(x, \epsilon)$ containing $y$, and $B$ in turn contains a basis element $B_{d}(y, \delta)$ centered at $y$.

Example 1. Given a set $X$, define

$$
\begin{array}{ll}
d(x, y)=1 & \text { if } x \neq y, \\
d(x, y)=0 & \text { if } x=y .
\end{array}
$$

It is trivial to check that $d$ is a metric. The topology it induces is the discrete topology; the basis element $B(x, 1)$, for example, consists of the point $x$ alone.

EXAMPLE 2. The standard metric on the real numbers $\mathbb{R}$ is defined by the equation

$$
d(x, y)=|x-y| .
$$

It is easy to check that $d$ is a metric. The topology it induces is the same as the order topology: Each basis element ( $a, b$ ) for the order topology is a basis element for the metric topology; indeed,

$$
(a, b)=B(x, \epsilon),
$$

where $x=(a+b) / 2$ and $\epsilon=(b-a) / 2$. And conversely, each $\epsilon$-ball $B(x, \epsilon)$ equals an open interval: the interval $(x-\epsilon, x+\epsilon)$.

Definition. If $X$ is a topological space, $X$ is said to be metrizable if there exists a metric $d$ on the set $X$ that induces the topology of $X$. A metric space is a metrizable space $X$ together with a specific metric $d$ that gives the topology of $X$.

Many of the spaces important for mathematics are metrizable, but some are not. Metrizability is always a highly desirable attribute for a space to possess, for the existence of a metric gives one a valuable tool for proving theorems about the space.

It is, therefore, a problem of fundamental importance in topology to find conditions on a topological space that will guarantee it is metrizable. One of our goals in Chapter 4 will be to find such conditions; they are expressed there in the famous theorem called Urysohn's metrization theorem. Further metrization theorems appear in Chapter 6. In the present section we shall content ourselves with proving merely that $\mathbb{R}^{n}$ and $\mathbb{R}^{\omega}$ are metrizable.

Although the metrizability problem is an important problem in topology, the study of metric spaces as such does not properly belong to topology as much as it does to analysis. Metrizability of a space depends only on the topology of the space in question, but properties that involve a specific metric for $X$ in general do not. For instance, one can make the following definition in a metric space:

Definition. Let $X$ be a metric space with metric $d$. A subset $A$ of $X$ is said to be bounded if there is some number $M$ such that

$$
d\left(a_{1}, a_{2}\right) \leq M
$$

for every pair $a_{1}, a_{2}$ of points of $A$. If $A$ is bounded and nonempty, the diameter of $A$ is defined to be the number

$$
\operatorname{diam} A=\sup \left\{d\left(a_{1}, a_{2}\right) \mid a_{1}, a_{2} \in A\right\}
$$

Boundedness of a set is not a topological property, for it depends on the particular metric $d$ that is used for $X$. For instance, if $X$ is a metric space with metric $d$, then there exists a metric $\bar{d}$ that gives the topology of $X$, relative to which every subset of $X$ is bounded. It is defined as follows:

Theorem 20.1. Let $X$ be a metric space with metric $d$. Define $\bar{d}: X \times X \rightarrow \mathbb{R}$ by the equation

$$
\bar{d}(x, y)=\min \{d(x, y), 1\}
$$

Then $\bar{d}$ is a metric that induces the same topology as $d$.\\
The metric $\bar{d}$ is called the standard bounded metric corresponding to $d$.\\
Proof. Checking the first two conditions for a metric is trivial. Let us check the triangle inequality:

$$
\bar{d}(x, z) \leq \bar{d}(x, y)+\bar{d}(y, z)
$$

Now if either $d(x, y) \geq 1$ or $d(y, z) \geq 1$, then the right side of this inequality is at least 1 ; since the left side is (by definition) at most 1 , the inequality holds. It remains to consider the case in which $d(x, y)<1$ and $d(y, z)<1$. In this case, we have

$$
d(x, z) \leq d(x, y)+d(y, z)=\bar{d}(x, y)+\bar{d}(y, z)
$$

Since $\bar{d}(x, z) \leq d(x, z)$ by definition, the triangle inequality holds for $\bar{d}$.

Now we note that in any metric space, the collection of $\epsilon$-balls with $\epsilon<1$ forms a basis for the metric topology, for every basis element containing $x$ contains such an $\epsilon$-ball centered at $x$. It follows that $d$ and $\bar{d}$ induce the same topology on $X$, because the collections of $\epsilon$-balls with $\epsilon<1$ under these two metrics are the same collection.

Now we consider some familiar spaces and show they are metrizable.\\
Definition. Given $\mathbf{x}=\left(x_{1}, \ldots, x_{n}\right)$ in $\mathbb{R}^{n}$, we define the norm $\mathbf{x}$ of by the equation

$$
\|x\|=\left(x_{1}^{2}+\cdots+x_{n}^{2}\right)^{1 / 2} ;
$$

and we define the euclidean metric $d$ on $\mathbb{R}^{n}$ by the equation

$$
d(\mathbf{x}, \mathbf{y})=\|\mathbf{x}-\mathbf{y}\|=\left[\left(x_{1}-y_{1}\right)^{2}+\cdots+\left(x_{n}-y_{n}\right)^{2}\right]^{1 / 2} .
$$

We define the square metric $\rho$ by the equation

$$
\rho(\mathbf{x}, \mathbf{y})=\max \left\{\left|x_{1}-y_{1}\right|, \ldots,\left|x_{n}-y_{n}\right|\right\} .
$$

The proof that $d$ is a metric requires some work; it is probably already familiar to you. If not, a proof is outlined in the exercises. We shall seldom have occasion to use this metric on $\mathbb{R}^{n}$.

To show that $\rho$ is a metric is easier. Only the triangle inequality is nontrivial. From the triangle inequality for $\mathbb{R}$ it follows that for each positive integer $i$,

$$
\left|x_{i}-z_{i}\right| \leq\left|x_{i}-y_{i}\right|+\left|y_{i}-z_{i}\right| .
$$

Then by definition of $\rho$,

$$
\left|x_{i}-z_{i}\right| \leq \rho(\mathbf{x}, \mathbf{y})+\rho(\mathbf{y}, \mathbf{z}) .
$$

As a result

$$
\rho(\mathbf{x}, \mathbf{z})=\max \left\{\left|x_{i}-z_{i}\right|\right\} \leq \rho(\mathbf{x}, \mathbf{y})+\rho(\mathbf{y}, \mathbf{z}),
$$

as desired.\\
On the real line $\mathbb{R}=\mathbb{R}^{1}$, these two metrics coincide with the standard metric for $\mathbb{R}$. In the plane $\mathbb{R}^{2}$, the basis elements under $d$ can be pictured as circular regions, while the basis elements under $\rho$ can be pictured as square regions.

We now show that each of these metrics induces the usual topology on $\mathbb{R}^{n}$. We need the following lemma:

Lemma 20.2. Let $d$ and $d^{\prime}$ be two metrics on the set $X$; let $\mathcal{T}$ and $\mathcal{T}^{\prime}$ be the topologies they induce, respectively. Then $\mathcal{T}^{\prime}$ is finer than $\mathcal{T}$ if and only if for each $x$ in $X$ and each $\epsilon>0$, there exists a $\delta>0$ such that

$$
B_{d^{\prime}}(x, \delta) \subset B_{d}(x, \epsilon)
$$

Proof. Suppose that $\mathcal{T}^{\prime}$ is finer than $\mathcal{T}$. Given the basis element $B_{d}(x, \epsilon)$ for $\mathcal{T}$, there is by Lemma 13.3 a basis element $B^{\prime}$ for the topology $\mathcal{T}^{\prime}$ such that $x \in B^{\prime} \subset B_{d}(x, \epsilon)$. Within $B^{\prime}$ we can find a ball $B_{d^{\prime}}(x, \delta)$ centered at $x$.

Conversely, suppose the $\delta-\epsilon$ condition holds. Given a basis element $B$ for $\mathcal{T}$ containing $x$, we can find within $B$ a ball $B_{d}(x, \epsilon)$ centered at $x$. By the given condition, there is a $\delta$ such that $B_{d^{\prime}}(x, \delta) \subset B_{d}(x, \epsilon)$. Then Lemma 13.3 applies to show $\mathcal{T}^{\prime}$ is finer than $\mathcal{T}$.

Theorem 20.3. The topologies on $\mathbb{R}^{n}$ induced by the euclidean metric $d$ and the square metric $\rho$ are the same as the product topology on $\mathbb{R}^{n}$.\\
Proof. Let $\mathbf{x}=\left(x_{1}, \ldots, x_{n}\right)$ and $\mathbf{y}=\left(y_{1}, \ldots, y_{n}\right)$ be two points of $\mathbb{R}^{n}$. It is simple algebra to check that

$$
\rho(\mathbf{x}, \mathbf{y}) \leq d(\mathbf{x}, \mathbf{y}) \leq \sqrt{n} \rho(\mathbf{x}, \mathbf{y}) .
$$

The first inequality shows that

$$
B_{d}(\mathbf{x}, \epsilon) \subset B_{\rho}(\mathbf{x}, \epsilon)
$$

for all $\mathbf{x}$ and $\epsilon$, since if $d(\mathbf{x}, \mathbf{y})<\epsilon$, then $\rho(\mathbf{x}, \mathbf{y})<\epsilon$ also. Similarly, the second inequality shows that

$$
B_{\rho}(\mathbf{x}, \epsilon / \sqrt{n}) \subset B_{d}(\mathbf{x}, \epsilon)
$$

for all $\mathbf{x}$ and $\epsilon$. It follows from the preceding lemma that the two metric topologies are the same.

Now we show that the product topology is the same as that given by the metric $\rho$. First, let

$$
B=\left(a_{1}, b_{1}\right) \times \cdots \times\left(a_{n}, b_{n}\right)
$$

be a basis element for the product topology, and let $\mathbf{x}=\left(x_{1}, \ldots, x_{n}\right)$ be an element of $B$. For each $i$, there is an $\epsilon_{i}$ such that

$$
\left(x_{i}-\epsilon_{i}, x_{i}+\epsilon_{i}\right) \subset\left(a_{i}, b_{i}\right)
$$

choose $\epsilon=\min \left\{\epsilon_{1}, \ldots, \epsilon_{n}\right\}$. Then $B_{\rho}(\mathbf{x}, \epsilon) \subset B$, as you can readily check. As a result, the $\rho$-topology is finer than the product topology.

Conversely, let $B_{\rho}(\mathbf{x}, \epsilon)$ be a basis element for the $\rho$-topology. Given the element $\mathbf{y} \in B_{\rho}(\mathbf{x}, \epsilon)$, we need to find a basis element $B$ for the product topology such that

$$
\mathbf{y} \in B \subset B_{\rho}(\mathbf{x}, \epsilon) .
$$

But this is trivial, for

$$
B_{\rho}(\mathbf{x}, \epsilon)=\left(x_{1}-\epsilon, x_{1}+\epsilon\right) \times \cdots \times\left(x_{n}-\epsilon, x_{n}+\epsilon\right)
$$

is itself a basis element for the product topology.

Now we consider the infinite cartesian product $\mathbb{R}^{\omega}$. It is natural to try to generalize the metrics $d$ and $\rho$ to this space. For instance, one can attempt to define a metric $d$ on $\mathbb{R}^{\omega}$ by the equation

$$
d(\mathbf{x}, \mathbf{y})=\left[\sum_{i=1}^{\infty}\left(x_{i}-y_{i}\right)^{2}\right]^{1 / 2}
$$

But this equation does not always make sense, for the series in question need not converge. (This equation does define a metric on a certain important subset of $\mathbb{R}^{\omega}$, however; see the exercises.)

Similarly, one can attempt to generalize the square metric $\rho$ to $\mathbb{R}^{\omega}$ by defining

$$
\rho(\mathbf{x}, \mathbf{y})=\sup \left\{\left|x_{n}-y_{n}\right|\right\}
$$

Again, this formula does not always make sense. If however we replace the usual metric $d(x, y)=|x-y|$ on $\mathbb{R}$ by its bounded counterpart $\bar{d}(x, y)=\min \{|x-y|, 1\}$, then this definition does make sense; it gives a metric on $\mathbb{R}^{\omega}$ called the uniform metric.

The uniform metric can be defined more generally on the cartesian product $\mathbb{R}^{J}$ for arbitrary $J$, as follows:

Definition. Given an index set $J$, and given points $\mathbf{x}=\left(x_{\alpha}\right)_{\alpha \in J}$ and $\mathbf{y}=\left(y_{\alpha}\right)_{\alpha \in J}$ of $\mathbb{R}^{J}$, let us define a metric $\bar{\rho}$ on $\mathbb{R}^{J}$ by the equation

$$
\bar{\rho}(\mathbf{x}, \mathbf{y})=\sup \left\{\bar{d}\left(x_{\alpha}, y_{\alpha}\right) \mid \alpha \in J\right\}
$$

where $\bar{d}$ is the standard bounded metric on $\mathbb{R}$. It is easy to check that $\bar{\rho}$ is indeed a metric; it is called the uniform metric on $\mathbb{R}^{J}$, and the topology it induces is called the uniform topology.

The relation between this topology and the product and box topologies is the following:

Theorem 20.4. The uniform topology on $\mathbb{R}^{J}$ is finer than the product topology and coarser than the box topology; these three topologies are all different if $J$ is infinite.\\
Proof. Suppose that we are given a point $\mathbf{x}=\left(x_{\alpha}\right)_{\alpha \in J}$ and a product topology basis element $\prod U_{\alpha}$ about $\mathbf{x}$. Let $\alpha_{1}, \ldots, \alpha_{n}$ be the indices for which $U_{\alpha} \neq \mathbb{R}$. Then for each $i$, choose $\epsilon_{i}>0$ so that the $\epsilon_{i}$-ball centered at $x_{\alpha_{i}}$ in the $\bar{d}$ metric is contained in $U_{\alpha_{i}}$; this we can do because $U_{\alpha_{i}}$ is open in $\mathbb{R}$. Let $\epsilon=\min \left\{\epsilon_{1}, \ldots, \epsilon_{n}\right\}$; then the $\epsilon$-ball centered at $\mathbf{x}$ in the $\bar{\rho}$ metric is contained in $\prod U_{\alpha}$. For if $\mathbf{z}$ is a point of $\mathbb{R}^{J}$ such that $\bar{\rho}(\mathbf{x}, \mathbf{z})<\epsilon$, then $\bar{d}\left(x_{\alpha}, z_{\alpha}\right)<\epsilon$ for all $\alpha$, so that $\mathbf{z} \in \Pi U_{\alpha}$. It follows that the uniform topology is finer than the product topology.

On the other hand, let $B$ be the $\epsilon$-ball centered at $\mathbf{x}$ in the $\bar{\rho}$ metric. Then the box neighborhood

$$
U=\prod\left(x_{\alpha}-\frac{1}{2} \epsilon, x_{\alpha}+\frac{1}{2} \epsilon\right)
$$

of $\mathbf{x}$ is contained in $B$. For if $\mathbf{y} \in U$, then $\bar{d}\left(x_{\alpha}, y_{\alpha}\right)<\frac{1}{2} \epsilon$ for all $\boldsymbol{\alpha}$, so that $\bar{\rho}(\mathbf{x}, \mathbf{y}) \leq \frac{1}{2} \epsilon$.

Showing these three topologies are different if $J$ is infinite is a task we leave to the exercises.

In the case where $J$ is infinite, we still have not determined whether $\mathbb{R}^{J}$ is metrizable in either the box or the product topology. It turns out that the only one of these cases where $\mathbb{R}^{J}$ is metrizable is the case where $J$ is countable and $\mathbb{R}^{J}$ has the product topology. As we shall see.

Theorem 20.5. Let $\bar{d}(a, b)=\min \{|a-b|, 1\}$ be the standard bounded metric on $\mathbb{R}$. If $\mathbf{x}$ and $\mathbf{y}$ are two points of $\mathbb{R}^{\omega}$, define

$$
D(\mathbf{x}, \mathbf{y})=\sup \left\{\frac{\bar{d}\left(x_{i}, y_{i}\right)}{i}\right\}
$$

Then $D$ is a metric that induces the product topology on $\mathbb{R}^{\omega}$.\\
Proof. The properties of a metric are satisfied trivially except for the triangle inequality, which is proved by noting that for all $i$,

$$
\frac{\bar{d}\left(x_{i}, z_{i}\right)}{i} \leq \frac{\bar{d}\left(x_{i}, y_{i}\right)}{i}+\frac{\bar{d}\left(y_{i}, z_{i}\right)}{i} \leq D(\mathbf{x}, \mathbf{y})+D(\mathbf{y}, \mathbf{z}),
$$

so that

$$
\sup \left\{\frac{\bar{d}\left(x_{i}, z_{i}\right)}{i}\right\} \leq D(\mathbf{x}, \mathbf{y})+D(\mathbf{y}, \mathbf{z})
$$

The fact that $D$ gives the product topology requires a little more work. First, let $U$ be open in the metric topology and let $\mathbf{x} \in U$; we find an open set $V$ in the product topology such that $\mathbf{x} \in V \subset U$. Choose an $\epsilon$-ball $B_{D}(\mathbf{x}, \epsilon)$ lying in $U$. Then choose $N$ large enough that $1 / N<\epsilon$. Finally, let $V$ be the basis element for the product topology

$$
V=\left(x_{1}-\epsilon, x_{1}+\epsilon\right) \times \cdots \times\left(x_{N}-\epsilon, x_{N}+\epsilon\right) \times \mathbb{R} \times \mathbb{R} \times \cdots
$$

We assert that $V \subset B_{D}(\mathbf{x}, \epsilon)$ : Given any $\mathbf{y}$ in $\mathbb{R}^{\omega}$,

$$
\frac{\bar{d}\left(x_{i}, y_{i}\right)}{i} \leq \frac{1}{N} \quad \text { for } i \geq N
$$

Therefore,

$$
D(\mathbf{x}, \mathbf{y}) \leq \max \left\{\frac{\bar{d}\left(x_{1}, y_{1}\right)}{1}, \cdots, \frac{\bar{d}\left(x_{N}, y_{N}\right)}{N}, \frac{1}{N}\right\}
$$

If $\mathbf{y}$ is in $V$, this expression is less than $\epsilon$, so that $V \subset B_{D}(\mathbf{x}, \epsilon)$, as desired.

Conversely, consider a basis element

$$
U=\prod_{i \in \mathbb{Z}_{+}} U_{i}
$$

for the product topology, where $U_{i}$ is open in $\mathbb{R}$ for $i=\alpha_{1}, \ldots, \alpha_{n}$ and $U_{i}=\mathbb{R}$ for all other indices $i$. Given $\mathbf{x} \in U$, we find an open set $V$ of the metric topology such that $\mathbf{x} \in V \subset U$. Choose an interval ( $x_{i}-\epsilon_{i}, x_{i}+\epsilon_{i}$ ) in $\mathbb{R}$ centered about $x_{i}$ and lying in $U_{i}$ for $i=\alpha_{1}, \ldots, \alpha_{n}$; choose each $\epsilon_{i} \leq 1$. Then define

$$
\epsilon=\min \left\{\epsilon_{i} / i \mid i=\alpha_{1}, \ldots, \alpha_{n}\right\} .
$$

We assert that

$$
\mathbf{x} \in B_{D}(\mathbf{x}, \epsilon) \subset U .
$$

Let $\mathbf{y}$ be a point of $B_{D}(\mathbf{x}, \epsilon)$. Then for all $i$,

$$
\frac{\bar{d}\left(x_{i}, y_{i}\right)}{i} \leq D(\mathbf{x}, \mathbf{y})<\epsilon .
$$

Now if $i=\alpha_{1}, \ldots, \alpha_{n}$, then $\epsilon \leq \epsilon_{i} / i$, so that $\bar{d}\left(x_{i}, y_{i}\right)<\epsilon_{i} \leq 1$; it follows that $\left|x_{i}-y_{i}\right|<\epsilon_{i}$. Therefore, $\mathbf{y} \in \prod U_{i}$, as desired.

\section*{Exercises}
\begin{enumerate}
  \item (a) In $\mathbb{R}^{n}$, define
\end{enumerate}

$$
d^{\prime}(\mathbf{x}, \mathbf{y})=\left|x_{1}-y_{1}\right|+\cdots+\left|x_{n}-y_{n}\right| .
$$

Show that $d^{\prime}$ is a metric that induces the usual topology of $\mathbb{R}^{n}$. Sketch the basis elements under $d^{\prime}$ when $n=2$.\\
(b) More generally, given $p \geq 1$, define

$$
d^{\prime}(\mathbf{x}, \mathbf{y})=\left[\sum_{i=1}^{n}\left|x_{i}-y_{i}\right|^{p}\right]^{1 / p}
$$

for $\mathbf{x}, \mathbf{y} \in \mathbb{R}^{n}$. Assume that $d^{\prime}$ is a metric. Show that it induces the usual topology on $\mathbb{R}^{n}$.\\
2. Show that $\mathbb{R} \times \mathbb{R}$ in the dictionary order topology is metrizable.\\
3. Let $X$ be a metric space with metric $d$.\\
(a) Show that $d: X \times X \rightarrow \mathbb{R}$ is continuous.\\
(b) Let $X^{\prime}$ denote a space having the same underlying set as $X$. Show that if $d: X^{\prime} \times X^{\prime} \rightarrow \mathbb{R}$ is continuous, then the topology of $X^{\prime}$ is finer than the topology of $X$.

One can summarize the result of this exercise as follows: If $X$ has a metric $d$, then the topology induced by $d$ is the coarsest topology relative to which the function $d$ is continuous.\\
4. Consider the product, uniform, and box topologies on $\mathbb{R}^{\omega}$.\\
(a) In which topologies are the following functions from $\mathbb{R}$ to $\mathbb{R}^{\omega}$ continuous?

$$
\begin{aligned}
f(t) & =(t, 2 t, 3 t, \ldots) \\
g(t) & =(t, t, t, \ldots) \\
h(t) & =\left(t, \frac{1}{2} t, \frac{1}{3} t, \ldots\right)
\end{aligned}
$$

(b) In which topologies do the following sequences converge?

$$
\begin{aligned}
\mathbf{w}_{1}=(1,1,1,1, \ldots), & \mathbf{x}_{1}=(1,1,1,1, \ldots) \\
\mathbf{w}_{2}=(0,2,2,2, \ldots), & \mathbf{x}_{2}=\left(0, \frac{1}{2}, \frac{1}{2}, \frac{1}{2}, \ldots\right) \\
\mathbf{w}_{3}=(0,0,3,3, \ldots), & \mathbf{x}_{3}=\left(0,0, \frac{1}{3}, \frac{1}{3} \ldots\right) \\
\ldots & \ldots \\
\mathbf{y}_{1}=(1,0,0,0, \ldots), & \mathbf{z}_{1}=(1,1,0,0, \ldots) \\
\mathbf{y}_{2}=\left(\frac{1}{2}, \frac{1}{2}, 0,0, \ldots\right), & \mathbf{z}_{2}=\left(\frac{1}{2}, \frac{1}{2}, 0,0, \ldots\right) \\
\mathbf{y}_{3}=\left(\frac{1}{3}, \frac{1}{3}, \frac{1}{3}, 0, \ldots\right), & \mathbf{z}_{3}=\left(\frac{1}{3}, \frac{1}{3}, 0,0, \ldots\right) \\
\ldots & \ldots
\end{aligned}
$$

\begin{enumerate}
  \setcounter{enumi}{4}
  \item Let $\mathbb{R}^{\infty}$ be the subset of $\mathbb{R}^{\omega}$ consisting of all sequences that are eventually zero. What is the closure of $\mathbb{R}^{\infty}$ in $\mathbb{R}^{\omega}$ in the uniform topology? Justify your answer.
  \item Let $\bar{\rho}$ be the uniform metric on $\mathbb{R}^{\omega}$. Given $\mathbf{x}=\left(x_{1}, x_{2}, \ldots\right) \in \mathbb{R}^{\omega}$ and given $0<\epsilon<1$, let
\end{enumerate}

$$
U(\mathbf{x}, \epsilon)=\left(x_{1}-\epsilon, x_{1}+\epsilon\right) \times \cdots \times\left(x_{n}-\epsilon, x_{n}+\epsilon\right) \times \cdots .
$$

(a) Show that $U(\mathbf{x}, \epsilon)$ is not equal to the $\epsilon$-ball $B_{\bar{\rho}}(\mathbf{x}, \epsilon)$.\\
(b) Show that $U(\mathbf{x}, \epsilon)$ is not even open in the uniform topology.\\
(c) Show that

$$
B_{\bar{\rho}}(\mathbf{x}, \epsilon)=\bigcup_{\delta<\epsilon} U(\mathbf{x}, \delta)
$$

\begin{enumerate}
  \setcounter{enumi}{6}
  \item Consider the map $h: \mathbb{R}^{\omega} \rightarrow \mathbb{R}^{\omega}$ defined in Exercise 8 of §19; give $\mathbb{R}^{\omega}$ the uniform topology. Under what conditions on the numbers $a_{i}$ and $b_{i}$ is $h$ continuous? a homeomorphism?
  \item Let $X$ be the subset of $\mathbb{R}^{\omega}$ consisting of all sequences $\mathbf{x}$ such that $\sum x_{i}^{2}$ converges. Then the formula
\end{enumerate}

$$
d(\mathbf{x}, \mathbf{y})=\left[\sum_{i=1}^{\infty}\left(x_{i}-y_{i}\right)^{2}\right]^{1 / 2}
$$

defines a metric on $X$. (See Exercise 10.) On $X$ we have the three topologies it inherits from the box, uniform, and product topologies on $\mathbb{R}^{\omega}$. We have also the topology given by the metric $d$, which we call the $\ell^{2}$-topology. (Read "little ell two.")\\
(a) Show that on $X$, we have the inclusions

$$
\text { box topology } \supset \ell^{2} \text {-topology } \supset \text { uniform topology. }
$$

(b) The set $\mathbb{R}^{\infty}$ of all sequences that are eventually zero is contained in $X$. Show that the four topologies that $\mathbb{R}^{\infty}$ inherits as a subspace of $X$ are all distinct.\\
(c) The set

$$
H=\prod_{n \in \mathbb{Z}_{+}}[0,1 / n]
$$

is contained in $X$; it is called the Hilbert cube. Compare the four topologies that $H$ inherits as a subspace of $X$.\\
9. Show that the euclidean metric $d$ on $\mathbb{R}^{n}$ is a metric, as follows: If $\mathbf{x}, \mathbf{y} \in \mathbb{R}^{n}$ and $c \in \mathbb{R}$, define

$$
\begin{aligned}
\mathbf{x}+\mathbf{y} & =\left(x_{1}+y_{1}, \ldots, x_{n}+y_{n}\right) \\
c \mathbf{x} & =\left(c x_{1}, \ldots, c x_{n}\right) \\
\mathbf{x} \cdot \mathbf{y} & =x_{1} y_{1}+\cdots+x_{n} y_{n}
\end{aligned}
$$

(a) Show that $\mathbf{x} \cdot(\mathbf{y}+\mathbf{z})=(\mathbf{x} \cdot \mathbf{y})+(\mathbf{x} \cdot \mathbf{z})$.\\
(b) Show that $|\mathbf{x} \cdot \mathbf{y}| \leq\|\mathbf{x}\|\|\mathbf{y}\|$. [Hint: If $\mathbf{x}, \mathbf{y} \neq 0$, let $a=1 /\|\mathbf{x}\|$ and $b=1 /\|\mathbf{y}\|$, and use the fact that $\|a \mathbf{x} \pm b \mathbf{y}\| \geq 0$.]\\
(c) Show that $\|\mathbf{x}+\mathbf{y}\| \leq\|\mathbf{x}\|+\|\mathbf{y}\|$. [Hint: Compute $(\mathbf{x}+\mathbf{y}) \cdot(\mathbf{x}+\mathbf{y})$ and apply (b).]\\
(d) Verify that $d$ is a metric.\\
10. Let $X$ denote the subset of $\mathbb{R}^{\omega}$ consisting of all sequences ( $x_{1}, x_{2}, \ldots$ ) such that $\sum x_{i}^{2}$ converges. (You may assume the standard facts about infinite series. In case they are not familiar to you, we shall give them in Exercise 11 of the next section.)\\
(a) Show that if $\mathbf{x}, \mathbf{y} \in X$, then $\sum\left|x_{i} y_{i}\right|$ converges. [Hint: Use (b) of Exercise 9 to show that the partial sums are bounded.]\\
(b) Let $c \in \mathbb{R}$. Show that if $\mathbf{x}, \mathbf{y} \in X$, then so are $\mathbf{x}+\mathbf{y}$ and $c \mathbf{x}$.\\
(c) Show that

$$
d(\mathbf{x}, \mathbf{y})=\left[\sum_{i=1}^{\infty}\left(x_{i}-y_{i}\right)^{2}\right]^{1 / 2}
$$

is a well-defined metric on $X$.\\
*11. Show that if $d$ is a metric for $X$, then

$$
d^{\prime}(x, y)=d(x, y) /(1+d(x, y))
$$

is a bounded metric that gives the topology of $X$. [Hint: If $f(x)=x /(1+x)$ for $x>0$, use the mean-value theorem to show that $f(a+b)-f(b) \leq f(a)$.]

\section*{§21 The Metric Topology (continued)}
In this section, we discuss the relation of the metric topology to the concepts we have previously introduced.

Subspaces of metric spaces behave the way one would wish them to; if $A$ is a subspace of the topological space $X$ and $d$ is a metric for $X$, then the restriction of $d$ to $A \times A$ is a metric for the topology of $A$. This we leave to you to check.

About order topologies there is nothing to be said; some are metrizable (for instance, $\mathbb{Z}_{+}$and $\mathbb{R}$ ), and others are not, as we shall see.

The Hausdorff axiom is satisfied by every metric topology. If $x$ and $y$ are distinct points of the metric space ( $X, d$ ), we let $\epsilon=\frac{1}{2} d(x, y)$; then the triangle inequality implies that $B_{d}(x, \epsilon)$ and $B_{d}(y, \epsilon)$ are disjoint.

The product topology we have already considered in special cases; we have proved that the products $\mathbb{R}^{n}$ and $\mathbb{R}^{\omega}$ are metrizable. It is true in general that countable products of metrizable spaces are metrizable; the proof follows a pattern similar to the proof for $\mathbb{R}^{\omega}$, so we leave it to the exercises.

About continuous functions there is a good deal to be said. Consideration of this topic will occupy the remainder of the section.

When we study continuous functions on metric spaces, we are about as close to the study of calculus and analysis as we shall come in this book. There are two things we want to do at this point.

First, we want to show that the familiar " $\epsilon-\delta$ definition" of continuity carries over to general metric spaces, and so does the "convergent sequence definition" of continuity.

Second, we want to consider two additional methods for constructing continuous functions, besides those discussed in §18. One is the process of taking sums, differences, products, and quotients of continuous real-valued functions. The other is the process of taking limits of uniformly convergent sequences of continuous functions.

Theorem 21.1. Let $f: X \rightarrow Y$; let $X$ and $Y$ be metrizable with metrics $d_{X}$ and $d_{Y}$, respectively. Then continuity of $f$ is equivalent to the requirement that given $x \in X$ and given $\epsilon>0$, there exists $\delta>0$ such that

$$
d_{X}(x, y)<\delta \Longrightarrow d_{Y}(f(x), f(y))<\epsilon
$$

Proof. Suppose that $f$ is continuous. Given $x$ and $\epsilon$, consider the set

$$
f^{-1}(B(f(x), \epsilon))
$$

which is open in $X$ and contains the point $x$. It contains some $\delta$-ball $B(x, \delta)$ centered at $x$. If $y$ is in this $\delta$-ball, then $f(y)$ is in the $\epsilon$-ball centered at $f(x)$, as desired.

Conversely, suppose that the $\epsilon-\delta$ condition is satisfied. Let $V$ be open in $Y$; we show that $f^{-1}(V)$ is open in $X$. Let $x$ be a point of the set $f^{-1}(V)$. Since $f(x) \in V$, there is an $\epsilon$-ball $B(f(x), \epsilon)$ centered at $f(x)$ and contained in $V$. By the $\epsilon$ $\delta$ condition, there is a $\delta$-ball $B(x, \delta)$ centered at $x$ such that $f(B(x, \delta)) \subset B(f(x), \epsilon)$. Then $B(x, \delta)$ is a neighborhood of $x$ contained in $f^{-1}(V)$, so that $f^{-1}(V)$ is open, as desired.

Now we turn to the convergent sequence definition of continuity. We begin by considering the relation between convergent sequences and closures of sets. It is certainly believable, from one's experience in analysis, that if $x$ lies in the closure of a subset $A$ of the space $X$, then there should exist a sequence of points of $A$ converging to $x$. This is not true in general, but it is true for metrizable spaces.

Lemma 21.2 (The sequence lemma). Let $X$ be a topological space; let $A \subset X$. If there is a sequence of points of $A$ converging to $x$, then $x \in \bar{A}$; the converse holds if $X$ is metrizable.

Proof. Suppose that $x_{n} \rightarrow x$, where $x_{n} \in A$. Then every neighborhood $U$ of $x$ contains a point of $A$, so $x \in \bar{A}$ by Theorem 17.5. Conversely, suppose that $X$ is metrizable and $x \in \bar{A}$. Let $d$ be a metric for the topology of $X$. For each positive integer $n$, take the neighborhood $B_{d}(x, 1 / n)$ of radius $1 / n$ of $x$, and choose $x_{n}$ to be a point of its intersection with $A$. We assert that the sequence $x_{n}$ converges to $x$ : Any open set $U$ containing $x$ contains an $\epsilon$-ball $B_{d}(x, \epsilon)$ centered at $x$; if we choose $N$ so that $1 / N<\epsilon$, then $U$ contains $x_{i}$ for all $i \geq N$.

Theorem 21.3. Let $f: X \rightarrow Y$. If the function $f$ is continuous, then for every convergent sequence $x_{n} \rightarrow x$ in $X$, the sequence $f\left(x_{n}\right)$ converges to $f(x)$. The converse holds if $X$ is metrizable.\\
Proof. Assume that $f$ is continuous. Given $x_{n} \rightarrow x$, we wish to show that $f\left(x_{n}\right) \rightarrow f(x)$. Let $V$ be a neighborhood of $f(x)$. Then $f^{-1}(V)$ is a neighborhood of $x$, and so there is an $N$ such that $x_{n} \in f^{-1}(V)$ for $n \geq N$. Then $f\left(x_{n}\right) \in V$ for $n \geq N$.

To prove the converse, assume that the convergent sequence condition is satisfied. Let $A$ be a subset of $X$; we show that $f(\bar{A}) \subset \overline{f(A)}$. If $x \in \bar{A}$, then there is a sequence $x_{n}$ of points of $A$ converging to $x$ (by the preceding lemma). By assumption, the sequence $f\left(x_{n}\right)$ converges to $f(x)$. Since $f\left(x_{n}\right) \in f(A)$, the preceding lemma implies that $f(x) \in \overline{f(A)}$. (Note that metrizability of $Y$ is not needed.) Hence $f(\bar{A}) \subset \overline{f(A)}$, as desired.

Incidentally, in proving Lemma 21.2 and Theorem 21.3 we did not use the full strength of the hypothesis that the space $X$ is metrizable. All we really needed was the countable collection $B_{d}(x, 1 / n)$ of balls about $x$. This fact leads us to make a new definition.

A space $X$ is said to have a countable basis at the point $x$ if there is a countable collection $\left\{U_{n}\right\}_{n \in \mathbb{Z}_{+}}$of neighborhoods of $x$ such that any neighborhood $U$ of $x$ contains at\\
least one of the sets $U_{n}$. A space $X$ that has a countable basis at each of its points is said to satisfy the first countability axiom.

If $X$ has a countable basis $\left\{U_{n}\right\}$ at $x$, then the proof of Lemma 21.2 goes through; one simply replaces the ball $B_{d}(x, 1 / n)$ throughout by the set

$$
B_{n}=U_{1} \cap U_{2} \cap \cdots \cap U_{n}
$$

The proof of Theorem 21.3 goes through unchanged.\\
A metrizable space always satisfies the first countability axiom, but the converse is not true, as we shall see. Like the Hausdorff axiom, the first countability axiom is a requirement that we sometimes impose on a topological space in order to prove stronger theorems about the space. We shall study it in more detail in Chapter 4.\\
Now we consider additional methods for constructing continuous functions. We need the following lemma:

Lemma 21.4. The addition, subtraction, and multiplication operations are continuous functions from $\mathbb{R} \times \mathbb{R}$ into $\mathbb{R}$; and the quotient operation is a continuous function from $\mathbb{R} \times(\mathbb{R}-\{0\})$ into $\mathbb{R}$.

You have probably seen this lemma proved before; it is a standard " $\epsilon-\delta$ argument." If not, a proof is outlined in Exercise 12 below; you should have no trouble filling in the details.

Theorem 21.5. If $X$ is a topological space, and if $f, g: X \rightarrow \mathbb{R}$ are continuous functions, then $f+g, f-g$, and $f \cdot g$ are continuous. If $g(x) \neq 0$ for all $x$, then $f / g$ is continuous.\\
Proof. The map $h: X \rightarrow \mathbb{R} \times \mathbb{R}$ defined by

$$
h(x)=f(x) \times g(x)
$$

is continuous, by Theorem 18.4. The function $f+g$ equals the composite of $h$ and the addition operation

$$
+: \mathbb{R} \times \mathbb{R} \rightarrow \mathbb{R}
$$

therefore $f+g$ is continuous. Similar arguments apply to $f-g, f \cdot g$, and $f / g$.\\
Finally, we come to the notion of uniform convergence.\\
Definition. Let $f_{n}: X \rightarrow Y$ be a sequence of functions from the set $X$ to the metric space $Y$. Let $d$ be the metric for $Y$. We say that the sequence $\left(f_{n}\right)$ converges uniformly to the function $f: X \rightarrow Y$ if given $\epsilon>0$, there exists an integer $N$ such that

$$
d\left(f_{n}(x), f(x)\right)<\epsilon
$$

for all $n>N$ and all $x$ in $X$.\\
Uniformity of convergence depends not only on the topology of $Y$ but also on its metric. We have the following theorem about uniformly convergent sequences:

Theorem 21.6 (Uniform limit theorem). Let $f_{n}: X \rightarrow Y$ be a sequence of continuous functions from the topological space $X$ to the metric space $Y$. If $\left(f_{n}\right)$ converges uniformly to $f$, then $f$ is continuous.\\
Proof. Let $V$ be open in $Y$; let $x_{0}$ be a point of $f^{-1}(V)$. We wish to find a neighborhood $U$ of $x_{0}$ such that $f(U) \subset V$.

Let $y_{0}=f\left(x_{0}\right)$. First choose $\epsilon$ so that the $\epsilon$-ball $B\left(y_{0}, \epsilon\right)$ is contained in $V$. Then, using uniform convergence, choose $N$ so that for all $n \geq N$ and all $x \in X$,

$$
d\left(f_{n}(x), f(x)\right)<\epsilon / 3 .
$$

Finally, using continuity of $f_{N}$, choose a neighborhood $U$ of $x_{0}$ such that $f_{N}$ carries $U$ into the $\epsilon / 3$ ball in $Y$ centered at $f_{N}\left(x_{0}\right)$.

We claim that $f$ carries $U$ into $B\left(y_{0}, \epsilon\right)$ and hence into $V$, as desired. For this purpose, note that if $x \in U$, then

$$
\begin{aligned}
d\left(f(x), f_{N}(x)\right)<\epsilon / 3 & (\text { by choice of } N), \\
d\left(f_{N}(x), f_{N}\left(x_{0}\right)\right)<\epsilon / 3 & (\text { by choice of } U), \\
d\left(f_{N}\left(x_{0}\right), f\left(x_{0}\right)\right)<\epsilon / 3 & (\text { by choice of } N) .
\end{aligned}
$$

Adding and using the triangle inequality, we see that $d\left(f(x), f\left(x_{0}\right)\right)<\epsilon$, as desired.

Let us remark that the notion of uniform convergence is related to the definition of the uniform metric, which we gave in the preceding section. Consider, for example, the space $\mathbb{R}^{X}$ of all functions $f: X \rightarrow \mathbb{R}$, in the uniform metric $\bar{\rho}$. It is not difficult to see that a sequence of functions $f_{n}: X \rightarrow \mathbb{R}$ converges uniformly to $f$ if and only if the sequence $\left(f_{n}\right)$ converges to $f$ when they are considered as elements of the metric space ( $\mathbb{R}^{X}, \bar{\rho}$ ). We leave the proof to the exercises.

We conclude the section with some examples of spaces that are not metrizable.\\
EXAMPLE 1. $\quad \mathbb{R}^{\omega}$ in the box topology is not metrizable.\\
We shall show that the sequence lemma does not hold for $\mathbb{R}^{\omega}$. Let $A$ be the subset of $\mathbb{R}^{\omega}$ consisting of those points all of whose coordinates are positive:

$$
A=\left\{\left(x_{1}, x_{2}, \ldots\right) \mid x_{i}>0 \text { for all } i \in \mathbb{Z}_{+}\right\}
$$

Let $\mathbf{0}$ be the "origin" in $\mathbb{R}^{\omega}$, that is, the point ( $0,0, \ldots$ ) each of whose coordinates is zero. In the box topology, 0 belongs to $\bar{A}$; for if

$$
B=\left(a_{1}, b_{1}\right) \times\left(a_{2}, b_{2}\right) \times \cdots
$$

is any basis element containing $\mathbf{0}$, then $B$ intersects $A$. For instance, the point

$$
\left(\frac{1}{2} b_{1}, \frac{1}{2} b_{2} \ldots\right)
$$

belongs to $B \cap A$.\\
But we assert that there is no sequence of points of $A$ converging to $\mathbf{0}$. For let ( $\mathbf{a}_{n}$ ) be a sequence of points of $A$, where

$$
\mathbf{a}_{n}=\left(x_{1 n}, x_{2 n}, \ldots, x_{i n}, \ldots\right)
$$

Every coordinate $x_{i n}$ is positive, so we can construct a basis element $B^{\prime}$ for the box topology on $\mathbb{R}$ by setting

$$
B^{\prime}=\left(-x_{11}, x_{11}\right) \times\left(-x_{22}, x_{22}\right) \times \cdots
$$

Then $B^{\prime}$ contains the origin $\mathbf{0}$, but it contains no member of the sequence $\left(\mathbf{a}_{n}\right)$; the point $\mathbf{a}_{n}$ cannot belong to $\boldsymbol{B}^{\prime}$ because its $n$th coordinate $x_{n n}$ does not belong to the interval ( $-x_{n n}, x_{n n}$ ). Hence the sequence ( $\mathbf{a}_{n}$ ) cannot converge to 0 in the box topology.

\section*{EXAMPLE 2. An uncountable product of $\mathbb{R}$ with itself is not metrizable.}
Let $J$ be an uncountable index set; we show that $\mathbb{R}^{J}$ does not satisfy the sequence lemma (in the product topology).

Let $A$ be the subset of $\mathbb{R}^{J}$ consisting of all points $\left(x_{\alpha}\right)$ such that $x_{\alpha}=1$ for all but finitely many values of $\alpha$. Let 0 be the "origin" in $\mathbb{R}^{J}$, the point each of whose coordinates is 0 .

We assert that 0 belongs to the closure of $A$. Let $\prod U_{\alpha}$ be a basis element containing 0 . Then $U_{\alpha} \neq \mathbb{R}$ for only finitely many values of $\alpha$, say for $\alpha=\alpha_{1}, \ldots, \alpha_{n}$. Let ( $x_{\alpha}$ ) be the point of $A$ defined by letting $x_{\alpha}=0$ for $\alpha=\alpha_{1}, \ldots, \alpha_{n}$ and $x_{\alpha}=1$ for all other values of $\alpha$; then $\left(x_{\alpha}\right) \in A \cap \prod U_{\alpha}$, as desired.

But there is no sequence of points of $A$ converging to 0 . For let $\mathbf{a}_{n}$ be a sequence of points of $A$. Given $n$, let $J_{n}$ denote the subset of $J$ consisting of those indices $\alpha$ for which the $\alpha$ th coordinate of $\mathbf{a}_{n}$ is different from 1 . The union of all the sets $J_{n}$ is a countable union of finite sets and therefore countable. Because $J$ itself is uncountable, there is an index in $J$, say $\beta$, that does not lie in any of the sets $J_{n}$. This means that for each of the points $\mathbf{a}_{n}$, its $\beta$ th coordinate equals 1 .

Now let $U_{\beta}$ be the open interval $(-1,1)$ in $\mathbb{R}$, and let $U$ be the open set $\pi_{\beta}^{-1}\left(U_{\beta}\right)$ in $\mathbb{R}^{J}$. The set $U$ is a neighborhood of $\mathbf{0}$ that contains none of the points $\mathbf{a}_{n}$; therefore, the sequence $\mathbf{a}_{n}$ cannot converge to $\mathbf{0}$.

\section*{Exercises}
\begin{enumerate}
  \item Let $A \subset X$. If $d$ is a metric for the topology of $X$, show that $d \mid A \times A$ is a metric for the subspace topology on $A$.
  \item Let $X$ and $Y$ be metric spaces with metrics $d_{X}$ and $d_{Y}$, respectively. Let $f$ : $X \rightarrow Y$ have the property that for every pair of points $x_{1}, x_{2}$ of $X$,
\end{enumerate}

$$
d_{Y}\left(f\left(x_{1}\right), f\left(x_{2}\right)\right)=d_{X}\left(x_{1}, x_{2}\right)
$$

Show that $f$ is an imbedding. It is called an isometric imbedding of $X$ in $Y$.\\
3. Let $X_{n}$ be a metric space with metric $d_{n}$, for $n \in \mathbb{Z}_{+}$.\\
(a) Show that

$$
\rho(x, y)=\max \left\{d_{1}\left(x_{1}, y_{1}\right), \ldots, d_{n}\left(x_{n}, y_{n}\right)\right\}
$$

is a metric for the product space $X_{1} \times \cdots \times X_{n}$.\\
(b) Let $\bar{d}_{i}=\min \left\{d_{i}, 1\right\}$. Show that

$$
D(x, y)=\sup \left\{\bar{d}_{i}\left(x_{i}, y_{i}\right) / i\right\}
$$

is a metric for the product space $\prod X_{i}$.\\
4. Show that $\mathbb{R}_{\ell}$ and the ordered square satisfy the first countability axiom. (This result does not, of course, imply that they are metrizable.)\\
5. Theorem. Let $x_{n} \rightarrow x$ and $y_{n} \rightarrow y$ in the space $\mathbb{R}$. Then

$$
\begin{aligned}
x_{n}+y_{n} & \rightarrow x+y, \\
x_{n}-y_{n} & \rightarrow x-y, \\
x_{n} y_{n} & \rightarrow x y,
\end{aligned}
$$

and provided that each $y_{n} \neq 0$ and $y \neq 0$,

$$
x_{n} / y_{n} \rightarrow x / y
$$

[Hint: Apply Lemma 21.4; recall from the exercises of §19 that if $x_{n} \rightarrow x$ and $y_{n} \rightarrow y$, then $x_{n} \times y_{n} \rightarrow x \times y$.]\\
6. Define $f_{n}:[0,1] \rightarrow \mathbb{R}$ by the equation $f_{n}(x)=x^{n}$. Show that the sequence ( $f_{n}(x)$ ) converges for each $x \in[0,1]$, but that the sequence $\left(f_{n}\right)$ does not converge uniformly.\\
7. Let $X$ be a set, and let $f_{n}: X \rightarrow \mathbb{R}$ be a sequence of functions. Let $\bar{\rho}$ be the uniform metric on the space $\mathbb{R}^{X}$. Show that the sequence $\left(f_{n}\right)$ converges uniformly to the function $f: X \rightarrow \mathbb{R}$ if and only if the sequence $\left(f_{n}\right)$ converges to $f$ as elements of the metric space $\left(\mathbb{R}^{X}, \bar{\rho}\right)$.\\
8. Let $X$ be a topological space and let $Y$ be a metric space. Let $f_{n}: X \rightarrow Y$ be a sequence of continuous functions. Let $x_{n}$ be a sequence of points of $X$ converging to $x$. Show that if the sequence $\left(f_{n}\right)$ converges uniformly to $f$, then $\left(f_{n}\left(x_{n}\right)\right)$ converges to $f(x)$.\\
9. Let $f_{n}: \mathbb{R} \rightarrow \mathbb{R}$ be the function

$$
f_{n}(x)=\frac{1}{n^{3}[x-(1 / n)]^{2}+1}
$$

See Figure 21.1. Let $f: \mathbb{R} \rightarrow \mathbb{R}$ be the zero function.\\
(a) Show that $f_{n}(x) \rightarrow f(x)$ for each $x \in \mathbb{R}$.\\
(b) Show that $f_{n}$ does not converge uniformly to $f$. (This shows that the converse of Theorem 21.6 does not hold; the limit function $f$ may be continuous even though the convergence is not uniform.)\\
10. Using the closed set formulation of continuity (Theorem 18.1), show that the following are closed subsets of $\mathbb{R}^{2}$ :

$$
\begin{aligned}
A & =\{x \times y \mid x y=1\}, \\
S^{1} & =\left\{x \times y \mid x^{2}+y^{2}=1\right\}, \\
B^{2} & =\left\{x \times y \mid x^{2}+y^{2} \leq 1\right\} .
\end{aligned}
$$

\begin{figure}[h]
\begin{center}
  \includegraphics[width=\textwidth]{2025_11_21_a8ffce9f36674b61ee7eg-061}
\captionsetup{labelformat=empty}
\caption{Figure 21.1}
\end{center}
\end{figure}

The set $B^{2}$ is called the (closed) unit ball in $\mathbb{R}^{2}$.\\
11. Prove the following standard facts about infinite series:\\
(a) Show that if $\left(s_{n}\right)$ is a bounded sequence of real numbers and $s_{n} \leq s_{n+1}$ for each $n$, then $\left(s_{n}\right)$ converges.\\
(b) Let ( $a_{n}$ ) be a sequence of real numbers; define

$$
s_{n}=\sum_{i=1}^{n} a_{i} .
$$

If $s_{n} \rightarrow s$, we say that the infinite series

$$
\sum_{i=1}^{\infty} a_{i}
$$

converges to $s$ also. Show that if $\sum a_{i}$ converges to $s$ and $\sum b_{i}$ converges to $t$, then $\sum\left(c a_{i}+b_{i}\right)$ converges to $c s+t$.\\
(c) Prove the comparison test for infinite series: If $\left|a_{i}\right| \leq b_{i}$ for each $i$, and if the series $\sum b_{i}$ converges, then the series $\sum a_{i}$ converges. [Hint: Show that the series $\sum\left|a_{i}\right|$ and $\sum c_{i}$ converge, where $c_{i}=\left|a_{i}\right|+a_{i}$.]\\
(d) Given a sequence of functions $f_{n}: X \rightarrow \mathbb{R}$, let

$$
s_{n}(x)=\sum_{i=1}^{n} f_{i}(x)
$$

Prove the Weierstrass $M$-test for uniform convergence: If $\left|f_{i}(x)\right| \leq M_{i}$ for all $x \in X$ and all $i$, and if the series $\sum M_{i}$ converges, then the sequence ( $s_{n}$ ) converges uniformly to a function $s$. [Hint: Let $r_{n}=\sum_{i=n+1}^{\infty} M_{i}$. Show that if $k>n$, then $\left|s_{k}(x)-s_{n}(x)\right| \leq r_{n}$; conclude that $\left|s(x)-s_{n}(x)\right| \leq r_{n}$.]\\
12. Prove continuity of the algebraic operations on $\mathbb{R}$, as follows: Use the metric $d(a, b)=|a-b|$ on $\mathbb{R}$ and the metric on $\mathbb{R}^{2}$ given by the equation

$$
\rho\left((x, y),\left(x_{0}, y_{0}\right)\right)=\max \left\{\left|x-x_{0}\right|,\left|y-y_{0}\right|\right\}
$$

(a) Show that addition is continuous. [Hint: Given $\epsilon$, let $\delta=\epsilon / 2$ and note that

$$
\left.d\left(x+y, x_{0}+y_{0}\right) \leq\left|x-x_{0}\right|+\left|y-y_{0}\right| .\right]
$$

(b) Show that multiplication is continuous. [Hint: Given $\left(x_{0}, y_{0}\right)$ and $0<\epsilon<$ 1 , let

$$
3 \delta=\epsilon /\left(\left|x_{0}\right|+\left|y_{0}\right|+1\right)
$$

and note that

$$
\left.d\left(x y, x_{0} y_{0}\right) \leq\left|x_{0}\right|\left|y-y_{0}\right|+\left|y_{0}\right|\left|x-x_{0}\right|+\left|x-x_{0}\right|\left|y-y_{0}\right| .\right]
$$

(c) Show that the operation of taking reciprocals is a continuous map from $\mathbb{R}-\{0\}$ to $\mathbb{R}$. [Hint: Show the inverse image of the interval $(a, b)$ is open. Consider five cases, according as $a$ and $b$ are positive, negative, or zero.]\\
(d) Show that the subtraction and quotient operations are continuous.

\section*{*§22 The Quotient Topology ${ }^{\dagger}$}
Unlike the topologies we have already considered in this chapter, the quotient topology is not a natural generalization of something you have already studied in analysis. Nevertheless, it is easy enough to motivate. One motivation comes from geometry, where one often has occasion to use "cut-and-paste" techniques to construct such geometric objects as surfaces. The torus (surface of a doughnut), for example, can be constructed by taking a rectangle and "pasting" its edges together appropriately, as in Figure 22.1. And the sphere (surface of a ball) can be constructed by taking a disc and collapsing its entire boundary to a single point; see Figure 22.2. Formalizing these constructions involves the concept of quotient topology.

\begin{figure}[h]
\begin{center}
  \includegraphics[width=\textwidth]{2025_11_21_a8ffce9f36674b61ee7eg-062}
\captionsetup{labelformat=empty}
\caption{Figure 22.1}
\end{center}
\end{figure}

\footnotetext{${ }^{\dagger}$ This section will be used throughout Part II of the book. It also is referred to in a number of exercises of Part I.
}\begin{figure}[h]
\begin{center}
  \includegraphics[width=\textwidth]{2025_11_21_a8ffce9f36674b61ee7eg-063}
\captionsetup{labelformat=empty}
\caption{Figure 22.2}
\end{center}
\end{figure}

Definition. Let $X$ and $Y$ be topological spaces; let $p: X \rightarrow Y$ be a surjective map. The map $p$ is said to be a quotient map provided a subset $U$ of $Y$ is open in $Y$ if and only if $p^{-1}(U)$ is open in $X$.

This condition is stronger than continuity; some mathematicians call it "strong continuity." An equivalent condition is to require that a subset $A$ of $Y$ be closed in $Y$ if and only if $p^{-1}(A)$ is closed in $X$. Equivalence of the two conditions follows from equation

$$
f^{-1}(Y-B)=X-f^{-1}(B) .
$$

Another way of describing a quotient map is as follows: We say that a subset $C$ of $X$ is saturated (with respect to the surjective map $p: X \rightarrow Y$ ) if $C$ contains every set $p^{-1}(\{y\})$ that it intersects. Thus $C$ is saturated if it equals the complete inverse image of a subset of $Y$. To say that $p$ is a quotient map is equivalent to saying that $p$ is continuous and $p$ maps saturated open sets of $X$ to open sets of $Y$ (or saturated closed sets of $X$ to closed sets of $Y$ ).

Two special kinds of quotient maps are the open maps and the closed maps. Recall that a map $f: X \rightarrow Y$ is said to be an open map if for each open set $U$ of $X$, the set $f(U)$ is open in $Y$. It is said to be a closed map if for each closed set $A$ of $X$, the set $f(A)$ is closed in $Y$. It follows immediately from the definition that if $p: X \rightarrow Y$ is a surjective continuous map that is either open or closed, then $p$ is a quotient map. There are quotient maps that are neither open nor closed. (See Exercise 3.)

EXAMPLE 1. Let $X$ be the subspace $[0,1] \cup[2,3]$ of $\mathbb{R}$, and let $Y$ be the subspace $[0,2]$ of $\mathbb{R}$. The map $p: X \rightarrow Y$ defined by

$$
p(x)= \begin{cases}x & \text { for } x \in[0,1], \\ x-1 & \text { for } x \in[2,3]\end{cases}
$$

is readily seen to be surjective, continuous, and closed. Therefore it is a quotient map. It is not, however, an open map; the image of the open set $[0,1]$ of $X$ is not open in $Y$.

Note that if $A$ is the subspace $[0,1) \cup[2,3]$ of $X$, then the map $q: A \rightarrow Y$ obtained by restricting $p$ is continuous and surjective, but it is not a quotient map. For the set [2,3] is open in $A$ and is saturated with respect to $q$, but its image is not open in $Y$.

EXAMPLE 2. Let $\pi_{1}: \mathbb{R} \times \mathbb{R} \rightarrow \mathbb{R}$ be projection onto the first coordinate; then $\pi_{1}$ is continuous and surjective. Furthermore, $\pi_{1}$ is an open map. For if $U \times V$ is a nonempty basis element for $\mathbb{R} \times \mathbb{R}$, then $\pi_{1}(U \times V)=U$ is open in $\mathbb{R}$; it follows that $\pi_{1}$ carries open sets of $\mathbb{R} \times \mathbb{R}$ to open sets of $\mathbb{R}$. However, $\pi_{1}$ is not a closed map. The subset

$$
C=\{x \times y \mid x y=1\}
$$

of $\mathbb{R} \times \mathbb{R}$ is closed, but $\pi_{1}(C)=\mathbb{R}-\{0\}$, which is not closed in $\mathbb{R}$.\\
Note that if $A$ is the subspace of $\mathbb{R} \times \mathbb{R}$ that is the union of $C$ and the origin $\{\mathbf{0}\}$, then the map $q: A \rightarrow \mathbb{R}$ obtained by restricting $\pi_{1}$ is continuous and surjective, but it is not a quotient map. For the one-point set $\{\mathbf{0}\}$ is open in $A$ and is saturated with respect to $q$, but its image is not open in $\mathbb{R}$.\\
Now we show how the notion of quotient map can be used to construct a topology on a set.

Definition. If $X$ is a space and $A$ is a set and if $p: X \rightarrow A$ is a surjective map, then there exists exactly one topology $\mathcal{T}$ on $A$ relative to which $p$ is a quotient map; it is called the quotient topology induced by $p$.

The topology $\mathcal{T}$ is of course defined by letting it consist of those subsets $U$ of $A$ such that $p^{-1}(U)$ is open in $X$. It is easy to check that $\mathcal{T}$ is a topology. The sets $\varnothing$ and $A$ are open because $p^{-1}(\varnothing)=\varnothing$ and $p^{-1}(A)=X$. The other two conditions follow from the equations

$$
\begin{aligned}
p^{-1}\left(\bigcup_{\alpha \in J} U_{\alpha}\right) & =\bigcup_{\alpha \in J} p^{-1}\left(U_{\alpha}\right) \\
p^{-1}\left(\bigcap_{i=1}^{n} U_{i}\right) & =\bigcap_{i=1}^{n} p^{-1}\left(U_{i}\right)
\end{aligned}
$$

EXAMPLE 3. Let $p$ be the map of the real line $\mathbb{R}$ onto the three-point set $A=\{a, b, c\}$ defined by

$$
p(x)= \begin{cases}a & \text { if } x>0, \\ b & \text { if } x<0, \\ c & \text { if } x=0 .\end{cases}
$$

You can check that the quotient topology on $A$ induced by $p$ is the one indicated in Figure 22.3.

\begin{figure}[h]
\begin{center}
  \includegraphics[width=\textwidth]{2025_11_21_a8ffce9f36674b61ee7eg-064}
\captionsetup{labelformat=empty}
\caption{Figure 22.3}
\end{center}
\end{figure}

There is a special situation in which the quotient topology occurs particularly frequently. It is the following:

Definition. Let $X$ be a topological space, and let $X^{*}$ be a partition of $X$ into disjoint subsets whose union is $X$. Let $p: X \rightarrow X^{*}$ be the surjective map that carries each point of $X$ to the element of $X^{*}$ containing it. In the quotient topology induced by $p$, the space $X^{*}$ is called a quotient space of $X$.

Given $X^{*}$, there is an equivalence relation on $X$ of which the elements of $X^{*}$ are the equivalence classes. One can think of $X^{*}$ as having been obtained by "identifying" each pair of equivalent points. For this reason, the quotient space $X^{*}$ is often called an identification space, or a decomposition space, of the space $X$.

We can describe the topology of $X^{*}$ in another way. A subset $U$ of $X^{*}$ is a collection of equivalence classes, and the set $p^{-1}(U)$ is just the union of the equivalence classes belonging to $U$. Thus the typical open set of $X^{*}$ is a collection of equivalence classes whose union is an open set of $X$.

Example 4. Let $X$ be the closed unit ball

$$
\left\{x \times y \mid x^{2}+y^{2} \leq 1\right\}
$$

in $\mathbb{R}^{2}$, and let $X^{*}$ be the partition of $X$ consisting of all the one-point sets $\{x \times y\}$ for which $x^{2}+y^{2}<1$, along with the set $\left.S^{1}=\{x \times y\} \mid x^{2}+y^{2}=1\right\}$. Typical saturated open sets in $X$ are pictured by the shaded regions in Figure 22.4. One can show that $X^{*}$ is homeomorphic with the subspace of $\mathbb{R}^{3}$ called the unit 2-sphere, defined by

$$
S^{2}=\left\{(x, y, z) \mid x^{2}+y^{2}+z^{2}=1\right\}
$$

\begin{figure}[h]
\begin{center}
  \includegraphics[width=\textwidth]{2025_11_21_a8ffce9f36674b61ee7eg-065}
\captionsetup{labelformat=empty}
\caption{Figure 22.4}
\end{center}
\end{figure}

EXAMPLE 5. Let $X$ be the rectangle $[0,1] \times[0,1]$. Define a partition $X^{*}$ of $X$ as follows: It consists of all the one-point sets $\{x \times y\}$ where $0<x<1$ and $0<y<1$, the following types of two-point sets:

$$
\begin{array}{ll}
\{x \times 0, x \times 1\} & \text { where } 0<x<1, \\
\{0 \times y, 1 \times y\} & \text { where } 0<y<1,
\end{array}
$$

and the four-point set

$$
\{0 \times 0,0 \times 1,1 \times 0,1 \times 1\} .
$$

Typical saturated open sets in $X$ are pictured by the shaded regions in Figure 22.5; each is an open set of $X$ that equals a union of elements of $X^{*}$.

The image of each of these sets under $p$ is an open set of $X^{*}$, as indicated in Figure 22.6. This description of $X^{*}$ is just the mathematical way of saying what we expressed in pictures when we pasted the edges of a rectangle together to form a torus.

\begin{figure}[h]
\begin{center}
  \includegraphics[width=\textwidth]{2025_11_21_a8ffce9f36674b61ee7eg-066}
\captionsetup{labelformat=empty}
\caption{Figure 22.5}
\end{center}
\end{figure}

\begin{figure}[h]
\begin{center}
  \includegraphics[width=\textwidth]{2025_11_21_a8ffce9f36674b61ee7eg-066(1)}
\captionsetup{labelformat=empty}
\caption{Figure 22.6}
\end{center}
\end{figure}

Now we explore the relationship between the notions of quotient map and quotient space and the concepts introduced previously. It is interesting to note that this relationship is not as simple as one might wish.

We have already noted that subspaces do not behave well; if $p: X \rightarrow Y$ is a quotient map and $A$ is a subspace of $X$, then the map $q: A \rightarrow p(A)$ obtained by restricting $p$ need not be a quotient map. One has, however, the following theorem:\\
Theorem 22.1. Let $p: X \rightarrow Y$ be a quotient map; let $A$ be a subspace of $X$ that is saturated with respect to $p$; let $q: A \rightarrow p(A)$ be the map obtained by restricting $p$.\\
(1) If $A$ is either open or closed in $X$, then $q$ is a quotient map.\\
*(2) If $p$ is either an open map or a closed map, then $q$ is a quotient map.\\
Proof. Step 1. We verify first the following two equations:

$$
\begin{aligned}
q^{-1}(V) & =p^{-1}(V) & & \text { if } V \subset p(A) ; \\
p(U \cap A) & =p(U) \cap p(A) & & \text { if } U \subset X .
\end{aligned}
$$

To check the first equation, we note that since $V \subset p(A)$ and $A$ is saturated, $p^{-1}(V)$ is contained in $A$. It follows that both $p^{-1}(V)$ and $q^{-1}(V)$ equal all points of $A$ that are mapped by $p$ into $V$. To check the second equation, we note that for any two subsets $U$ and $A$ of $X$, we have the inclusion

$$
p(U \cap A) \subset p(U) \cap p(A)
$$

To prove the reverse inclusion, suppose $y=p(u)=p(a)$, for $u \in U$ and $a \in A$. Since $A$ is saturated, $A$ contains the set $p^{-1}(p(a))$, so that in particular $A$ contains $u$. Then $y=p(u)$, where $u \in U \cap A$.

Step 2. Now suppose $A$ is open or $p$ is open. Given the subset $V$ of $p(A)$, we assume that $q^{-1}(V)$ is open in $A$ and show that $V$ is open in $p(A)$.

Suppose first that $A$ is open. Since $q^{-1}(V)$ is open in $A$ and $A$ is open in $X$, the set $q^{-1}(V)$ is open in $X$. Since $q^{-1}(V)=p^{-1}(V)$, the latter set is open in $X$, so that $V$ is open in $Y$ because $p$ is a quotient map. In particular, $V$ is open in $p(A)$.

Now suppose $p$ is open. Since $q^{-1}(V)=p^{-1}(V)$ and $q^{-1}(V)$ is open in $A$, we have $p^{-1}(V)=U \cap A$ for some set $U$ open in $X$. Now $p\left(p^{-1}(V)\right)=V$ because $p$ is surjective; then

$$
V=p\left(p^{-1}(V)\right)=p(U \cap A)=p(U) \cap p(A)
$$

The set $p(U)$ is open in $Y$ because $p$ is an open map; hence $V$ is open in $p(A)$.\\
Step 3. The proof when $A$ or $p$ is closed is obtained by replacing the word "open" by the word "closed" throughout Step 2.

Now we consider other concepts introduced previously. Composites of maps behave nicely; it is easy to check that the composite of two quotient maps is a quotient map; this fact follows from the equation

$$
p^{-1}\left(q^{-1}(U)\right)=(q \circ p)^{-1}(U)
$$

On the other hand, products of maps do not behave well; the cartesian product of two quotient maps need not be a quotient map. See Example 7 following. One needs further conditions on either the maps or the spaces in order for this statement to be true. One such, a condition on the spaces, is called local compactness; we shall study it later. Another, a condition on the maps, is the condition that both the maps $p$ and $q$ be open maps. In that case, it is easy to see that $p \times q$ is also an open map, so it is a quotient map.

Finally, the Hausdorff condition does not behave well; even if $X$ is Hausdorff, there is no reason that the quotient space $X^{*}$ needs to be Hausdorff. There is a simple condition for $X^{*}$ to satisfy the $T_{1}$ axiom; one simply requires that each element of the partition $X^{*}$ be a closed subset of $X$. Conditions that will ensure $X^{*}$ is Hausdorff are harder to find. This is one of the more delicate questions concerning quotient spaces; we shall return to it several times later in the book.

Perhaps the most important result in the study of quotient spaces has to do with the problem of constructing continuous functions on a quotient space. We consider that\\
problem now. When we studied product spaces, we had a criterion for determining whether a map $f: Z \rightarrow \prod X_{\alpha}$ into a product space was continuous. Its counterpart in the theory of quotient spaces is a criterion for determining when a map $f: X^{*} \rightarrow Z$ out of a quotient space is continuous. One has the following theorem:

Theorem 22.2. Let $p: X \rightarrow Y$ be a quotient map. Let $Z$ be a space and let $g: X \rightarrow Z$ be a map that is constant on each set $p^{-1}(\{y\})$, for $y \in Y$. Then $g$ induces a map $f: Y \rightarrow Z$ such that $f \circ p=g$. The induced map $f$ is continuous if and only if $g$ is continuous; $f$ is a quotient map if and only if $g$ is a quotient map.\\
\includegraphics[max width=\textwidth, center]{2025_11_21_a8ffce9f36674b61ee7eg-068(1)}

Proof. For each $y \in Y$, the set $g\left(p^{-1}(\{y\})\right)$ is a one-point set in $Z$ (since $g$ is constant on $\left.p^{-1}(\{y\})\right)$. If we let $f(y)$ denote this point, then we have defined a map $f: Y \rightarrow Z$ such that for each $x \in X, f(p(x))=g(x)$. If $f$ is continuous, then $g=f \circ p$ is continuous. Conversely, suppose $g$ is continuous. Given an open set $V$ of $Z, g^{-1}(V)$ is open in $X$. But $g^{-1}(V)=p^{-1}\left(f^{-1}(V)\right)$; because $p$ is a quotient map, it follows that $f^{-1}(V)$ is open in $Y$. Hence $f$ is continuous.

If $f$ is a quotient map, then $g$ is the composite of two quotient maps and is thus a quotient map. Conversely, suppose that $g$ is a quotient map. Since $g$ is surjective, so is $f$. Let $V$ be a subset of $Z$; we show that $V$ is open in $Z$ if $f^{-1}(V)$ is open in $Y$. Now the set $p^{-1}\left(f^{-1}(V)\right)$ is open in $X$ because $p$ is continuous. Since this set equals $g^{-1}(V)$, the latter is open in $X$. Then because $g$ is a quotient map, $V$ is open in $Z$. \(\square\)

Corollary 22.3. Let $g: X \rightarrow Z$ be a surjective continuous map. Let $X^{*}$ be the following collection of subsets of $X$ :

$$
X^{*}=\left\{g^{-1}(\{z\}) \mid z \in Z\right\} .
$$

Give $X^{*}$ the quotient topology.\\
(a) The map $g$ induces a bijective continuous map $f: X^{*} \rightarrow Z$, which is a homeomorphism if and only if $g$ is a quotient map.\\
\includegraphics[max width=\textwidth, center]{2025_11_21_a8ffce9f36674b61ee7eg-068}\\
(b) If $Z$ is Hausdorff, so is $X^{*}$.

Proof. By the preceding theorem, $g$ induces a continuous map $f: X^{*} \rightarrow Z$; it is clear that $f$ is bijective. Suppose that $f$ is a homeomorphism. Then both $f$ and the\\
projection map $p: X \rightarrow X^{*}$ are quotient maps, so that their composite $q$ is a quotient map. Conversely, suppose that $g$ is a quotient map. Then it follows from the preceding theorem that $f$ is a quotient map. Being bijective, $f$ is thus a homeomorphism.

Suppose $Z$ is Hausdorff. Given distinct points of $X^{*}$, their images under $f$ are distinct and thus possess disjoint neighborhoods $U$ and $V$. Then $f^{-1}(U)$ and $f^{-1}(V)$ are disjoint neighborhoods of the two given points of $X^{*}$.

EXAMPLE 6. Let $X$ be the subspace of $\mathbb{R}^{2}$ that is the union of the line segments $[0,1] \times \{n\}$, for $n \in \mathbb{Z}_{+}$, and let $Z$ be the subspace of $\mathbb{R}^{2}$ consisting of all points of the form $x \times(x / n)$ for $x \in[0,1]$ and $n \in \mathbb{Z}_{+}$. Then $X$ is the union of countably many disjoint line segments, and $Z$ is the union of countably many line segments having an end point in common. See Figure 22.7.

Define a map $g: X \rightarrow Z$ by the equation $g(x \times n)=x \times(x / n)$; then $g$ is surjective and continuous. The quotient space $X^{*}$ whose elements are the sets $g^{-1}(\{z\})$ is simply the space obtained from $X$ by identifying the subset $\{0\} \times \mathbb{Z}_{+}$to a point. The map $g$ induces a bijective continuous map $f: X^{*} \rightarrow Z$. But $f$ is not a homeomorphism.

To verify this fact, it suffices to show that $g$ is not a quotient map. Consider the sequence of points $x_{n}=(1 / n) \times n$ of $X$. The set $A=\left\{x_{n}\right\}$ is a closed subset of $X$ because it has no limit points. Also, it is saturated with respect to $g$. On the other hand, the set $g(A)$ is not closed in $Z$, for it consists of the points $z_{n}=(1 / n) \times\left(1 / n^{2}\right)$; this set has the origin as a limit point.

\begin{figure}[h]
\begin{center}
  \includegraphics[width=\textwidth]{2025_11_21_a8ffce9f36674b61ee7eg-069}
\captionsetup{labelformat=empty}
\caption{Figure 22.7}
\end{center}
\end{figure}

EXAMPLE 7. The product of two quotient maps need not be a quotient map.\\
We give an example that involves non-Hausdorff spaces in the exercises. Here is another involving spaces that are nicer.

Let $X=\mathbb{R}$ and let $X^{*}$ be the quotient space obtained from $X$ by identifying the subset $\mathbb{Z}_{+}$to a point $b$; let $p: X \rightarrow X^{*}$ be the quotient map. Let $\mathbb{Q}$ be the subspace of $\mathbb{R}$ consisting of the rational numbers; let $i: \mathbb{Q} \rightarrow \mathbb{Q}$ be the identity map. We show that

$$
p \times i: X \times \mathbb{Q} \rightarrow X^{*} \times \mathbb{Q}
$$

is not a quotient map.\\
For each $n$, let $c_{n}=\sqrt{2} / n$, and consider the straight lines in $\mathbb{R}^{2}$ with slopes 1 and -1 , respectively, through the point $n \times c_{n}$. Let $U_{n}$ consist of all points of $X \times \mathbb{Q}$ that lie above both of these lines or beneath both of them, and also between the vertical lines $x=n-1 / 4$ and $x=n+1 / 4$. Then $U_{n}$ is open in $X \times \mathbb{Q}$; it contains the set $\{n\} \times \mathbb{Q}$ because $c_{n}$ is not rational. See Figure 22.8.

Let $U$ be the union of the sets $U_{n}$; then $U$ is open in $X \times \mathbb{Q}$. It is saturated with respect to $p \times i$ because it contains the entire set $\mathbb{Z}_{+} \times\{q\}$ for each $q \in \mathbb{Q}$. We assume that $U^{\prime}=(p \times i)(U)$ is open in $X^{*} \times \mathbb{Q}$ and derive a contradiction.

Because $U$ contains, in particular, the set $\mathbb{Z}_{+} \times 0$, the set $U^{\prime}$ contains the point $b \times 0$. Hence $U^{\prime}$ contains an open set of the form $W \times I_{\delta}$, where $W$ is a neighborhood of $b$ in $X^{*}$ and $I_{\delta}$ consists of all rational numbers $y$ with $|y|<\delta$. Then

$$
p^{-1}(W) \times I_{\delta} \subset U .
$$

Choose $n$ large enough that $c_{n}<\delta$. Then since $p^{-1}(W)$ is open in $X$ and contains $\mathbb{Z}_{+}$, we can choose $\epsilon<1 / 4$ so that the interval ( $n-\epsilon, n+\epsilon$ ) is contained in $p^{-1}(W)$. Then $U$ contains the subset $V=(n-\epsilon, n+\epsilon) \times I_{\delta}$ of $X \times \mathbb{Q}$. But the figure makes clear that there are many points $x \times y$ of $V$ that do not lie in $U!$ (One such is the point $x \times y$, where $x=n+\frac{1}{2} \epsilon$ and $y$ is a rational number with $\left|y-c_{n}\right|<\frac{1}{2} \epsilon$.)

\begin{figure}[h]
\begin{center}
  \includegraphics[width=\textwidth]{2025_11_21_a8ffce9f36674b61ee7eg-070}
\captionsetup{labelformat=empty}
\caption{Figure 22.8}
\end{center}
\end{figure}

\section*{Exercises}
\begin{enumerate}
  \item Check the details of Example 3.
  \item (a) Let $p: X \rightarrow Y$ be a continuous map. Show that if there is a continuous map $f: Y \rightarrow X$ such that $p \circ f$ equals the identity map of $Y$, then $p$ is a quotient map.\\
(b) If $A \subset X$, a retraction of $X$ onto $A$ is a continuous map $r: X \rightarrow A$ such that $r(a)=a$ for each $a \in A$. Show that a retraction is a quotient map.
  \item Let $\pi_{1}: \mathbb{R} \times \mathbb{R} \rightarrow \mathbb{R}$ be projection on the first coordinate. Let $A$ be the subspace of $\mathbb{R} \times \mathbb{R}$ consisting of all points $x \times y$ for which either $x \geq 0$ or $y=0$ (or both); let $q: A \rightarrow \mathbb{R}$ be obtained by restricting $\pi_{1}$. Show that $q$ is a quotient map that is neither open nor closed.
  \item (a) Define an equivalence relation on the plane $X=\mathbb{R}^{2}$ as follows:
\end{enumerate}

$$
x_{0} \times y_{0} \sim x_{1} \times y_{1} \quad \text { if } x_{0}+y_{0}^{2}=x_{1}+y_{1}^{2} .
$$

Let $X^{*}$ be the corresponding quotient space. It is homeomorphic to a familiar space; what is it? [Hint: Set $g(x \times y)=x+y^{2}$.]\\
(b) Repeat (a) for the equivalence relation

$$
x_{0} \times y_{0} \sim x_{1} \times y_{1} \quad \text { if } x_{0}^{2}+y_{0}^{2}=x_{1}^{2}+y_{1}^{2} .
$$

\begin{enumerate}
  \setcounter{enumi}{4}
  \item Let $p: X \rightarrow Y$ be an open map. Show that if $A$ is open in $X$, then the map $q: A \rightarrow p(A)$ obtained by restricting $p$ is an open map.
  \item Recall that $\mathbb{R}_{K}$ denotes the real line in the $K$-topology. (See § 13.) Let $Y$ be the quotient space obtained from $\mathbb{R}_{K}$ by collapsing the set $K$ to a point; let $p: \mathbb{R}_{K} \rightarrow Y$ be the quotient map.\\
(a) Show that $Y$ satisfies the $T_{1}$ axiom, but is not Hausdorff.\\
(b) Show that $p \times p: \mathbb{R}_{K} \times \mathbb{R}_{K} \rightarrow Y \times Y$ is not a quotient map. [Hint: The diagonal is not closed in $Y \times Y$, but its inverse image is closed in $\mathbb{R}_{K} \times \mathbb{R}_{K}$.]
\end{enumerate}

\section*{*Supplementary Exercises: Topological Groups}
In these exercises we consider topological groups and some of their properties. The quotient topology gets its name from the special case that arises when one forms the quotient of a topological group by a subgroup.

A topological group $G$ is a group that is also a topological space satisfying the $T_{1}$ axiom, such that the map of $G \times G$ into $G$ sending $x \times y$ into $x \cdot y$, and the map of $G$ into $G$ sending $x$ into $x^{-1}$, are continuous maps. Throughout the following exercises, let $G$ denote a topological group.

\begin{enumerate}
  \item Let $H$ denote a group that is also a topological space satisfying the $T_{1}$ axiom. Show that $H$ is a topological group if and only if the map of $H \times H$ into $H$ sending $x \times y$ into $x \cdot y^{-1}$ is continuous.
  \item Show that the following are topological groups:\\
(a) $(\mathbb{Z},+)$\\
(b) $(\mathbb{R},+)$\\
(c) $\left(\mathbb{R}_{+}, \cdot\right)$\\
(d) $\left(S^{1}, \cdot\right)$, where we take $S^{1}$ to be the space of all complex numbers $z$ for which $|z|=1$.\\
(e) The general linear group $\mathrm{GL}(n)$, under the operation of matrix multiplication. (GL( $n$ ) is the set of all nonsingular $n$ by $n$ matrices, topologized by considering it as a subset of euclidean space of dimension $n^{2}$ in the obvious way.)
  \item Let $H$ be a subspace of $G$. Show that if $H$ is also a subgroup of $G$, then both $H$ and $\bar{H}$ are topological groups.
  \item Let $\alpha$ be an element of $G$. Show that the maps $f_{\alpha}, g_{\alpha}: G \rightarrow G$ defined by
\end{enumerate}

$$
f_{\alpha}(x)=\alpha \cdot x \quad \text { and } \quad g_{\alpha}(x)=x \cdot \alpha
$$

are homeomorphisms of $G$. Conclude that $G$ is a homogeneous space. (This means that for every pair $x, y$ of points of $G$, there exists a homeomorphism of $G$ onto itself that carries $x$ to $y$.)\\
5. Let $H$ be a subgroup of $G$. If $x \in G$, define $x H=\{x \cdot h \mid h \in H\}$; this set is called a left coset of $H$ in $G$. Let $G / H$ denote the collection of left cosets of $H$ in $G$; it is a partition of $G$. Give $G / H$ the quotient topology.\\
(a) Show that if $\alpha \in G$, the map $f_{\alpha}$ of the preceding exercise induces a homeomorphism of $G / H$ carrying $x H$ to $(\alpha \cdot x) H$. Conclude that $G / H$ is a homogeneous space.\\
(b) Show that if $H$ is a closed set in the topology of $G$, then one-point sets are closed in $G / H$.\\
(c) Show that the quotient map $p: G \rightarrow G / H$ is open.\\
(d) Show that if $H$ is closed in the topology of $G$ and is a normal subgroup of $G$, then $G / H$ is a topological group.\\
6. The integers $\mathbb{Z}$ are a normal subgroup of ( $\mathbb{R},+$ ). The quotient $\mathbb{R} / \mathbb{Z}$ is a familiar topological group; what is it?\\
7. If $A$ and $B$ are subsets of $G$, let $A \cdot B$ denote the set of all points $a \cdot b$ for $a \in A$ and $b \in B$. Let $A^{-1}$ denote the set of all points $a^{-1}$, for $a \in A$.\\
(a) A neighborhood $V$ of the identity element $e$ is said to be symmetric if $V= V^{-1}$. If $U$ is a neighborhood of $e$, show there is a symmetric neighborhood $V$ of $e$ such that $V \cdot V \subset U$. [Hint: If $W$ is a neighborhood of $e$, then $W \cdot W^{-1}$ is symmetric.]\\
(b) Show that $G$ is Hausdorff. In fact, show that if $x \neq y$, there is a neighborhood $V$ of $e$ such that $V \cdot x$ and $V \cdot y$ are disjoint.\\
(c) Show that $G$ satisfies the following separation axiom, which is called the regularity axiom: Given a closed set $A$ and a point $x$ not in $A$, there exist disjoint open sets containing $A$ and $x$, respectively. [Hint: There is a neighborhood $V$ of $e$ such that $V \cdot x$ and $V \cdot A$ are disjoint.]\\
(d) Let $H$ be a subgroup of $G$ that is closed in the topology of $G$; let $p: G \rightarrow G / H$ be the quotient map. Show that $G / H$ satisfies the regularity axiom.\\[0pt]
[Hint: Examine the proof of (c) when $A$ is saturated.]

\section*{Chapter 3}
\section*{Connectedness and Compactness}
In the study of calculus, there are three basic theorems about continuous functions, and on these theorems the rest of calculus depends. They are the following:

Intermediate value theorem. If $f:[a, b] \rightarrow \mathbb{R}$ is continuous and if $r$ is a real number between $f(a)$ and $f(b)$, then there exists an element $c \in[a, b]$ such that $f(c)=r$.

Maximum value theorem. If $f:[a, b] \rightarrow R$ is continuous, then there exists an element $c \in[a, b]$ such that $f(x) \leq f(c)$ for every $x \in[a, b]$.

Uniform continuity theorem. If $f:[a, b] \rightarrow \mathbb{R}$ is continuous, then given $\epsilon>0$, there exists $\delta>0$ such that $\left|f\left(x_{1}\right)-f\left(x_{2}\right)\right|<\epsilon$ for every pair of numbers $x_{1}, x_{2}$ of $[a, b]$ for which $\left|x_{1}-x_{2}\right|<\delta$.

These theorems are used in a number of places. The intermediate value theorem is used for instance in constructing inverse functions, such as $\sqrt[3]{x}$ and $\arcsin x$; and the maximum value theorem is used for proving the mean value theorem for derivatives, upon which the two fundamental theorems of calculus depend. The uniform continuity theorem is used, among other things, for proving that every continuous function is integrable.

We have spoken of these three theorems as theorems about continuous functions. But they can also be considered as theorems about the closed interval $[a, b]$ of real numbers. The theorems depend not only on the continuity of $f$ but also on properties of the topological space $[a, b]$.

The property of the space $[a, b]$ on which the intermediate value theorem depends\\
is the property called connectedness, and the property on which the other two depend is the property called compactness. In this chapter, we shall define these properties for arbitrary topological spaces, and shall prove the appropriate generalized versions of these theorems.

As the three quoted theorems are fundamental for the theory of calculus, so are the notions of connectedness and compactness fundamental in higher analysis, geometry, and topology-indeed, in almost any subject for which the notion of topological space itself is relevant.

\section*{§23 Connected Spaces}
The definition of connectedness for a topological space is a quite natural one. One says that a space can be "separated" if it can be broken up into two "globs"-disjoint open sets. Otherwise, one says that it is connected. From this simple idea much follows.

Definition. Let $X$ be a topological space. A separation of $X$ is a pair $U, V$ of disjoint nonempty open subsets of $X$ whose union is $X$. The space $X$ is said to be connected if there does not exist a separation of $X$.

Connectedness is obviously a topological property, since it is formulated entirely in terms of the collection of open sets of $X$. Said differently, if $X$ is connected, so is any space homeomorphic to $X$.

Another way of formulating the definition of connectedness is the following:

\begin{displayquote}
A space $X$ is connected if and only if the only subsets of $X$ that are both open and closed in $X$ are the empty set and $X$ itself.
\end{displayquote}

For if $A$ is a nonempty proper subset of $X$ that is both open and closed in $X$, then the sets $U=A$ and $V=X-A$ constitute a separation of $X$, for they are open, disjoint, and nonempty, and their union is $X$. Conversely, if $U$ and $V$ form a separation of $X$, then $U$ is nonempty and different from $X$, and it is both open and closed in $X$.

For a subspace $Y$ of a topological space $X$, there is another useful way of formulating the definition of connectedness:

Lemma 23.1. If $Y$ is a subspace of $X$, a separation of $Y$ is a pair of disjoint nonempty sets $A$ and $B$ whose union is $Y$, neither of which contains a limit point of the other. The space $Y$ is connected if there exists no separation of $Y$.

Proof. Suppose first that $A$ and $B$ form a separation of $Y$. Then $A$ is both open and closed in $Y$. The closure of $A$ in $Y$ is the set $\bar{A} \cap Y$ (where $\bar{A}$ as usual denotes the closure of $A$ in $X$ ). Since $A$ is closed in $Y, A=\bar{A} \cap Y$; or to say the same thing, $\bar{A} \cap B=\varnothing$. Since $\bar{A}$ is the union of $A$ and its limit points, $B$ contains no limit points of $A$. A similar argument shows that $A$ contains no limit points of $B$.

Conversely, suppose that $A$ and $B$ are disjoint nonempty sets whose union is $Y$, neither of which contains a limit point of the other. Then $\bar{A} \cap B=\varnothing$ and $A \cap \bar{B}=\varnothing$;\\
therefore, we conclude that $\bar{A} \cap Y=A$ and $\bar{B} \cap Y=B$. Thus both $A$ and $B$ are closed in $Y$, and since $A=Y-B$ and $B=Y-A$, they are open in $Y$ as well. \(\square\)

EXAMPLE 1. Let $X$ denote a two-point space in the indiscrete topology. Obviously there is no separation of $X$, so $X$ is connected.

EXAMPLE 2. Let $Y$ denote the subspace $[-1,0) \cup(0,1]$ of the real line $\mathbb{R}$. Each of the sets $[-1,0)$ and $(0,1]$ is nonempty and open in $Y$ (although not in $\mathbb{R}$ ); therefore, they form a separation of $Y$. Alternatively, note that neither of these sets contains a limit point of the other. (They do have a limit point 0 in common, but that does not matter.)

EXAMPLE 3. Let $X$ be the subspace $[-1,1]$ of the real line. The sets $[-1,0]$ and $(0,1]$ are disjoint and nonempty, but they do not form a separation of $X$, because the first set is not open in $X$. Alternatively, note that the first set contains a limit point, 0 , of the second. Indeed, there exists no separation of the space $[-1,1]$. We shall prove this fact shortly.

Example 4. The rationals $\mathbb{Q}$ are not connected. Indeed, the only connected subspaces of $\mathbb{Q}$ are the one-point sets: If $Y$ is a subspace of $\mathbb{Q}$ containing two points $p$ and $q$, one can choose an irrational number $a$ lying between $p$ and $q$, and write $Y$ as the union of the open sets

$$
Y \cap(-\infty, a) \quad \text { and } \quad Y \cap(a,+\infty) .
$$

EXAMPLE 5. Consider the following subset of the plane $\mathbb{R}^{2}$ :

$$
X=\{x \times y \mid y=0\} \cup\{x \times y \mid x>0 \text { and } y=1 / x\} .
$$

Then $X$ is not connected; indeed, the two indicated sets form a separation of $X$ because neither contains a limit point of the other. See Figure 23.1.

\begin{figure}[h]
\begin{center}
  \includegraphics[width=\textwidth]{2025_11_21_a8ffce9f36674b61ee7eg-075}
\captionsetup{labelformat=empty}
\caption{Figure 23.1}
\end{center}
\end{figure}

We have given several examples of spaces that are not connected. How can one construct spaces that are connected? We shall now prove several theorems that tell how to form new connected spaces from given ones. In the next section we shall apply these theorems to show that some specific spaces, such as intervals in $\mathbb{R}$, and balls and cubes in $\mathbb{R}^{n}$, are connected. First, a lemma:

Lemma 23.2. If the sets $C$ and $D$ form a separation of $X$, and if $Y$ is a connected subspace of $X$, then $Y$ lies entirely within either $C$ or $D$.\\
Proof. Since $C$ and $D$ are both open in $X$, the sets $C \cap Y$ and $D \cap Y$ are open in $Y$. These two sets are disjoint and their union is $Y$; if they were both nonempty, they would constitute a separation of $Y$. Therefore, one of them is empty. Hence $Y$ must lie entirely in $C$ or in $D$. \(\square\)

Theorem 23.3. The union of a collection of connected subspaces of $X$ that have a point in common is connected.

Proof. Let $\left\{A_{\alpha}\right\}$ be a collection of connected subspaces of a space $X$; let $p$ be a point of $\bigcap A_{\alpha}$. We prove that the space $Y=\bigcup A_{\alpha}$ is connected. Suppose that $Y=C \cup D$ is a separation of $Y$. The point $p$ is in one of the sets $C$ or $D$; suppose $p \in C$. Since $A_{\alpha}$ is connected, it must lie entirely in either $C$ or $D$, and it cannot lie in $D$ because it contains the point $p$ of $C$. Hence $A_{\alpha} \subset C$ for every $\alpha$, so that $\bigcup A_{\alpha} \subset C$, contradicting the fact that $D$ is nonempty. \(\square\)

Theorem 23.4. Let $A$ be a connected subspace of $X$. If $A \subset B \subset \bar{A}$, then $B$ is also connected.

Said differently: If $B$ is formed by adjoining to the connected subspace $A$ some or all of its limit points, then $B$ is connected.\\
Proof. Let $A$ be connected and let $A \subset B \subset \bar{A}$. Suppose that $B=C \cup D$ is a separation of $B$. By Lemma 23.2, the set $A$ must lie entirely in $C$ or in $D$; suppose that $A \subset C$. Then $\bar{A} \subset \bar{C}$; since $\bar{C}$ and $D$ are disjoint, $B$ cannot intersect $D$. This contradicts the fact that $D$ is a nonempty subset of $B$. \(\square\)

Theorem 23.5. The image of a connected space under a continuous map is connected.

Proof. Let $f: X \rightarrow Y$ be a continuous map; let $X$ be connected. We wish to prove the image space $Z=f(X)$ is connected. Since the map obtained from $f$ by restricting its range to the space $Z$ is also continuous, it suffices to consider the case of a continuous surjective map

$$
g: X \rightarrow Z .
$$

Suppose that $Z=A \cup B$ is a separation of $Z$ into two disjoint nonempty sets open in $Z$. Then $g^{-1}(A)$ and $g^{-1}(B)$ are disjoint sets whose union is $X$; they are open in $X$ because $g$ is continuous, and nonempty because $g$ is surjective. Therefore, they form a separation of $X$, contradicting the assumption that $X$ is connected. \(\square\)

Theorem 23.6. A finite cartesian product of connected spaces is connected.\\
Proof. We prove the theorem first for the product of two connected spaces $X$ and $Y$. This proof is easy to visualize. Choose a "base point" $a \times b$ in the product $X \times Y$. Note that the "horizontal slice" $X \times b$ is connected, being homeomorphic with $X$, and each "vertical slice" $x \times Y$ is connected, being homeomorphic with $Y$. As a result, each "T-shaped" space

$$
T_{x}=(X \times b) \cup(x \times Y)
$$

is connected, being the union of two connected spaces that have the point $x \times b$ in common. See Figure 23.2. Now form the union $\bigcup_{x \in X} T_{x}$ of all these T-shaped spaces.

This union is connected because it is the union of a collection of connected spaces that have the point $a \times b$ in common. Since this union equals $X \times Y$, the space $X \times Y$ is connected.

\begin{figure}[h]
\begin{center}
  \includegraphics[width=\textwidth]{2025_11_21_a8ffce9f36674b61ee7eg-077}
\captionsetup{labelformat=empty}
\caption{Figure 23.2}
\end{center}
\end{figure}

The proof for any finite product of connected spaces follows by induction, using the fact (easily proved) that $X_{1} \times \cdots \times X_{n}$ is homeomorphic with $\left(X_{1} \times \cdots \times X_{n-1}\right) \times X_{n}$. \(\square\)

It is natural to ask whether this theorem extends to arbitrary products of connected spaces. The answer depends on which topology is used for the product, as the following examples show.

Example 6. Consider the cartesian product $\mathbb{R}^{\omega}$ in the box topology. We can write $\mathbb{R}^{\omega}$ as the union of the set $A$ consisting of all bounded sequences of real numbers, and the set $B$ of all unbounded sequences. These sets are disjoint, and each is open in the box topology. For if $\mathbf{a}$ is a point of $\mathbb{R}^{\omega}$, the open set

$$
U=\left(a_{1}-1, a_{1}+1\right) \times\left(a_{2}-1, a_{2}+1\right) \times \cdots
$$

consists entirely of bounded sequences if $\mathbf{a}$ is bounded, and of unbounded sequences if $\mathbf{a}$ if unbounded. Thus, even though $\mathbb{R}$ is connected (as we shall prove in the next section), $\mathbb{R}^{\omega}$ is not connected in the box topology.

Example 7. Now consider $\mathbb{R}^{\omega}$ in the product topology. Assuming that $\mathbb{R}$ is connected, we show that $\mathbb{R}^{\omega}$ is connected. Let $\tilde{\mathbb{R}}^{n}$ denote the subspace of $\mathbb{R}^{\omega}$ consisting of all sequences $\mathbf{x}=\left(x_{1}, x_{2}, \ldots\right)$ such that $x_{i}=0$ for $i>n$. The space $\tilde{\mathbb{R}}^{n}$ is clearly homeomorphic to $\mathbb{R}^{n}$, so that it is connected, by the preceding theorem. It follows that the space $\mathbb{R}^{\infty}$ that is the union of the spaces $\tilde{\mathbb{R}}^{n}$ is connected, for these spaces have the point $\mathbf{0}=(0,0, \ldots)$ in common. We show that the closure of $\mathbb{R}^{\infty}$ equals all of $\mathbb{R}^{\omega}$, from which it follows that $\mathbb{R}^{\omega}$ is connected as well.

Let $\mathbf{a}=\left(a_{1}, a_{2}, \ldots\right)$ be a point of $\mathbb{R}^{\omega}$. Let $U=\prod U_{i}$ be a basis element for the product topology that contains $\mathbf{a}$. We show that $U$ intersects $\mathbb{R}^{\infty}$. There is an integer $N$ such that $U_{i}=\mathbb{R}$ for $i>N$. Then the point

$$
\mathbf{x}=\left(a_{1}, \ldots, a_{n}, 0,0, \ldots\right)
$$

of $\mathbb{R}^{\infty}$ belongs to $U$, since $a_{i} \in U_{i}$ for all $i$, and $0 \in U_{i}$ for $i>N$.

The argument just given generalizes to show that an arbitrary product of connected spaces is connected in the product topology. Since we shall not need this result, we leave the proof to the exercises.

\section*{Exercises}
\begin{enumerate}
  \item Let $\mathcal{T}$ and $\mathcal{T}^{\prime}$ be two topologies on $X$. If $\mathcal{T}^{\prime} \supset \mathcal{T}$, what does connectedness of $X$ in one topology imply about connectedness in the other?
  \item Let $\left\{A_{n}\right\}$ be a sequence of connected subspaces of $X$, such that $A_{n} \cap A_{n+1} \neq \varnothing$ for all $n$. Show that $\cup A_{n}$ is connected.
  \item Let $\left\{A_{\alpha}\right\}$ be a collection of connected subspaces of $X$; let $A$ be a connected subspace of $X$. Show that if $A \cap A_{\alpha} \neq \varnothing$ for all $\alpha$, then $A \cup\left(\cup A_{\alpha}\right)$ is connected.
  \item Show that if $X$ is an infinite set, it is connected in the finite complement topology.
  \item A space is totally disconnected if its only connected subspaces are one-point sets. Show that if $X$ has the discrete topology, then $X$ is totally disconnected. Does the converse hold?
  \item Let $A \subset X$. Show that if $C$ is a connected subspace of $X$ that intersects both $A$ and $X-A$, then $C$ intersects $\operatorname{Bd} A$.
  \item Is the space $\mathbb{R}_{\ell}$ connected? Justify your answer.
  \item Determine whether or not $\mathbb{R}^{\omega}$ is connected in the uniform topology.
  \item Let $A$ be a proper subset of $X$, and let $B$ be a proper subset of $Y$. If $X$ and $Y$ are connected, show that
\end{enumerate}

$$
(X \times Y)-(A \times B)
$$

is connected.\\
10. Let $\left\{X_{\alpha}\right\}_{\alpha \in J}$ be an indexed family of connected spaces; let $X$ be the product space

$$
X=\prod_{\alpha \in J} X_{\alpha} .
$$

Let $\mathbf{a}=\left(a_{\alpha}\right)$ be a fixed point of $X$.\\
(a) Given any finite subset $K$ of $J$, let $X_{K}$ denote the subspace of $X$ consisting of all points $\mathbf{x}=\left(x_{\alpha}\right)$ such that $x_{\alpha}=a_{\alpha}$ for $\alpha \notin K$. Show that $X_{K}$ is connected.\\
(b) Show that the union $Y$ of the spaces $X_{K}$ is connected.\\
(c) Show that $X$ equals the closure of $Y$; conclude that $X$ is connected.\\
11. Let $p: X \rightarrow Y$ be a quotient map. Show that if each set $p^{-1}(\{y\})$ is connected, and if $Y$ is connected, then $X$ is connected.\\
12. Let $Y \subset X$; let $X$ and $Y$ be connected. Show that if $A$ and $B$ form a separation of $X-Y$, then $Y \cup A$ and $Y \cup B$ are connected.

\section*{§24 Connected Subspaces of the Real Line}
The theorems of the preceding section show us how to construct new connected spaces out of given ones. But where can we find some connected spaces to start with? The best place to begin is the real line. We shall prove that $\mathbb{R}$ is connected, and so are the intervals and rays in $\mathbb{R}$.

One application is the intermediate value theorem of calculus, suitably generalized. Another is the result that such familiar spaces as balls and spheres in euclidean space are connected; the proof involves a new notion, called path connectedness, which we also discuss.

The fact that intervals and rays in $\mathbb{R}$ are connected may be familiar to you from analysis. We prove it again here, in generalized form. It turns out that this fact does not depend on the algebraic properties of $\mathbb{R}$, but only on its order properties. To make this clear, we shall prove the theorem for an arbitrary ordered set that has the order properties of $\mathbb{R}$. Such a set is called a linear continuum.

Definition. A simply ordered set $L$ having more than one element is called a linear continuum if the following hold:\\
(1) $L$ has the least upper bound property.\\
(2) If $x<y$, there exists $z$ such that $x<z<y$.

Theorem 24.1. If $L$ is a linear continuum in the order topology, then $L$ is connected, and so are intervals and rays in $L$.

Proof. Recall that a subspace $Y$ of $L$ is said to be convex if for every pair of points $a, b$ of $Y$ with $a<b$, the entire interval $[a, b]$ of points of $L$ lies in $Y$. We prove that if $Y$ is a convex subspace of $L$, then $Y$ is connected.

So suppose that $Y$ is the union of the disjoint nonempty sets $A$ and $B$, each of which is open in $Y$. Choose $a \in A$ and $b \in B$; suppose for convenience that $a<b$. The interval $[a, b]$ of points of $L$ is contained in $Y$. Hence $[a, b]$ is the union of the disjoint sets

$$
A_{0}=A \cap[a, b] \quad \text { and } \quad B_{0}=B \cap[a, b],
$$

each of which is open in $[a, b]$ in the subspace topology, which is the same as the order topology. The sets $A_{0}$ and $B_{0}$ are nonempty because $a \in A_{0}$ and $b \in B_{0}$. Thus, $A_{0}$ and $B_{0}$ constitute a separation of $[a, b]$.

Let $c=\sup A_{0}$. We show that $c$ belongs neither to $A_{0}$ nor to $B_{0}$, which contradicts the fact that $[a, b]$ is the union of $A_{0}$ and $B_{0}$.

Case 1. Suppose that $c \in B_{0}$. Then $c \neq a$, so either $c=b$ or $a<c<b$. In either case, it follows from the fact that $B_{0}$ is open in $[a, b]$ that there is some interval of the form $(d, c]$ contained in $B_{0}$. If $c=b$, we have a contradiction at once, for $d$ is a smaller upper bound on $A_{0}$ than $c$. If $c<b$, we note that ( $c, b$ ] does not intersect $A_{0}$\\
(because $c$ is an upper bound on $A_{0}$ ). Then

$$
(d, b]=(d, c] \cup(c, b]
$$

does not intersect $A_{0}$. Again, $d$ is a smaller upper bound on $A_{0}$ than $c$, contrary to construction. See Figure 24.1.

\begin{figure}[h]
\begin{center}
  \includegraphics[width=\textwidth]{2025_11_21_a8ffce9f36674b61ee7eg-080(1)}
\captionsetup{labelformat=empty}
\caption{Figure 24.1}
\end{center}
\end{figure}

\begin{figure}[h]
\begin{center}
  \includegraphics[width=\textwidth]{2025_11_21_a8ffce9f36674b61ee7eg-080}
\captionsetup{labelformat=empty}
\caption{Figure 24.2}
\end{center}
\end{figure}

Case 2. Suppose that $c \in A_{0}$. Then $c \neq b$, so either $c=a$ or $a<c<b$. Because $A_{0}$ is open in $[a, b]$, there must be some interval of the form $[c, e)$ contained in $A_{0}$. See Figure 24.2. Because of order property (2) of the linear continuum $L$, we can choose a point $z$ of $L$ such that $c<z<e$. Then $z \in A_{0}$, contrary to the fact that $c$ is an upper bound for $A_{0}$.

Corollary 24.2. The real line $\mathbb{R}$ is connected and so are intervals and rays in $\mathbb{R}$.\\
As an application, we prove the intermediate value theorem of calculus, suitably generalized.

Theorem 24.3 (Intermediate value theorem). Let $f: X \rightarrow Y$ be a continuous map, where $X$ is a connected space and $Y$ is an ordered set in the order topology. If a and $b$ are two points of $X$ and if $r$ is a point of $Y$ lying between $f(a)$ and $f(b)$, then there exists a point $c$ of $X$ such that $f(c)=r$.

The intermediate value theorem of calculus is the special case of this theorem that occurs when we take $X$ to be a closed interval in $\mathbb{R}$ and $Y$ to be $\mathbb{R}$.

Proof. Assume the hypotheses of the theorem. The sets

$$
A=f(X) \cap(-\infty, r) \quad \text { and } \quad B=f(X) \cap(r,+\infty)
$$

are disjoint, and they are nonempty because one contains $f(a)$ and the other contains $f(b)$. Each is open in $f(X)$, being the intersection of an open ray in $Y$ with $f(X)$. If there were no point $c$ of $X$ such that $f(c)=r$, then $f(X)$ would be the union of the sets $A$ and $B$. Then $A$ and $B$ would constitute a separation of $f(X)$, contradicting the fact that the image of a connected space under a continuous map is connected. \(\square\)

EXAMPLE 1. One example of a linear continuum different from $\mathbb{R}$ is the ordered square. We check the least upper bound property. (The second property of a linear continuum is trivial to check.) Let $A$ be a subset of $I \times I$; let $\pi_{1}: I \times I \rightarrow I$ be projection on the first coordinate; let $b=\sup \pi_{1}(A)$. If $b \in \pi_{1}(A)$, then $A$ intersects the subset $b \times I$ of $I \times I$. Because $b \times I$ has the order type of $I$, the set $A \cap(b \times I)$ will have a least upper bound $b \times c$, which will be the least upper bound of $A$. See Figure 24.3. If $b \notin \pi_{1}(A)$, then $b \times 0$ is the least upper bound of $A$; no element of the form $b^{\prime} \times c$ with $b^{\prime}<b$ can be an upper bound for $A$, for then $b^{\prime}$ would be an upper bound for $\pi_{1}(A)$.

\begin{figure}[h]
\begin{center}
  \includegraphics[width=\textwidth]{2025_11_21_a8ffce9f36674b61ee7eg-081}
\captionsetup{labelformat=empty}
\caption{Figure 24.3}
\end{center}
\end{figure}

Example 2. If $X$ is a well-ordered set, then $X \times[0,1)$ is a linear continuum in the dictionary order; this we leave to you to check. This set can be thought of as having been constructed by "fitting in" a set of the order type of $(0,1)$ immediately following each element of $X$.

Connectedness of intervals in $\mathbb{R}$ gives rise to an especially useful criterion for showing that a space $X$ is connected; namely, the condition that every pair of points of $X$ can be joined by a path in $X$ :

Definition. Given points $x$ and $y$ of the space $X$, a path in $X$ from $x$ to $y$ is a continuous map $f:[a, b] \rightarrow X$ of some closed interval in the real line into $X$, such that $f(a)=x$ and $f(b)=y$. A space $X$ is said to be path connected if every pair of points of $X$ can be joined by a path in $X$.

It is easy to see that a path-connected space $X$ is connected. Suppose $X=A \cup B$ is a separation of $X$. Let $f:[a, b] \rightarrow X$ be any path in $X$. Being the continuous image of a connected set, the set $f([a, b])$ is connected, so that it lies entirely in either $A$ or $B$. Therefore, there is no path in $X$ joining a point of $A$ to a point of $B$, contrary to the assumption that $X$ is path connected.

The converse does not hold; a connected space need not be path connected. See Examples 6 and 7 following.

Example 3. Define the unit ball $B^{n}$ in $\mathbb{R}^{n}$ by the equation

$$
B^{n}=\{\mathbf{x} \mid\|\mathbf{x}\| \leq 1\},
$$

where

$$
\|\mathbf{x}\|=\left\|\left(x_{1}, \ldots, x_{n}\right)\right\|=\left(x_{1}^{2}+\cdots+x_{n}^{2}\right)^{1 / 2} .
$$

The unit ball is path connected; given any two points $\mathbf{x}$ and $\mathbf{y}$ of $B^{n}$, the straight-line path $f:[0,1] \rightarrow \mathbb{R}^{n}$ defined by

$$
f(t)=(1-t) \mathbf{x}+t \mathbf{y}
$$

lies in $B^{n}$. For if $\mathbf{x}$ and $\mathbf{y}$ are in $B^{n}$ and $t$ is in $[0,1]$,

$$
\|f(t)\| \leq(1-t)\|\mathbf{x}\|+t\|\mathbf{y}\| \leq 1 .
$$

A similar argument shows that every open ball $B_{d}(\mathbf{x}, \epsilon)$ and every closed ball $\bar{B}_{d}(\mathbf{x}, \epsilon)$ in $\mathbb{R}^{n}$ is path connected.

Example 4. Define punctured euclidean space to be the space $\mathbb{R}^{n}-\{0\}$, where 0 is the origin in $\mathbb{R}^{n}$. If $n>1$, this space is path connected: Given $\mathbf{x}$ and $\mathbf{y}$ different from $\mathbf{0}$, we can join $\mathbf{x}$ and $\mathbf{y}$ by the straight-line path between them if that path does not go through the origin. Otherwise, we can choose a point $\mathbf{z}$ not on the line joining $\mathbf{x}$ and $\mathbf{y}$, and take the broken-line path from $\mathbf{x}$ to $\mathbf{z}$, and then from $\mathbf{z}$ to $\mathbf{y}$.

EXAMPLE 5. Define the unit sphere $S^{n-1}$ in $\mathbb{R}^{n}$ by the equation

$$
S^{n-1}=\{\mathbf{x} \mid\|\mathbf{x}\|=1\} .
$$

If $n>1$, it is path connected. For the map $g: \mathbb{R}^{n}-\{\mathbf{0}\} \rightarrow S^{n-1}$ defined by $g(\mathbf{x})=\mathbf{x} /\|\mathbf{x}\|$ is continuous and surjective; and it is easy to show that the continuous image of a pathconnected space is path connected.

EXAMPLE 6. The ordered square $I_{o}^{2}$ is connected but not path connected.\\
Being a linear continuum, the ordered square is connected. Let $p=0 \times 0$ and $q= 1 \times 1$. We suppose there is a path $f:[a, b] \rightarrow I_{o}^{2}$ joining $p$ and $q$ and derive a contradiction. The image set $f([a, b])$ must contain every point $x \times y$ of $I_{o}^{2}$, by the intermediate value theorem. Therefore, for each $x \in I$, the set

$$
U_{x}=f^{-1}(x \times(0,1))
$$

is a nonempty subset of $[a, b]$; by continuity, it is open in $[a, b]$. See Figure 24.4. Choose, for each $x \in I$, a rational number $q_{x}$ belonging to $U_{x}$. Since the sets $U_{x}$ are disjoint, the map $x \rightarrow q_{x}$ is an injective mapping of $I$ into $\mathbb{Q}$. This contradicts the fact that the interval $I$ is uncountable (which we shall prove later).

Example 7. Let $S$ denote the following subset of the plane.

$$
S=\{x \times \sin (1 / x) \mid 0<x \leq 1\} .
$$

Because $S$ is the image of the connected set ( 0,1 ] under a continuous map, $S$ is connected. Therefore, its closure $\bar{S}$ in $\mathbb{R}^{2}$ is also connected. The set $\bar{S}$ is a classical example in topology

\begin{figure}[h]
\begin{center}
  \includegraphics[width=\textwidth]{2025_11_21_a8ffce9f36674b61ee7eg-083}
\captionsetup{labelformat=empty}
\caption{Figure 24.4}
\end{center}
\end{figure}

called the topologist's sine curve. It is illustrated in Figure 24.5; it equals the union of $S$ and the vertical interval $0 \times[-1,1]$. We show that $\bar{S}$ is not path connected.

Suppose there is a path $f:[a, c] \rightarrow \bar{S}$ beginning at the origin and ending at a point of $S$. The set of those $t$ for which $f(t) \in 0 \times[-1,1]$ is closed, so it has a largest element $b$. Then $f:[b, c] \rightarrow \bar{S}$ is a path that maps $b$ into the vertical interval $0 \times[-1,1]$ and maps the other points of $[b, c]$ to points of $S$.

Replace $[b, c]$ by $[0,1]$ for convenience; let $f(t)=(x(t), y(t))$. Then $x(0)=0$, while $x(t)>0$ and $y(t)=\sin (1 / x(t))$ for $t>0$. We show there is a sequence of points $t_{n} \rightarrow 0$ such that $y\left(t_{n}\right)=(-1)^{n}$. Then the sequence $y\left(t_{n}\right)$ does not converge, contradicting continuity of $f$.

To find $t_{n}$, we proceed as follows: Given $n$, choose $u$ with $0<u<x(1 / n)$ such that $\sin (1 / u)=(-1)^{n}$. Then use the intermediate value theorem to find $t_{n}$ with $0<t_{n}<1 / n$ such that $x\left(t_{n}\right)=u$.

\begin{figure}[h]
\begin{center}
  \includegraphics[width=\textwidth]{2025_11_21_a8ffce9f36674b61ee7eg-083(1)}
\captionsetup{labelformat=empty}
\caption{Figure 24.5}
\end{center}
\end{figure}

\section*{Exercises}
\begin{enumerate}
  \item (a) Show that no two of the spaces $(0,1),(0,1]$, and $[0,1]$ are homeomorphic. [Hint: What happens if you remove a point from each of these spaces?)]\\
(b) Suppose that there exist imbeddings $f: X \rightarrow Y$ and $g: Y \rightarrow X$. Show by means of an example that $X$ and $Y$ need not be homeomorphic.\\
(c) Show $\mathbb{R}^{n}$ and $\mathbb{R}$ are not homeomorphic if $n>1$.
  \item Let $f: S^{1} \rightarrow \mathbb{R}$ be a continuous map. Show there exists a point $x$ of $S^{1}$ such that $f(x)=f(-x)$.
  \item Let $f: X \rightarrow X$ be continuous. Show that if $X=[0,1]$, there is a point $x$ such that $f(x)=x$. The point $x$ is called a fixed point of $f$. What happens if $X$ equals $[0,1)$ or $(0,1)$ ?
  \item Let $X$ be an ordered set in the order topology. Show that if $X$ is connected, then $X$ is a linear continuum.
  \item Consider the following sets in the dictionary order. Which are linear continua?\\
(a) $\mathbb{Z}_{+} \times[0,1)$\\
(b) $[0,1) \times \mathbb{Z}_{+}$\\
(c) $[0,1) \times[0,1]$\\
(d) $[0,1] \times[0,1)$
  \item Show that if $X$ is a well-ordered set, then $X \times[0,1)$ in the dictionary order is a linear continuum.
  \item (a) Let $X$ and $Y$ be ordered sets in the order topology. Show that if $f: X \rightarrow Y$ is order preserving and surjective, then $f$ is a homeomorphism.\\
(b) Let $X=Y=\overline{\mathbb{R}}_{+}$. Given a positive integer $n$, show that the function $f(x)= x^{n}$ is order preserving and surjective. Conclude that its inverse, the $n$th root function, is continuous.\\
(c) Let $X$ be the subspace $(-\infty,-1) \cup[0, \infty)$ of $\mathbb{R}$. Show that the function $f: X \rightarrow \mathbb{R}$ defined by setting $f(x)=x+1$ if $x<-1$, and $f(x)=x$ if $x \geq 0$, is order preserving and surjective. Is $f$ a homeomorphism? Compare with (a).
  \item (a) Is a product of path-connected spaces necessarily path connected?\\
(b) If $A \subset X$ and $A$ is path connected, is $\bar{A}$ necessarily path connected?\\
(c) If $f: X \rightarrow Y$ is continuous and $X$ is path connected, is $f(X)$ necessarily path connected?\\
(d) If $\left\{A_{\alpha}\right\}$ is a collection of path-connected subspaces of $X$ and if $\bigcap A_{\alpha} \neq \varnothing$, is $\bigcup A_{\alpha}$ necessarily path connected?
  \item Assume that $\mathbb{R}$ is uncountable. Show that if $A$ is a countable subset of $\mathbb{R}^{2}$, then $\mathbb{R}^{2}-A$ is path connected. [Hint: How many lines are there passing through a given point of $\mathbb{R}^{2}$ ?]
  \item Show that if $U$ is an open connected subspace of $\mathbb{R}^{2}$, then $U$ is path connected. [Hint: Show that given $x_{0} \in U$, the set of points that can be joined to $x_{0}$ by a path in $U$ is both open and closed in $U$.]
  \item If $A$ is a connected subspace of $X$, does it follow that $\operatorname{Int} A$ and $\operatorname{Bd} A$ are connected? Does the converse hold? Justify your answers.\\
*12. Recall that $S_{\Omega}$ denotes the minimal uncountable well-ordered set. Let $L$ denote the ordered set $S_{\Omega} \times[0,1)$ in the dictionary order, with its smallest element deleted. The set $L$ is a classical example in topology called the long line.
\end{enumerate}

Theorem. The long line is path connected and locally homeomorphic to $\mathbb{R}$, but it cannot be imbedded in $\mathbb{R}$.\\
(a) Let $X$ be an ordered set; let $a<b<c$ be points of $X$. Show that $[a, c)$ has the order type of $[0,1)$ if and only if both $[a, b)$ and $[b, c)$ have the order type of $[0,1)$.\\
(b) Let $X$ be an ordered set. Let $x_{0}<x_{1}<\cdots$ be an increasing sequence of points of $X$; suppose $b=\sup \left\{x_{i}\right\}$. Show that $\left[x_{0}, b\right)$ has the order type of $[0,1)$ if and only if each interval $\left[x_{i}, x_{i+1}\right)$ has the order type of $[0,1)$.\\
(c) Let $a_{0}$ denote the smallest element of $S_{\Omega}$. For each element $a$ of $S_{\Omega}$ different from $a_{0}$, show that the interval $\left[a_{0} \times 0, a \times 0\right)$ of $S_{\Omega} \times[0,1)$ has the order type of $[0,1)$. [Hint: Proceed by transfinite induction. Either $a$ has an immediate predecessor in $S_{\Omega}$, or there is an increasing sequence $a_{i}$ in $S_{\Omega}$ with $a=\sup \left\{a_{i}\right\}$.]\\
(d) Show that $L$ is path connected.\\
(e) Show that every point of $L$ has a neighborhood homeomorphic with an open interval in $\mathbb{R}$.\\
(f) Show that $L$ cannot be imbedded in $\mathbb{R}$, or indeed in $\mathbb{R}^{n}$ for any $n$. [Hint: Any subspace of $\mathbb{R}^{n}$ has a countable basis for its topology.]

\section*{*§25 Components and Local Connectedness ${ }^{\boldsymbol{\dagger}}$}
Given an arbitrary space $X$, there is a natural way to break it up into pieces that are connected (or path connected). We consider that process now.

Definition. Given $X$, define an equivalence relation on $X$ by setting $x \sim y$ if there is a connected subspace of $X$ containing both $x$ and $y$. The equivalence classes are called the components (or the "connected components") of $X$.

Symmetry and reflexivity of the relation are obvious. Transitivity follows by noting that if $A$ is a connected subspace containing $x$ and $y$, and if $B$ is a connected subspace containing $y$ and $z$, then $A \cup B$ is a subspace containing $x$ and $z$ that is connected because $A$ and $B$ have the point $y$ in common.

The components of $X$ can also be described as follows:

Theorem 25.1. The components of $X$ are connected disjoint subspaces of $X$ whose union is $X$, such that each nonempty connected subspace of $X$ intersects only one of them.

Proof. Being equivalence classes, the components of $X$ are disjoint and their union is $X$. Each connected subspace $A$ of $X$ intersects only one of them. For if $A$ intersects the components $C_{1}$ and $C_{2}$ of $X$, say in points $x_{1}$ and $x_{2}$, respectively, then $x_{1} \sim x_{2}$ by definition; this cannot happen unless $C_{1}=C_{2}$.

\footnotetext{${ }^{\dagger}$ This section will be assumed in Part II of the book.
}To show the component $C$ is connected, choose a point $x_{0}$ of $C$. For each point $x$ of $C$, we know that $x_{0} \sim x$, so there is a connected subspace $A_{x}$ containing $x_{0}$ and $x$. By the result just proved, $A_{x} \subset C$. Therefore,

$$
C=\bigcup_{x \in C} A_{x} .
$$

Since the subspaces $\boldsymbol{A}_{\boldsymbol{x}}$ are connected and have the point $x_{0}$ in common, their union is connected.

Definition. We define another equivalence relation on the space $X$ by defining $x \sim y$ if there is a path in $X$ from $x$ to $y$. The equivalence classes are called the path components of $X$.

Let us show this is an equivalence relation. First we note that if there exists a path $f:[a, b] \rightarrow X$ from $x$ to $y$ whose domain is the interval $[a, b]$, then there is also a path $g$ from $x$ to $y$ having the closed interval $[c, d]$ as its domain. (This follows from the fact that any two closed intervals in $\mathbb{R}$ are homeomorphic.) Now the fact that $x \sim x$ for each $x$ in $X$ follows from the existence of the constant path $f:[a, b] \rightarrow X$ defined by the equation $f(t)=x$ for all $t$. Symmetry follows from the fact that if $f:[0,1] \rightarrow X$ is a path from $x$ to $y$, then the "reverse path" $g:[0,1] \rightarrow X$ defined by $g(t)=f(1-t)$ is a path from $y$ to $x$. Finally, transitivity is proved as follows: Let $f:[0,1] \rightarrow X$ be a path from $x$ to $y$, and let $g:[1,2] \rightarrow X$ be a path from $y$ to $z$. We can "paste $f$ and $g$ together" to get a path $h:[0,2] \rightarrow X$ from $x$ to $z$; the path $h$ will be continuous by the "pasting lemma," Theorem 18.3.

One has the following theorem, whose proof is similar to that of the theorem preceding:

Theorem 25.2. The path components of $X$ are path-connected disjoint subspaces of $X$ whose union is $X$, such that each nonempty path-connected subspace of $X$ intersects only one of them.

Note that each component of a space $X$ is closed in $X$, since the closure of a connected subspace of $X$ is connected. If $X$ has only finitely many components, then each component is also open in $X$, since its complement is a finite union of closed sets. But in general the components of $X$ need not be open in $X$.

One can say even less about the path components of $X$, for they need be neither open nor closed in $X$. Consider the following examples:

EXAMPLE 1. If $\mathbb{Q}$ is the subspace of $\mathbb{R}$ consisting of the rational numbers, then each component of $\mathbb{Q}$ consists of a single point. None of the components of $\mathbb{Q}$ are open in $\mathbb{Q}$.

EXAMPLE 2. The "topologist's sine curve" $\bar{S}$ of the preceding section is a space that has a single component (since it is connected) and two path components. One path component is the curve $S$ and the other is the vertical interval $V=0 \times[-1,1]$. Note that $S$ is open in $\bar{S}$ but not closed, while $V$ is closed but not open.

If one forms a space from $\bar{S}$ by deleting all points of $V$ having rational second coordinate, one obtains a space that has only one component but uncountably many path components.

Connectedness is a useful property for a space to possess. But for some purposes, it is more important that the space satisfy a connectedness condition locally. Roughly speaking, local connectedness means that each point has "arbitrarily small" neighborhoods that are connected. More precisely, one has the following definition:

Definition. A space $X$ is said to be locally connected at $\boldsymbol{x}$ if for every neighborhood $U$ of $x$, there is a connected neighborhood $V$ of $x$ contained in $U$. If $X$ is locally connected at each of its points, it is said simply to be locally connected. Similarly, a space $X$ is said to be locally path connected at $x$ if for every neighborhood $U$ of $x$, there is a path-connected neighborhood $V$ of $x$ contained in $U$. If $X$ is locally path connected at each of its points, then it is said to be locally path connected.

\begin{displayquote}
EXAMPLE 3. Each interval and each ray in the real line is both connected and locally connected. The subspace $[-1,0) \cup(0,1]$ of $\mathbb{R}$ is not connected, but it is locally connected. The topologist's sine curve is connected but not locally connected. The rationals $\mathbb{Q}$ are neither connected nor locally connected.
\end{displayquote}

Theorem 25.3. A space $X$ is locally connected if and only if for every open set $U$ of $X$, each component of $U$ is open in $X$.

Proof. Suppose that $X$ is locally connected; let $U$ be an open set in $X$; let $C$ be a component of $U$. If $x$ is a point of $C$, we can choose a connected neighborhood $V$ of $x$ such that $V \subset U$. Since $V$ is connected, it must lie entirely in the component $C$ of $U$. Therefore, $C$ is open in $X$.

Conversely, suppose that components of open sets in $X$ are open. Given a point $x$ of $X$ and a neighborhood $U$ of $x$, let $C$ be the component of $U$ containing $x$. Now $C$ is connected; since it is open in $X$ by hypothesis, $X$ is locally connected at $x$. \(\square\)

A similar proof holds for the following theorem:

Theorem 25.4. A space $X$ is locally path connected if and only if for every open set $U$ of $X$, each path component of $U$ is open in $X$.

The relation between path components and components is given in the following theorem:

Theorem 25.5. If $X$ is a topological space, each path component of $X$ lies in a component of $X$. If $X$ is locally path connected, then the components and the path components of $X$ are the same.

Proof. Let $C$ be a component of $X$; let $x$ be a point of $C$; let $P$ be the path component of $X$ containing $x$. Since $P$ is connected, $P \subset C$. We wish to show that if $X$ is locally path connected, $P=C$. Suppose that $P \subsetneq C$. Let $Q$ denote the union of all the path\\
components of $X$ that are different from $P$ and intersect $C$; each of them necessarily lies in $C$, so that

$$
C=P \cup Q .
$$

Because $X$ is locally path connected, each path component of $X$ is open in $X$. Therefore, $P$ (which is a path component) and $Q$ (which is a union of path components) are open in $X$, so they constitute a separation of $C$. This contradicts the fact that $C$ is connected.

\section*{Exercises}
\begin{enumerate}
  \item What are the components and path components of $\mathbb{R}_{\ell}$ ? What are the continuous maps $f: \mathbb{R} \rightarrow \mathbb{R}_{\ell}$ ?
  \item (a) What are the components and path components of $\mathbb{R}^{\omega}$ (in the product topology)?\\
(b) Consider $\mathbb{R}^{\omega}$ in the uniform topology. Show that $\mathbf{x}$ and $\mathbf{y}$ lie in the same component of $\mathbb{R}^{\omega}$ if and only if the sequence
\end{enumerate}

$$
\mathbf{x}-\mathbf{y}=\left(x_{1}-y_{1}, x_{2}-y_{2}, \ldots\right)
$$

is bounded. [Hint: It suffices to consider the case where $\mathbf{y}=\mathbf{0}$.]\\
(c) Give $\mathbb{R}^{\omega}$ the box topology. Show that $\mathbf{x}$ and $\mathbf{y}$ lie in the same component of $\mathbb{R}^{\omega}$ if and only if the sequence $\mathbf{x}-\mathbf{y}$ is "eventually zero." [Hint: If $\mathbf{x}-\mathbf{y}$ is not eventually zero, show there is homeomorphism $h$ of $\mathbb{R}^{\omega}$ with itself such that $h(\mathbf{x})$ is bounded and $h(\mathbf{y})$ is unbounded.]\\
3. Show that the ordered square is locally connected but not locally path connected. What are the path components of this space?\\
4. Let $X$ be locally path connected. Show that every connected open set in $X$ is path connected.\\
5. Let $X$ denote the rational points of the interval $[0,1] \times 0$ of $\mathbb{R}^{2}$. Let $T$ denote the union of all line segments joining the point $p=0 \times 1$ to points of $X$.\\
(a) Show that $T$ is path connected, but is locally connected only at the point $p$.\\
(b) Find a subset of $\mathbb{R}^{2}$ that is path connected but is locally connected at none of its points.\\
6. A space $X$ is said to be weakly locally connected at $\boldsymbol{x}$ if for every neighborhood $U$ of $x$, there is a connected subspace of $X$ contained in $U$ that contains a neighborhood of $x$. Show that if $X$ is weakly locally connected at each of its points, then $X$ is locally connected. [Hint: Show that components of open sets are open.]\\
7. Consider the "infinite broom" $X$ pictured in Figure 25.1. Show that $X$ is not locally connected at $p$, but is weakly locally connected at $p$. [Hint: Any connected neighborhood of $p$ must contain all the points $a_{i}$.]

\begin{figure}[h]
\begin{center}
  \includegraphics[width=\textwidth]{2025_11_21_a8ffce9f36674b61ee7eg-089}
\captionsetup{labelformat=empty}
\caption{Figure 25.1}
\end{center}
\end{figure}

\begin{enumerate}
  \setcounter{enumi}{7}
  \item Let $p: X \rightarrow Y$ be a quotient map. Show that if $X$ is locally connected, then $Y$ is locally connected. [Hint: If $C$ is a component of the open set $U$ of $Y$, show that $p^{-1}(C)$ is a union of components of $p^{-1}(U)$.]
  \item Let $G$ be a topological group; let $C$ be the component of $G$ containing the identity element $e$. Show that $C$ is a normal subgroup of $G$. [Hint: If $x \in G$, then $x C$ is the component of $G$ containing $x$.]
  \item Let $X$ be a space. Let us define $x \sim y$ if there is no separation $X=A \cup B$ of $X$ into disjoint open sets such that $x \in A$ and $y \in B$.\\
(a) Show this relation is an equivalence relation. The equivalence classes are called the quasicomponents of $X$.\\
(b) Show that each component of $X$ lies in a quasicomponent of $X$, and that the components and quasicomponents of $X$ are the same if $X$ is locally connected.\\
(c) Let $K$ denote the set $\left\{1 / n \mid n \in \mathbb{Z}_{+}\right\}$and let $-K$ denote the set $\{-1 / n \mid n \in \left.\mathbb{Z}_{+}\right\}$. Determine the components, path components, and quasicomponents of the following subspaces of $\mathbb{R}^{2}$ :
\end{enumerate}

$$
\begin{aligned}
& A=(K \times[0,1]) \cup\{0 \times 0\} \cup\{0 \times 1\} \\
& B=A \cup([0,1] \times\{0\}) \\
& C=(K \times[0,1]) \cup(-K \times[-1,0]) \cup([0,1] \times-K) \cup([-1,0] \times K) .
\end{aligned}
$$

\section*{§26 Compact Spaces}
The notion of compactness is not nearly so natural as that of connectedness. From the beginnings of topology, it was clear that the closed interval $[a, b]$ of the real line had a certain property that was crucial for proving such theorems as the maximum value theorem and the uniform continuity theorem. But for a long time, it was not clear how this property should be formulated for an arbitrary topological space. It used to be thought that the crucial property of $[a, b]$ was the fact that every infinite subset of $[a, b]$ has a limit point, and this property was the one dignified with the name of compactness. Later, mathematicians realized that this formulation does not lie at the heart of the matter, but rather that a stronger formulation, in terms of open coverings of the space, is more central. The latter formulation is what we now call compactness.

It is not as natural or intuitive as the former; some familiarity with it is needed before its usefulness becomes apparent.

Definition. A collection $\mathcal{A}$ of subsets of a space $X$ is said to cover $X$, or to be a covering of $X$, if the union of the elements of $\mathcal{A}$ is equal to $X$. It is called an open covering of $X$ if its elements are open subsets of $X$.

Definition. A space $X$ is said to be compact if every open covering $\mathcal{A}$ of $X$ contains a finite subcollection that also covers $X$.

EXAMPLE 1. The real line $\mathbb{R}$ is not compact, for the covering of $\mathbb{R}$ by open intervals

$$
\mathcal{A}=\{(n, n+2) \mid n \in \mathbb{Z}\}
$$

contains no finite subcollection that covers $\mathbb{R}$.

EXAMPLE 2. The following subspace of $\mathbb{R}$ is compact:

$$
X=\{0\} \cup\left\{1 / n \mid n \in \mathbb{Z}_{+}\right\} .
$$

Given an open covering $\mathcal{A}$ of $X$, there is an element $U$ of $\mathcal{A}$ containing 0 . The set $U$ contains all but finitely many of the points $1 / n$; choose, for each point of $X$ not in $U$, an element of $\mathcal{A}$ containing it. The collection consisting of these elements of $\mathcal{A}$, along with the element $U$, is a finite subcollection of $\mathcal{A}$ that covers $X$.

EXAMPLE 3. Any space $X$ containing only finitely many points is necessarily compact, because in this case every open covering of $X$ is finite.

Example 4. The interval $(0,1]$ is not compact; the open covering

$$
\mathcal{A}=\left\{(1 / n, 1] \mid n \in \mathbb{Z}_{+}\right\}
$$

contains no finite subcollection covering ( 0,1 ]. Nor is the interval ( 0,1 ) compact; the same argument applies. On the other hand, the interval $[0,1]$ is compact; you are probably already familiar with this fact from analysis. In any case, we shall prove it shortly.

In general, it takes some effort to decide whether a given space is compact or not. First we shall prove some general theorems that show us how to construct new compact spaces out of existing ones. Then in the next section we shall show certain specific spaces are compact. These spaces include all closed intervals in the real line, and all closed and bounded subsets of $\mathbb{R}^{n}$.

Let us first prove some facts about subspaces. If $Y$ is a subspace of $X$, a collection $\mathcal{A}$ of subsets of $X$ is said to cover $Y$ if the union of its elements contains $Y$.

Lemma 26.1. Let $Y$ be a subspace of $X$. Then $Y$ is compact if and only if every covering of $Y$ by sets open in $X$ contains a finite subcollection covering $Y$.

Proof. Suppose that $Y$ is compact and $\mathcal{A}=\left\{A_{\alpha}\right\}_{\alpha \in J}$ is a covering of $Y$ by sets open in $X$. Then the collection

$$
\left\{A_{\alpha} \cap Y \mid \alpha \in J\right\}
$$

is a covering of $Y$ by sets open in $Y$; hence a finite subcollection

$$
\left\{A_{\alpha_{1}} \cap Y, \ldots, A_{\alpha_{n}} \cap Y\right\}
$$

covers $Y$. Then $\left\{A_{\alpha_{1}}, \ldots, A_{\alpha_{n}}\right\}$ is a subcollection of $\mathcal{A}$ that covers $Y$.\\
Conversely, suppose the given condition holds; we wish to prove $Y$ compact. Let $\mathcal{A}^{\prime}=\left\{A_{\alpha}^{\prime}\right\}$ be a covering of $Y$ by sets open in $Y$. For each $\alpha$, choose a set $A_{\alpha}$ open in $X$ such that

$$
A_{\alpha}^{\prime}=A_{\alpha} \cap Y .
$$

The collection $\mathcal{A}=\left\{A_{\alpha}\right\}$ is a covering of $Y$ by sets open in $X$. By hypothesis, some finite subcollection $\left\{A_{\alpha_{1}}, \ldots, A_{\alpha_{n}}\right\}$ covers $Y$. Then $\left\{A_{\alpha_{1}}^{\prime}, \ldots, A_{\alpha_{n}}^{\prime}\right\}$ is a subcollection of $\mathcal{A}^{\prime}$ that covers $Y$.

Theorem 26.2. Every closed subspace of a compact space is compact.\\
Proof. Let $Y$ be a closed subspace of the compact space $X$. Given a covering $\mathcal{A}$ of $Y$ by sets open in $X$, let us form an open covering $B$ of $X$ by adjoining to $\mathcal{A}$ the single open set $X-Y$, that is,

$$
\mathcal{B}=\mathcal{A} \cup\{X-Y\} .
$$

Some finite subcollection of $\mathfrak{B}$ covers $X$. If this subcollection contains the set $X-Y$, discard $X-Y$; otherwise, leave the subcollection alone. The resulting collection is a finite subcollection of $\mathcal{A}$ that covers $Y$.

Theorem 26.3. Every compact subspace of a Hausdorff space is closed.\\
Proof. Let $Y$ be a compact subspace of the Hausdorff space $X$. We shall prove that $X-Y$ is open, so that $Y$ is closed.

Let $x_{0}$ be a point of $X-Y$. We show there is a neighborhood of $x_{0}$ that is disjoint from $Y$. For each point $y$ of $Y$, let us choose disjoint neighborhoods $U_{y}$ and $V_{y}$ of the points $x_{0}$ and $y$, respectively (using the Hausdorff condition). The collection $\left\{V_{y} \mid y \in\right. Y\}$ is a covering of $Y$ by sets open in $X$; therefore, finitely many of them $V_{y_{1}}, \ldots, V_{y_{n}}$ cover $Y$. The open set

$$
V=V_{y_{1}} \cup \cdots \cup V_{y_{n}}
$$

contains $Y$, and it is disjoint from the open set

$$
U=U_{y_{1}} \cap \cdots \cap U_{y_{n}}
$$

formed by taking the intersection of the corresponding neighborhoods of $x_{0}$. For if $z$ is a point of $V$, then $z \in V_{y_{i}}$ for some $i$, hence $z \notin U_{y_{i}}$ and so $z \notin U$. See Figure 26.1.

Then $U$ is a neighborhood of $x_{0}$ disjoint from $Y$, as desired.

\begin{figure}[h]
\begin{center}
  \includegraphics[width=\textwidth]{2025_11_21_a8ffce9f36674b61ee7eg-092}
\captionsetup{labelformat=empty}
\caption{Figure 26.1}
\end{center}
\end{figure}

The statement we proved in the course of the preceding proof will be useful to us later, so we repeat it here for reference purposes:

Lemma 26.4. If $Y$ is a compact subspace of the Hausdorff space $X$ and $x_{0}$ is not in $Y$, then there exist disjoint open sets $U$ and $V$ of $X$ containing $x_{0}$ and $Y$, respectively.

Example 5. Once we prove that the interval $[a, b]$ in $\mathbb{R}$ is compact, it follows from Theorem 26.2 that any closed subspace of $[a, b]$ is compact. On the other hand, it follows from Theorem 26.3 that the intervals $(a, b]$ and $(a, b)$ in $\mathbb{R}$ cannot be compact (which we knew already) because they are not closed in the Hausdorff space $\mathbb{R}$.

Example 6. One needs the Hausdorff condition in the hypothesis of Theorem 26.3. Consider, for example, the finite complement topology on the real line. The only proper subsets of $\mathbb{R}$ that are closed in this topology are the finite sets. But every subset of $\mathbb{R}$ is compact in this topology, as you can check.

Theorem 26.5. The image of a compact space under a continuous map is compact.\\
Proof. Let $f: X \rightarrow Y$ be continuous; let $X$ be compact. Let $\mathcal{A}$ be a covering of the set $f(X)$ by sets open in $Y$. The collection

$$
\left\{f^{-1}(A) \mid A \in \mathcal{A}\right\}
$$

is a collection of sets covering $X$; these sets are open in $X$ because $f$ is continuous. Hence finitely many of them, say

$$
f^{-1}\left(A_{1}\right), \ldots, f^{-1}\left(A_{n}\right),
$$

cover $X$. Then the sets $A_{1}, \ldots, A_{n}$ cover $f(X)$. \(\square\)

One important use of the preceding theorem is as a tool for verifying that a map is a homeomorphism:

Theorem 26.6. Let $f: X \rightarrow Y$ be a bijective continuous function. If $X$ is compact and $Y$ is Hausdorff, then $f$ is a homeomorphism.

Proof. We shall prove that images of closed sets of $X$ under $f$ are closed in $Y$; this will prove continuity of the map $f^{-1}$. If $A$ is closed in $X$, then $A$ is compact, by Theorem 26.2. Therefore, by the theorem just proved, $f(A)$ is compact. Since $Y$ is Hausdorff, $f(A)$ is closed in $Y$, by Theorem 26.3.

Theorem 26.7. The product of finitely many compact spaces is compact.\\
Proof. We shall prove that the product of two compact spaces is compact; the theorem follows by induction for any finite product.

Step 1. Suppose that we are given spaces $X$ and $Y$, with $Y$ compact. Suppose that $x_{0}$ is a point of $X$, and $N$ is an open set of $X \times Y$ containing the "slice" $x_{0} \times Y$ of $X \times Y$. We prove the following:

There is a neighborhood $W$ of $x_{0}$ in $X$ such that $N$ contains the entire set $W \times Y$.\\
The set $W \times Y$ is often called a tube about $x_{0} \times Y$.\\
First let us cover $x_{0} \times Y$ by basis elements $U \times V$ (for the topology of $X \times Y$ ) lying in $N$. The space $x_{0} \times Y$ is compact, being homeomorphic to $Y$. Therefore, we can cover $x_{0} \times Y$ by finitely many such basis elements

$$
U_{1} \times V_{1}, \ldots, U_{n} \times V_{n}
$$

(We assume that each of the basis elements $U_{i} \times V_{i}$ actually intersects $x_{0} \times Y$, since otherwise that basis element would be superfluous; we could discard it from the finite collection and still have a covering of $x_{0} \times Y$.) Define

$$
W=U_{1} \cap \cdots \cap U_{n}
$$

The set $W$ is open, and it contains $x_{0}$ because each set $U_{i} \times V_{i}$ intersects $x_{0} \times Y$.\\
We assert that the sets $U_{i} \times V_{i}$, which were chosen to cover the slice $x_{0} \times Y$, actually cover the tube $W \times Y$. Let $x \times y$ be a point of $W \times Y$. Consider the point $x_{0} \times y$ of the slice $x_{0} \times Y$ having the same $y$-coordinate as this point. Now $x_{0} \times y$ belongs to $U_{i} \times V_{i}$ for some $i$, so that $y \in V_{i}$. But $x \in U_{j}$ for every $j$ (because $x \in W$ ). Therefore, we have $x \times y \in U_{i} \times V_{i}$, as desired.

Since all the sets $U_{i} \times V_{i}$ lie in $N$, and since they cover $W \times Y$, the tube $W \times Y$ lies in $N$ also. See Figure 26.2.

Step 2. Now we prove the theorem. Let $X$ and $Y$ be compact spaces. Let $\mathcal{A}$ be an open covering of $X \times Y$. Given $x_{0} \in X$, the slice $x_{0} \times Y$ is compact and may therefore be covered by finitely many elements $A_{1}, \ldots, A_{m}$ of $\mathcal{A}$. Their union $N=A_{1} \cup \cdots \cup A_{m}$ is an open set containing $x_{0} \times Y$; by Step 1, the open set $N$ contains

\begin{figure}[h]
\begin{center}
  \includegraphics[width=\textwidth]{2025_11_21_a8ffce9f36674b61ee7eg-094}
\captionsetup{labelformat=empty}
\caption{Figure 26.2}
\end{center}
\end{figure}

a tube $W \times Y$ about $x_{0} \times Y$, where $W$ is open in $X$. Then $W \times Y$ is covered by finitely many elements $A_{1}, \ldots, A_{m}$ of $A$.

Thus, for each $x$ in $X$, we can choose a neighborhood $W_{x}$ of $x$ such that the tube $W_{x} \times Y$ can be covered by finitely many elements of $\mathcal{A}$. The collection of all the neighborhoods $W_{x}$ is an open covering of $X$; therefore by compactness of $X$, there exists a finite subcollection

$$
\left\{W_{1}, \ldots, W_{k}\right\}
$$

covering $X$. The union of the tubes

$$
W_{1} \times Y, \ldots, W_{k} \times Y
$$

is all of $X \times Y$; since each may be covered by finitely many elements of $\mathcal{A}$, so may $X \times Y$ be covered.

The statement proved in Step 1 of the preceding proof will be useful to us later, so we repeat it here as a lemma, for reference purposes:

Lemma 26.8 (The tube lemma). Consider the product space $X \times Y$, where $Y$ is compact. If $N$ is an open set of $X \times Y$ containing the slice $x_{0} \times Y$ of $X \times Y$, then $N$ contains some tube $W \times Y$ about $x_{0} \times Y$, where $W$ is a neighborhood of $x_{0}$ in $X$.

EXAMPLE 7. The tube lemma is certainly not true if $Y$ is not compact. For example, let $Y$ be the $y$-axis in $\mathbb{R}^{2}$, and let

$$
N=\left\{x \times y ;|x|<1 /\left(y^{2}+1\right)\right\}
$$

Then $N$ is an open set containing the set $0 \times \mathbb{R}$, but it contains no tube about $0 \times \mathbb{R}$. It is illustrated in Figure 26.3.

\begin{figure}[h]
\begin{center}
  \includegraphics[width=\textwidth]{2025_11_21_a8ffce9f36674b61ee7eg-095}
\captionsetup{labelformat=empty}
\caption{Figure 26.3}
\end{center}
\end{figure}

There is an obvious question to ask at this point. Is the product of infinitely many compact spaces compact? One would hope that the answer is "yes," and in fact it is. The result is important (and difficult) enough to be called by the name of the man who proved it; it is called the Tychonoff theorem.

In proving the fact that a cartesian product of connected spaces is connected, one proves it first for finite products and derives the general case from that. In proving that cartesian products of compact spaces are compact, however, there is no way to go directly from finite products to infinite ones. The infinite case demands a new approach, and the proof is a difficult one. Because of its difficulty, and also to avoid losing the main thread of our discussion in this chapter, we have decided to postpone it until later. However, you can study it now if you wish; the section in which it is proved (§37) can be studied immediately after this section without causing any disruption in continuity.

There is one final criterion for a space to be compact, a criterion that is formulated in terms of closed sets rather than open sets. It does not look very natural nor very useful at first glance, but it in fact proves to be useful on a number of occasions. First we make a definition.

Definition. A collection $\mathcal{C}$ of subsets of $X$ is said to have the finite intersection property if for every finite subcollection

$$
\left\{C_{1}, \ldots, C_{n}\right\}
$$

of $\mathcal{C}$, the intersection $C_{1} \cap \cdots \cap C_{n}$ is nonempty.

Theorem 26.9. Let $X$ be a topological space. Then $X$ is compact if and only if for every collection $\mathcal{C}$ of closed sets in $X$ having the finite intersection property, the intersection $\bigcap_{C \in \mathcal{C}} C$ of all the elements of $\mathcal{C}$ is nonempty.

Proof. Given a collection $\mathcal{A}$ of subsets of $X$, let

$$
\mathcal{C}=\{X-A \mid A \in \mathcal{A}\}
$$

be the collection of their complements. Then the following statements hold:\\
(1) $\mathcal{A}$ is a collection of open sets if and only if $\mathcal{C}$ is a collection of closed sets.\\
(2) The collection $\mathcal{A}$ covers $X$ if and only if the intersection $\bigcap_{C \in \mathcal{C}} C$ of all the elements of $\mathcal{C}$ is empty.\\
(3) The finite subcollection $\left\{A_{1}, \ldots, A_{n}\right\}$ of $\mathcal{A}$ covers $X$ if and only if the intersection of the corresponding elements $C_{i}=X-A_{i}$ of $\mathcal{C}$ is empty.\\
The first statement is trivial, while the second and third follow from DeMorgan's law:

$$
X-\left(\bigcup_{\alpha \in J} A_{\alpha}\right)=\bigcap_{\alpha \in J}\left(X-A_{\alpha}\right) .
$$

The proof of the theorem now proceeds in two easy steps: taking the contrapositive (of the theorem), and then the complement (of the sets)!

The statement that $X$ is compact is equivalent to saying: "Given any collection $\mathcal{A}$ of open subsets of $X$, if $\mathcal{A}$ covers $X$, then some finite subcollection of $\mathcal{A}$ covers $X$." This statement is equivalent to its contrapositive, which is the following: "Given any collection $\mathcal{A}$ of open sets, if no finite subcollection of $\mathcal{A}$ covers $X$, then $\mathcal{A}$ does not cover $X$." Letting $\mathcal{C}$ be, as earlier, the collection $\{X-A \mid A \in \mathcal{A}\}$ and applying (1)-(3), we see that this statement is in turn equivalent to the following: "Given any collection $\mathcal{C}$ of closed sets, if every finite intersection of elements of $\mathcal{C}$ is nonempty, then the intersection of all the elements of $\mathcal{C}$ is nonempty." This is just the condition of our theorem.

A special case of this theorem occurs when we have a nested sequence $C_{1} \supset C_{2} \supset \cdots \supset C_{n} \supset C_{n+1} \supset \ldots$ of closed sets in a compact space $X$. If each of the sets $C_{n}$ is nonempty, then the collection $\mathcal{C}=\left\{C_{n}\right\}_{n \in \mathbb{Z}_{+}}$automatically has the finite intersection property. Then the intersection

$$
\bigcap_{n \in \mathbb{Z}_{+}} C_{n}
$$

is nonempty.\\
We shall use the closed set criterion for compactness in the next section to prove the uncountability of the set of real numbers, in Chapter 5 when we prove the Tychonoff theorem, and again in Chapter 8 when we prove the Baire category theorem.

\section*{Exercises}
\begin{enumerate}
  \item (a) Let $\mathcal{T}$ and $\mathcal{T}^{\prime}$ be two topologies on the set $X$; suppose that $\mathcal{T}^{\prime} \supset \mathcal{T}$. What does compactness of $X$ under one of these topologies imply about compactness under the other?\\
(b) Show that if $X$ is compact Hausdorff under both $\mathcal{T}$ and $\mathcal{T}^{\prime}$, then either $\mathcal{T}$ and $\mathcal{T}^{\prime}$ are equal or they are not comparable.
  \item (a) Show that in the finite complement topology on $\mathbb{R}$, every subspace is compact.\\
(b) If $\mathbb{R}$ has the topology consisting of all sets $A$ such that $\mathbb{R}-A$ is either countable or all of $\mathbb{R}$, is $[0,1]$ a compact subspace?
  \item Show that a finite union of compact subspaces of $X$ is compact.
  \item Show that every compact subspace of a metric space is bounded in that metric and is closed. Find a metric space in which not every closed bounded subspace is compact.
  \item Let $A$ and $B$ be disjoint compact subspaces of the Hausdorff space $X$. Show that there exist disjoint open sets $U$ and $V$ containing $A$ and $B$, respectively.
  \item Show that if $f: X \rightarrow Y$ is continuous, where $X$ is compact and $Y$ is Hausdorff, then $f$ is a closed map (that is, $f$ carries closed sets to closed sets).
  \item Show that if $Y$ is compact, then the projection $\pi_{1}: X \times Y \rightarrow X$ is a closed map.
  \item Theorem. Let $f: X \rightarrow Y$; let $Y$ be compact Hausdorff. Then $f$ is continuous if and only if the graph of $f$,
\end{enumerate}

$$
G_{f}=\{x \times f(x) \mid x \in X\}
$$

is closed in $X \times Y$. [Hint: If $G_{f}$ is closed and $V$ is a neighborhood of $f\left(x_{0}\right)$, then the intersection of $G_{f}$ and $X \times(Y-V)$ is closed. Apply Exercise 7.]\\
9. Generalize the tube lemma as follows:

Theorem. Let $A$ and $B$ be subspaces of $X$ and $Y$, respectively; let $N$ be an open set in $X \times Y$ containing $A \times B$. If $A$ and $B$ are compact, then there exist open sets $U$ and $V$ in $X$ and $Y$, respectively, such that

$$
A \times B \subset U \times V \subset N
$$

\begin{enumerate}
  \setcounter{enumi}{9}
  \item (a) Prove the following partial converse to the uniform limit theorem:
\end{enumerate}

Theorem. Let $f_{n}: X \rightarrow \mathbb{R}$ be a sequence of continuous functions, with $f_{n}(x) \rightarrow f(x)$ for each $x \in X$. If $f$ is continuous, and if the sequence $f_{n}$ is monotone increasing, and if $X$ is compact, then the convergence is uniform. [We say that $f_{n}$ is monotone increasing if $f_{n}(x) \leq f_{n+1}(x)$ for all $n$ and $x$.]\\
(b) Give examples to show that this theorem fails if you delete the requirement that $X$ be compact, or if you delete the requirement that the sequence be monotone. [Hint: See the exercises of §21.]\\
11. Theorem. Let $X$ be a compact Hausdorff space. Let $A$ be a collection of closed connected subsets of $X$ that is simply ordered by proper inclusion. Then

$$
Y=\bigcap_{A \in \mathcal{A}} A
$$

is connected. [Hint: If $C \cup D$ is a separation of $Y$, choose disjoint open sets $U$ and $V$ of $X$ containing $C$ and $D$, respectively, and show that

$$
\bigcap_{A \in \mathscr{A}}(A-(U \cup V))
$$

is not empty.]\\
12. Let $p: X \rightarrow Y$ be a closed continuous surjective map such that $p^{-1}(\{y\})$ is compact, for each $y \in Y$. (Such a map is called a perfect map.) Show that if $Y$ is compact, then $X$ is compact. [Hint: If $U$ is an open set containing $p^{-1}(\{y\})$, there is a neighborhood $W$ of $y$ such that $p^{-1}(W)$ is contained in $U$.]\\
13. Let $G$ be a topological group.\\
(a) Let $A$ and $B$ be subspaces of $G$. If $A$ is closed and $B$ is compact, show $A \cdot B$ is closed. [Hint: If $c$ is not in $A \cdot B$, find a neighborhood $W$ of $c$ such that $W \cdot B^{-1}$ is disjoint from $A$.]\\
(b) Let $H$ be a subgroup of $G$; let $p: G \rightarrow G / H$ be the quotient map. If $H$ is compact, show that $p$ is a closed map.\\
(c) Let $H$ be a compact subgroup of $G$. Show that if $G / H$ is compact, then $G$ is compact.

\section*{§27 Compact Subspaces of the Real Line}
The theorems of the preceding section enable us to construct new compact spaces from existing ones, but in order to get very far we have to find some compact spaces to start with. The natural place to begin is the real line; we shall prove that every closed interval in $\mathbb{R}$ is compact. Applications include the extreme value theorem and the uniform continuity theorem of calculus, suitably generalized. We also give a characterization of all compact subspaces of $\mathbb{R}^{n}$, and a proof of the uncountability of the set of real numbers.

It turns out that in order to prove every closed interval in $\mathbb{R}$ is compact, we need only one of the order properties of the real line-the least upper bound property. We shall prove the theorem using only this hypothesis; then it will apply not only to the real line, but to well-ordered sets and other ordered sets as well.

Theorem 27.1. Let $X$ be a simply ordered set having the least upper bound property. In the order topology, each closed interval in $X$ is compact.

Proof. Step 1. Given $a<b$, let $\mathcal{A}$ be a covering of $[a, b]$ by sets open in $[a, b]$ in the subspace topology (which is the same as the order topology). We wish to prove the existence of a finite subcollection of $\mathcal{A}$ covering $[a, b]$. First we prove the following: If $x$ is a point of $[a, b]$ different from $b$, then there is a point $y>x$ of $[a, b]$ such that the interval $[x, y]$ can be covered by at most two elements of $\mathcal{A}$.

If $x$ has an immediate successor in $X$, let $y$ be this immediate successor. Then $[x, y]$ consists of the two points $x$ and $y$, so that it can be covered by at most two elements of $\mathcal{A}$. If $x$ has no immediate successor in $X$, choose an element $A$ of $\mathcal{A}$ containing $x$. Because $x \neq b$ and $A$ is open, $A$ contains an interval of the form $[x, c)$, for some $c$ in $[a, b]$. Choose a point $y$ in $(x, c)$; then the interval $[x, y]$ is covered by the single element $A$ of $\mathcal{A}$.

Step 2. Let $C$ be the set of all points $y>a$ of $[a, b]$ such that the interval $[a, y]$ can be covered by finitely many elements of $\mathcal{A}$. Applying Step 1 to the case $x=a$, we see that there exists at least one such $y$, so $C$ is not empty. Let $c$ be the least upper bound of the set $C$; then $a<c \leq b$.

Step 3. We show that $c$ belongs to $C$; that is, we show that the interval $[a, c]$ can be covered by finitely many elements of $\mathcal{A}$. Choose an element $A$ of $\mathcal{A}$ containing $c$; since $A$ is open, it contains an interval of the form $(d, c]$ for some $d$ in $[a, b]$. If $c$ is not in $C$, there must be a point $z$ of $C$ lying in the interval ( $d, c$ ), because otherwise $d$ would be a smaller upper bound on $C$ than $c$. See Figure 27.1. Since $z$ is in $C$, the interval $[a, z]$ can be covered by finitely many, say $n$, elements of $\mathcal{A}$. Now $[z, c]$ lies in the single element $A$ of $\mathcal{A}$, hence $[a, c]=[a, z] \cup[z, c]$ can be covered by $n+1$ elements of $\mathcal{A}$. Thus $c$ is in $C$, contrary to assumption.

\begin{figure}[h]
\begin{center}
  \includegraphics[width=\textwidth]{2025_11_21_a8ffce9f36674b61ee7eg-099(1)}
\captionsetup{labelformat=empty}
\caption{Figure 27.1}
\end{center}
\end{figure}

\begin{figure}[h]
\begin{center}
  \includegraphics[width=\textwidth]{2025_11_21_a8ffce9f36674b61ee7eg-099}
\captionsetup{labelformat=empty}
\caption{Figure 27.2}
\end{center}
\end{figure}

Step 4. Finally, we show that $c=b$, and our theorem is proved. Suppose that $c<b$. Applying Step 1 to the case $x=c$, we conclude that there exists a point $y>c$ of $[a, b]$ such that the interval $[c, y]$ can be covered by finitely many elements of $\mathcal{A}$. See Figure 27.2. We proved in Step 3 that $c$ is in $C$, so $[a, c]$ can be covered by finitely many elements of $\mathcal{A}$. Therefore, the interval

$$
[a, y]=[a, c] \cup[c, y]
$$

can also be covered by finitely many elements of $\mathcal{A}$. This means that $y$ is in $C$, contradicting the fact that $c$ is an upper bound on $C$. \(\square\)

Corollary 27.2. Every closed interval in $\mathbb{R}$ is compact.\\
Now we characterize the compact subspaces of $\mathbb{R}^{n}$ :\\
Theorem 27.3. A subspace $A$ of $\mathbb{R}^{n}$ is compact if and only if it is closed and is bounded in the euclidean metric $d$ or the square metric $\rho$.

Proof. It will suffice to consider only the metric $\rho$; the inequalities

$$
\rho(x, y) \leq d(x, y) \leq \sqrt{n} \rho(x, y)
$$

imply that $A$ is bounded under $d$ if and only if it is bounded under $\rho$.\\
Suppose that $A$ is compact. Then, by Theorem 26.3, it is closed. Consider the collection of open sets

$$
\left\{B_{\rho}(\mathbf{0}, m) \mid m \in \mathbb{Z}_{+}\right\},
$$

whose union is all of $\mathbb{R}^{n}$. Some finite subcollection covers $A$. It follows that $A \subset \boldsymbol{B}_{\rho}(\mathbf{0}, \boldsymbol{M})$ for some $\boldsymbol{M}$. Therefore, for any two points $x$ and $y$ of $A$, we have $\rho(x, y) \leq 2 M$. Thus $A$ is bounded under $\rho$.

Conversely, suppose that $A$ is closed and bounded under $\rho$; suppose that $\rho(x, y) \leq N$ for every pair $x, y$ of points of $A$. Choose a point $x_{0}$ of $A$, and let $\rho\left(x_{0}, \mathbf{0}\right)=b$. The triangle inequality implies that $\rho(x, \mathbf{0}) \leq N+b$ for every $x$ in $A$. If $P=N+b$, then $A$ is a subset of the cube $[-P, P]^{n}$, which is compact. Being closed, $A$ is also compact.

Students often remember this theorem as stating that the collection of compact sets in a metric space equals the collection of closed and bounded sets. This statement is clearly ridiculous as it stands, because the question as to which sets are bounded depends for its answer on the metric, whereas which sets are compact depends only on the topology of the space.

Example 1. The unit sphere $S^{n-1}$ and the closed unit ball $B^{n}$ in $\mathbb{R}^{n}$ are compact because they are closed and bounded. The set

$$
A=\{x \times(1 / x) \mid 0<x \leq 1\}
$$

is closed in $\mathbb{R}^{2}$, but it is not compact because it is not bounded. The set

$$
S=\{x \times(\sin (1 / x)) \mid 0<x \leq 1\}
$$

is bounded in $\mathbb{R}^{2}$, but it is not compact because it is not closed.\\
Now we prove the extreme value theorem of calculus, in suitably generalized form.

Theorem 27.4 (Extreme value theorem). Let $f: X \rightarrow Y$ be continuous, where $Y$ is an ordered set in the order topology. If $X$ is compact, then there exist points $c$ and $d$ in $X$ such that $f(c) \leq f(x) \leq f(d)$ for every $x \in X$.

The extreme value theorem of calculus is the special case of this theorem that occurs when we take $X$ to be a closed interval in $\mathbb{R}$ and $Y$ to be $\mathbb{R}$.\\
Proof. Since $f$ is continuous and $X$ is compact, the set $A=f(X)$ is compact. We show that $A$ has a largest element $M$ and a smallest element $m$. Then since $m$ and $M$ belong to $A$, we must have $m=f(c)$ and $M=f(d)$ for some points $c$ and $d$ of $X$.

If $A$ has no largest element, then the collection

$$
\{(-\infty, a) \mid a \in A\}
$$

forms an open covering of $A$. Since $A$ is compact, some finite subcollection

$$
\left\{\left(-\infty, a_{1}\right), \ldots,\left(-\infty, a_{n}\right)\right\}
$$

covers $A$. If $a_{i}$ is the largest of the elements $a_{1}, \ldots a_{n}$, then $a_{i}$ belongs to none of these sets, contrary to the fact that they cover $A$.

A similar argument shows that $A$ has a smallest element.

Now we prove the uniform continuity theorem of calculus. In the process, we are led to introduce a new notion that will prove to be surprisingly useful, that of a Lebesgue number for an open covering of a metric space. First, a preliminary notion:

Definition. Let $(X, d)$ be a metric space; let $A$ be a nonempty subset of $X$. For each $x \in X$, we define the distance from $x$ to $A$ by the equation

$$
d(x, A)=\inf \{d(x, a) \mid a \in A\}
$$

It is easy to show that for fixed $A$, the function $d(x, A)$ is a continuous function of $x$ : Given $x, y \in X$, one has the inequalities

$$
d(x, A) \leq d(x, a) \leq d(x, y)+d(y, a)
$$

for each $a \in A$. It follows that

$$
d(x, A)-d(x, y) \leq \inf d(y, a)=d(y, A)
$$

so that

$$
d(x, A)-d(y, A) \leq d(x, y)
$$

The same inequality holds with $x$ and $y$ interchanged; continuity of the function $d(x, A)$ follows.

Now we introduce the notion of Lebesgue number. Recall that the diameter of a bounded subset $A$ of a metric space ( $X, d$ ) is the number

$$
\sup \left\{d\left(a_{1}, a_{2}\right) \mid a_{1}, a_{2} \in A\right\}
$$

Lemma 27.5 (The Lebesgue number lemma). Let $\mathcal{A}$ be an open covering of the metric space ( $X, d$ ). If $X$ is compact, there is a $\delta>0$ such that for each subset of $X$ having diameter less than $\delta$, there exists an element of $A$ containing it.

The number $\delta$ is called a Lebesgue number for the covering $\mathcal{A}$.\\
Proof. Let $\mathscr{A}$ be an open covering of $X$. If $X$ itself is an element of $\mathscr{A}$, then any positive number is a Lebesgue number for $\mathcal{A}$. So assume $X$ is not an element of $\mathcal{A}$.

Choose a finite subcollection $\left\{A_{1}, \ldots, A_{n}\right\}$ of $\mathcal{A}$ that covers $X$. For each $i$, set $C_{i}=X-A_{i}$, and define $f: X \rightarrow \mathbb{R}$ by letting $f(x)$ be the average of the numbers $d\left(x, C_{i}\right)$. That is,

$$
f(x)=\frac{1}{n} \sum_{i=1}^{n} d\left(x, C_{i}\right)
$$

We show that $f(x)>0$ for all $x$. Given $x \in X$, choose $i$ so that $x \in A_{i}$. Then choose $\epsilon$ so the $\epsilon$-neighborhood of $x$ lies in $A_{i}$. Then $d\left(x, C_{i}\right) \geq \epsilon$, so that $f(x) \geq \epsilon / n$.

Since $f$ is continuous, it has a minimum value $\delta$; we show that $\delta$ is our required Lebesgue number. Let $B$ be a subset of $X$ of diameter less than $\delta$. Choose a point $x_{0}$ of $B$; then $B$ lies in the $\delta$-neighborhood of $x_{0}$. Now

$$
\delta \leq f\left(x_{0}\right) \leq d\left(x_{0}, C_{m}\right)
$$

where $d\left(x_{0}, C_{m}\right)$ is the largest of the numbers $d\left(x_{0}, C_{i}\right)$. Then the $\delta$-neighborhood of $x_{0}$ is contained in the element $A_{m}=X-C_{m}$ of the covering $\mathcal{A}$.

Definition. A function $f$ from the metric space ( $X, d_{X}$ ) to the metric space ( $Y, d_{Y}$ ) is said to be uniformly continuous if given $\epsilon>0$, there is a $\delta>0$ such that for every pair of points $x_{0}, x_{1}$ of $X$,

$$
d_{X}\left(x_{0}, x_{1}\right)<\delta \Longrightarrow d_{Y}\left(f\left(x_{0}\right), f\left(x_{1}\right)\right)<\epsilon .
$$

Theorem 27.6 (Uniform continuity theorem). Let $f: X \rightarrow Y$ be a continuous map of the compact metric space ( $X, d_{X}$ ) to the metric space ( $Y, d_{Y}$ ). Then $f$ is uniformly continuous.

Proof. Given $\epsilon>0$, take the open covering of $Y$ by balls $B(y, \epsilon / 2)$ of radius $\epsilon / 2$. Let $\mathcal{A}$ be the open covering of $X$ by the inverse images of these balls under $f$. Choose $\delta$ to be a Lebesgue number for the covering $\mathcal{A}$. Then if $x_{1}$ and $x_{2}$ are two points of $X$ such that $d_{X}\left(x_{1}, x_{2}\right)<\delta$, the two-point set $\left\{x_{1}, x_{2}\right\}$ has diameter less than $\delta$, so that its image $\left\{f\left(x_{1}\right), f\left(x_{2}\right)\right\}$ lies in some ball $B(y, \epsilon / 2)$. Then $d_{Y}\left(f\left(x_{1}\right), f\left(x_{2}\right)\right)<\epsilon$, as desired.

Finally, we prove that the real numbers are uncountable. The interesting thing about this proof is that it involves no algebra at all-no decimal or binary expansions of real numbers or the like-just the order properties of $\mathbb{R}$.

Definition. If $X$ is a space, a point $x$ of $X$ is said to be an isolated point of $X$ if the one-point set $\{x\}$ is open in $X$.

Theorem 27.7. Let $X$ be a nonempty compact Hausdorff space. If $X$ has no isolated points, then $X$ is uncountable.

Proof. Step 1. We show first that given any nonempty open set $U$ of $X$ and any point $x$ of $X$, there exists a nonempty open set $V$ contained in $U$ such that $x \notin \bar{V}$.

Choose a point $y$ of $U$ different from $x$; this is possible if $x$ is in $U$ because $x$ is not an isolated point of $X$ and it is possible if $x$ is not in $U$ simply because $U$ is nonempty. Now choose disjoint open sets $W_{1}$ and $W_{2}$ about $x$ and $y$, respectively. Then the set $V=W_{2} \cap U$ is the desired open set; it is contained in $U$, it is nonempty because it contains $y$, and its closure does not contain $x$. See Figure 27.3.

Step 2. We show that given $f: \mathbb{Z}_{+} \rightarrow X$, the function $f$ is not surjective. It follows that $X$ is uncountable.

\begin{figure}[h]
\begin{center}
  \includegraphics[width=\textwidth]{2025_11_21_a8ffce9f36674b61ee7eg-103}
\captionsetup{labelformat=empty}
\caption{Figure 27.3}
\end{center}
\end{figure}

Let $x_{n}=f(n)$. Apply Step 1 to the nonempty open set $U=X$ to choose a nonempty open set $V_{1} \subset X$ such that $\vec{V}_{1}$ does not contain $x_{1}$. In general, given $V_{n-1}$ open and nonempty, choose $V_{n}$ to be a nonempty open set such that $V_{n} \subset V_{n-1}$ and $\bar{V}_{n}$ does not contain $x_{n}$. Consider the nested sequence

$$
\bar{V}_{1} \supset \bar{V}_{2} \supset \cdots
$$

of nonempty closed sets of $X$. Because $X$ is compact, there is a point $x \in \bigcap \bar{V}_{n}$, by Theorem 26.9. Now $x$ cannot equal $x_{n}$ for any $n$, since $x$ belongs to $\bar{V}_{n}$ and $x_{n}$ does not. \(\square\)

Corollary 27.8. Every closed interval in $\mathbb{R}$ is uncountable.

\section*{Exercises}
\begin{enumerate}
  \item Prove that if $X$ is an ordered set in which every closed interval is compact, then $X$ has the least upper bound property.
  \item Let $X$ be a metric space with metric $d$; let $A \subset X$ be nonempty.\\
(a) Show that $d(x, A)=0$ if and only if $x \in \bar{A}$.\\
(b) Show that if $A$ is compact, $d(x, A)=d(x, a)$ for some $a \in A$.\\
(c) Define the $\epsilon$-neighborhood of $A$ in $X$ to be the set
\end{enumerate}

$$
U(A, \epsilon)=\{x \mid d(x, A)<\epsilon\} .
$$

Show that $U(A, \epsilon)$ equals the union of the open balls $B_{d}(a . \epsilon)$ for $a \in A$.\\
(d) Assume that $A$ is compact; let $U$ be an open set containing $A$. Show that some $\epsilon$-neighborhood of $A$ is contained in $U$.\\
(e) Show the result in (d) need not hold if $A$ is closed but not compact.\\
3. Recall that $\mathbb{R}_{K}$ denotes $\mathbb{R}$ in the $K$-topology.\\
(a) Show that $[0,1]$ is not compact as a subspace of $\mathbb{R}_{K}$.\\
(b) Show that $\mathbb{R}_{K}$ is connected. [Hint: $(-\infty, 0)$ and ( $0, \infty$ ) inherit their usual topologies as subspaces of $\mathbb{R}_{K}$.]\\
(c) Show that $\mathbb{R}_{K}$ is not path connected.\\
4. Show that a connected metric space having more than one point is uncountable.\\
5. Let $X$ be a compact Hausdorff space; let $\left\{A_{n}\right\}$ be a countable collection of closed sets of $X$. Show that if each set $A_{n}$ has empty interior in $X$, then the union $\cup A_{n}$ has empty interior in $X$. [Hint: Imitate the proof of Theorem 27.7.]

This is a special case of the Baire category theorem, which we shall study in Chapter 8.\\
6. Let $A_{0}$ be the closed interval [ 0,1 ] in $\mathbb{R}$. Let $A_{1}$ be the set obtained from $A_{0}$ by deleting its "middle third" ( $\frac{1}{3}, \frac{2}{3}$ ). Let $A_{2}$ be the set obtained from $A_{1}$ by deleting its "middle thirds" $\left(\frac{1}{9}, \frac{2}{9}\right)$ and $\left(\frac{7}{9}, \frac{8}{9}\right)$. In general, define $A_{n}$ by the equation

$$
A_{n}=A_{n-1}-\bigcup_{k=0}^{\infty}\left(\frac{1+3 k}{3^{n}}, \frac{2+3 k}{3^{n}}\right) .
$$

The intersection

$$
C=\bigcap_{n \in \mathbb{Z}_{+}} A_{n}
$$

is called the Cantor set; it is a subspace of [ 0,1 ].\\
(a) Show that $C$ is totally disconnected.\\
(b) Show that $C$ is compact.\\
(c) Show that each set $A_{n}$ is a union of finitely many disjoint closed intervals of length $1 / 3^{n}$; and show that the end points of these intervals lie in $C$.\\
(d) Show that $C$ has no isolated points.\\
(e) Conclude that $C$ is uncountable.

\section*{§28 Limit Point Compactness}
As indicated when we first mentioned compact sets, there are other formulations of the notion of compactness that are frequently useful. In this section we introduce one of them. Weaker in general than compactness, it coincides with compactness for metrizable spaces.

Definition. A space $X$ is said to be limit point compact if every infinite subset of $X$ has a limit point.

In some ways this property is more natural and intuitive than that of compactness. In the early days of topology, it was given the name "compactness," while the open covering formulation was called "bicompactness." Later, the word "compact" was shifted to apply to the open covering definition, leaving this one to search for a new\\
name. It still has not found a name on which everyone agrees. On historical grounds, some call it "Fréchet compactness"; others call it the "Bolzano-Weierstrass property." We have invented the term "limit point compactness." It seems as good a term as any; at least it describes what the property is about.

Theorem 28.1. Compactness implies limit point compactness, but not conversely.\\
Proof. Let $X$ be a compact space. Given a subset $A$ of $X$, we wish to prove that if $A$ is infinite, then $A$ has a limit point. We prove the contrapositive-if $A$ has no limit point, then $A$ must be finite.

So suppose $A$ has no limit point. Then $A$ contains all its limit points, so that $A$ is closed. Furthermore, for each $a \in A$ we can choose a neighborhood $U_{a}$ of $a$ such that $U_{a}$ intersects $A$ in the point $a$ alone. The space $X$ is covered by the open set $X-A$ and the open sets $U_{a}$; being compact, it can be covered by finitely many of these sets. Since $X-A$ does not intersect $A$, and each set $U_{a}$ contains only one point of $A$, the set $A$ must be finite.

EXAMPLE 1. Let $Y$ consist of two points; give $Y$ the topology consisting of $Y$ and the empty set. Then the space $X=\mathbb{Z}_{+} \times Y$ is limit point compact, for every nonempty subset of $X$ has a limit point. It is not compact, for the covering of $X$ by the open sets $U_{n}=\{n\} \times Y$ has no finite subcollection covering $X$.

Example 2. Here is a less trivial example. Consider the minimal uncountable wellordered set $S_{\Omega}$, in the order topology. The space $S_{\Omega}$ is not compact, since it has no largest element. However, it is limit point compact: Let $A$ be an infinite subset of $S_{\Omega}$. Choose a subset $B$ of $A$ that is countably infinite. Being countable, the set $B$ has an upper bound $b$ in $S_{\Omega}$; then $B$ is a subset of the interval $\left[a_{0}, b\right]$ of $S_{\Omega}$, where $a_{0}$ is the smallest element of $S_{\Omega}$. Since $S_{\Omega}$ has the least upper bound property, the interval $\left[a_{0}, b\right]$ is compact. By the preceding theorem, $B$ has a limit point $x$ in $\left[a_{0}, b\right]$. The point $x$ is also a limit point of $A$. Thus $S_{\Omega}$ is limit point compact.\\
We now show these two versions of compactness coincide for metrizable spaces; for this purpose, we introduce yet another version of compactness called sequential compactness. This result will be used in Chapter 7.

Definition. Let $X$ be a topological space. If $\left(x_{n}\right)$ is a sequence of points of $X$, and if

$$
n_{1}<n_{2}<\cdots<n_{i}<\cdots
$$

is an increasing sequence of positive integers, then the sequence ( $y_{i}$ ) defined by setting $y_{i}=x_{n_{i}}$ is called a subsequence of the sequence $\left(x_{n}\right)$. The space $X$ is said to be sequentially compact if every sequence of points of $X$ has a convergent subsequence.\\
*Theorem 28.2. Let $X$ be a metrizable space. Then the following are equivalent:\\
(1) $X$ is compact.\\
(2) $X$ is limit point compact.\\
(3) $X$ is sequentially compact.

Proof. We have already proved that (1) ⇒ (2). To show that (2) $\Rightarrow$ (3), assume that $X$ is limit point compact. Given a sequence $\left(x_{n}\right)$ of points of $X$, consider the set $A=\left\{x_{n} \mid n \in \mathbb{Z}_{+}\right\}$. If the set $A$ is finite, then there is a point $x$ such that $x=x_{n}$ for infinitely many values of $n$. In this case, the sequence $\left(x_{n}\right)$ has a subsequence that is constant, and therefore converges trivially. On the other hand, if $A$ is infinite, then $A$ has a limit point $x$. We define a subsequence of $\left(x_{n}\right)$ converging to $x$ as follows: First choose $n_{1}$ so that

$$
x_{n_{1}} \in B(x, 1)
$$

Then suppose that the positive integer $n_{i-1}$ is given. Because the ball $B(x, 1 / i)$ intersects $A$ in infinitely many points, we can choose an index $n_{i}>n_{i-1}$ such that

$$
x_{n_{i}} \in B(x, 1 / i)
$$

Then the subsequence $x_{n_{1}}, x_{n_{2}}, \ldots$ converges to $x$.\\
Finally, we show that (3) $\Rightarrow(1)$. This is the hardest part of the proof.\\
First, we show that if $X$ is sequentially compact, then the Lebesgue number lemma holds for $X$. (This would follow from compactness, but compactness is what we are trying to prove!) Let $\mathcal{A}$ be an open covering of $X$. We assume that there is no $\delta>0$ such that each set of diameter less than $\delta$ has an element of $\mathcal{A}$ containing it, and derive a contradiction.

Our assumption implies in particular that for each positive integer $n$, there exists a set of diameter less than $1 / n$ that is not contained in any element of $\mathcal{A}$; let $C_{n}$ be such a set. Choose a point $x_{n} \in C_{n}$, for each $n$. By hypothesis, some subsequence ( $x_{n_{i}}$ ) of the sequence ( $x_{n}$ ) converges, say to the point $a$. Now $a$ belongs to some element $A$ of the collection $\mathcal{A}$; because $A$ is open, we may choose an $\epsilon>0$ such that $B(a, \epsilon) \subset A$. If $i$ is large enough that $1 / n_{i}<\epsilon / 2$, then the set $C_{n_{i}}$ lies in the $\epsilon / 2$-neighborhood of $x_{n_{i}}$; if $i$ is also chosen large enough that $d\left(x_{n_{i}}, a\right)<\epsilon / 2$, then $C_{n_{i}}$ lies in the $\epsilon$-neighborhood of $a$. But this means that $C_{n_{i}} \subset A$, contrary to hypothesis.

Second, we show that if $X$ is sequentially compact, then given $\epsilon>0$, there exists a finite covering of $X$ by open $\epsilon$-balls. Once again, we proceed by contradiction. Assume that there exists an $\epsilon>0$ such that $X$ cannot be covered by finitely many $\epsilon$-balls. Construct a sequence of points $x_{n}$ of $X$ as follows: First, choose $x_{1}$ to be any point of $X$. Noting that the ball $B\left(x_{1}, \epsilon\right)$ is not all of $X$ (otherwise $X$ could be covered by a single $\epsilon$-ball), choose $x_{2}$ to be a point of $X$ not in $B\left(x_{1}, \epsilon\right)$. In general, given $x_{1}, \ldots, x_{n}$, choose $x_{n+1}$ to be a point not in the union

$$
B\left(x_{1}, \epsilon\right) \cup \cdots \cup B\left(x_{n}, \epsilon\right)
$$

using the fact that these balls do not cover $X$. Note that by construction $d\left(x_{n+1}, x_{i}\right) \geq \epsilon$ for $i=1, \ldots, n$. Therefore, the sequence ( $x_{n}$ ) can have no convergent subsequence; in fact, any ball of radius $\epsilon / 2$ can contain $x_{n}$ for at most one value of $n$.

Finally, we show that if $X$ is sequentially compact, then $X$ is compact. Let $\mathcal{A}$ be an open covering of $X$. Because $X$ is sequentially compact, the open covering $\mathcal{A}$ has a Lebesgue number $\delta$. Let $\epsilon=\delta / 3$; use sequential compactness of $X$ to find a finite\\
covering of $X$ by open $\epsilon$-balls. Each of these balls has diameter at most $2 \delta / 3$, so it lies in an element of $\mathcal{A}$. Choosing one such element of $\mathcal{A}$ for each of these $\epsilon$-balls, we obtain a finite subcollection of $\mathcal{A}$ that covers $X$. \(\square\)

EXAMPLE 3. Recall that $\bar{S}_{\Omega}$ denotes the minimal uncountable well-ordered set $S_{\Omega}$ with the point $\Omega$ adjoined. (In the order topology, $\Omega$ is a limit point of $S_{\Omega}$, which is why we introduced the notation $\bar{S}_{\Omega}$ for $S_{\Omega} \cup\{\Omega\}$, back in §10.) It is easy to see that the space $\bar{S}_{\Omega}$ is not metrizable, for it does not satisfy the sequence lemma: The point $\Omega$ is a limit point of $S_{\Omega}$; but it is not the limit of a sequence of points of $S_{\Omega}$, for any sequence of points of $S_{\Omega}$ has an upper bound in $S_{\Omega}$. The space $S_{\Omega}$, on the other hand, does satisfy the sequence lemma, as you can readily check. Nevertheless, $S_{\Omega}$ is not metrizable, for it is limit point compact but not compact.

\section*{Exercises}
\begin{enumerate}
  \item Give $[0,1]^{\omega}$ the uniform topology. Find an infinite subset of this space that has no limit point.
  \item Show that $[0,1]$ is not limit point compact as a subspace of $\mathbb{R}_{\ell}$.
  \item Let $X$ be limit point compact.\\
(a) If $f: X \rightarrow Y$ is continuous, does it follow that $f(X)$ is limit point compact?\\
(b) If $A$ is a closed subset of $X$, does it follow that $A$ is limit point compact?\\
(c) If $X$ is a subspace of the Hausdorff space $Z$, does it follow that $X$ is closed in $Z$ ?\\[0pt]
We comment that it is not in general true that the product of two limit point compact spaces is limit point compact, even if the Hausdorff condition is assumed. But the examples are fairly sophisticated. See [S-S], Example 112.
  \item A space $X$ is said to be countably compact if every countable open covering of $X$ contains a finite subcollection that covers $X$. Show that for a $T_{1}$ space $X$, countable compactness is equivalent to limit point compactness. [Hint: If no finite subcollection of $U_{n}$ covers $X$, choose $x_{n} \notin U_{1} \cup \cdots \cup U_{n}$, for each $n$.]
  \item Show that $X$ is countably compact if and only if every nested sequence $C_{1} \supset C_{2} \supset \cdots$ of closed nonempty sets of $X$ has a nonempty intersection.
  \item Let $(X, d)$ be a metric space. If $f: X \rightarrow X$ satisfies the condition
\end{enumerate}

$$
d(f(x), f(y))=d(x, y)
$$

for all $x, y \in X$, then $f$ is called an isometry of $X$. Show that if $f$ is an isometry and $X$ is compact, then $f$ is bijective and hence a homeomorphism. [Hint: If $a \notin f(X)$, choose $\epsilon$ so that the $\epsilon$-neighborhood of $a$ is disjoint from $f(X)$. Set $x_{1}=a$, and $x_{n+1}=f\left(x_{n}\right)$ in general. Show that $d\left(x_{n}, x_{m}\right) \geq \epsilon$ for $n \neq m$.]\\
7. Let $(X, d)$ be a metric space. If $f$ satisfies the condition

$$
d(f(x), f(y))<d(x, y)
$$

for all $x, y \in X$ with $x \neq y$, then $f$ is called a shrinking map. If there is a number $\alpha<1$ such that

$$
d(f(x), f(y)) \leq \alpha d(x, y)
$$

for all $x, y \in X$, then $f$ is called a contraction. A fixed point of $f$ is a point $x$ such that $f(x)=x$.\\
(a) If $f$ is a contraction and $X$ is compact, show $f$ has a unique fixed point. [Hint: Define $f^{1}=f$ and $f^{n+1}=f \circ f^{n}$. Consider the intersection $A$ of the sets $A_{n}=f^{n}(X)$.]\\
(b) Show more generally that if $f$ is a shrinking map and $X$ is compact, then $f$ has a unique fixed point. [Hint: Let $A$ be as before. Given $x \in A$, choose $x_{n}$ so that $x=f^{n+1}\left(x_{n}\right)$. If $a$ is the limit of some subsequence of the sequence $y_{n}=f^{n}\left(x_{n}\right)$, show that $a \in A$ and $f(a)=x$. Conclude that $A=f(A)$, so that $\operatorname{diam} A=0$.]\\
(c) Let $X=[0,1]$. Show that $f(x)=x-x^{2} / 2$ maps $X$ into $X$ and is a shrinking map that is not a contraction. [Hint: Use the mean-value theorem of calculus.]\\
(d) The result in (a) holds if $X$ is a complete metric space, such as $\mathbb{R}$; see the exercises of §43. The result in (b) does not: Show that the map $f: \mathbb{R} \rightarrow \mathbb{R}$ given by $f(x)=\left[x+\left(x^{2}+1\right)^{1 / 2}\right] / 2$ is a shrinking map that is not a contraction and has no fixed point.

\section*{§29 Local Compactness}
In this section we study the notion of local compactness, and we prove the basic theorem that any locally compact Hausdorff space can be imbedded in a certain compact Hausdorff space that is called its one-point compactification.

Definition. A space $X$ is said to be locally compact at $\boldsymbol{x}$ if there is some compact subspace $C$ of $X$ that contains a neighborhood of $x$. If $X$ is locally compact at each of its points, $X$ is said simply to be locally compact.

Note that a compact space is automatically locally compact.\\
EXAMPLE 1. The real line $\mathbb{R}$ is locally compact. The point $x$ lies in some interval $(a, b)$. which in turn is contained in the compact subspace $[a, b]$. The subspace $\mathbb{Q}$ of rational numbers is not locally compact, as you can check.

EXAMPLE 2. The space $\mathbb{R}^{n}$ is locally compact; the point $x$ lies in some basis element $\left(a_{1}, b_{1}\right) \times \cdots \times\left(a_{n}, b_{n}\right)$, which in turn lies in the compact subspace $\left[a_{1}, b_{1}\right] \times \cdots \times\left[a_{n}, b_{n}\right]$. The space $\mathbb{R}^{\omega}$ is not locally compact; none of its basis elements are contained in compact subspaces. For if

$$
B=\left(a_{1}, b_{1}\right) \times \cdots \times\left(a_{n}, b_{n}\right) \times \mathbb{R} \times \cdots \times \mathbb{R} \times \cdots
$$

were contained in a compact subspace, then its closure

$$
\bar{B}=\left[a_{1}, b_{1}\right] \times \cdots \times\left[a_{n}, b_{n}\right] \times \mathbb{R} \times \cdots
$$

would be compact, which it is not.\\
EXAMPLE 3. Every simply ordered set $X$ having the least upper bound property is locally compact: Given a basis element for $X$, it is contained in a closed interval in $X$, which is compact.

Two of the most well-behaved classes of spaces to deal with in mathematics are the metrizable spaces and the compact Hausdorff spaces. Such spaces have many useful properties, which one can use in proving theorems and making constructions and the like. If a given space is not of one of these types, the next best thing one can hope for is that it is a subspace of one of these spaces. Of course, a subspace of a metrizable space is itself metrizable, so one does not get any new spaces in this way. But a subspace of a compact Hausdorff space need not be compact. Thus arises the question: Under what conditions is a space homeomorphic with a subspace of a compact Hausdorff space? We give one answer here. We shall return to this question in Chapter 5 when we study compactifications in general.

Theorem 29.1. Let $X$ be a space. Then $X$ is locally compact Hausdorff if and only if there exists a space $Y$ satisfying the following conditions:\\
(1) $X$ is a subspace of $Y$.\\
(2) The set $Y-X$ consists of a single point.\\
(3) $Y$ is a compact Hausdorff space.

If $Y$ and $Y^{\prime}$ are two spaces satisfying these conditions, then there is a homeomorphism of $Y$ with $Y^{\prime}$ that equals the identity map on $X$.

Proof. Step 1. We first verify uniqueness. Let $Y$ and $Y^{\prime}$ be two spaces satisfying these conditions. Define $h: Y \rightarrow Y^{\prime}$ by letting $h$ map the single point $p$ of $Y-X$ to the point $q$ of $Y^{\prime}-X$, and letting $h$ equal the identity on $X$. We show that if $U$ is open in $Y$, then $h(U)$ is open in $Y^{\prime}$. Symmetry then implies that $h$ is a homeomorphism.

First, consider the case where $U$ does not contain $p$. Then $h(U)=U$. Since $U$ is open in $Y$ and is contained in $X$, it is open in $X$. Because $X$ is open in $Y^{\prime}$, the set $U$ is also open in $Y^{\prime}$, as desired.

Second, suppose that $U$ contains $p$. Since $C=Y-U$ is closed in $Y$, it is compact as a subspace of $Y$. Because $C$ is contained in $X$, it is a compact subspace of $X$. Then because $X$ is a subspace of $Y^{\prime}$, the space $C$ is also a compact subspace of $Y^{\prime}$. Because $Y^{\prime}$ is Hausdorff, $C$ is closed in $Y^{\prime}$, so that $h(U)=Y^{\prime}-C$ is open in $Y^{\prime}$, as desired.

Step 2. Now we suppose $X$ is locally compact Hausdorff and construct the space $Y$. Step 1 gives us an idea how to proceed. Let us take some object that is not a point of $X$, denote it by the symbol $\infty$ for convenience, and adjoin it to $X$, forming the set $Y=X \cup\{\infty\}$. Topologize $Y$ by defining the collection of open sets of $Y$ to consist\\
of (1) all sets $U$ that are open in $X$, and (2) all sets of the form $Y-C$, where $C$ is a compact subspace of $X$.

We need to check that this collection is, in fact, a topology on $Y$. The empty set is a set of type (1), and the space $Y$ is a set of type (2). Checking that the intersection of two open sets is open involves three cases:

$$
\begin{array}{rlrl}
U_{1} \cap U_{2} & & \text { is of type (1) } \\
\left(Y-C_{1}\right) \cap\left(Y-C_{2}\right) & =Y-\left(C_{1} \cup C_{2}\right) & & \text { is of type (2). } \\
U_{1} \cap\left(Y-C_{1}\right) & =U_{1} \cap\left(X-C_{1}\right) & & \text { is of type (1), }
\end{array}
$$

because $C_{1}$ is closed in $X$. Similarly, one checks that the union of any collection of open sets is open:

$$
\begin{array}{rlrl}
\bigcup U_{\alpha} & =U & & \text { is of type (1) } \\
\bigcup\left(Y-C_{\beta}\right) & =Y-\left(\bigcap C_{\beta}\right) & =Y-C & \\
\left(\bigcup U_{\alpha}\right) \cup\left(\bigcup\left(Y-C_{\beta}\right)\right) & =U \cup(Y-C) & =Y-(C-U),
\end{array} .
$$

which is of type (2) because $C-U$ is a closed subspace of $C$ and therefore compact.\\
Now we show that $X$ is a subspace of $Y$. Given any open set of $Y$, we show its intersection with $X$ is open in $X$. If $U$ is of type (1), then $U \cap X=U$; if $Y-C$ is of type (2), then $(Y-C) \cap X=X-C$; both of these sets are open in $X$. Conversely, any set open in $X$ is a set of type (1) and therefore open in $Y$ by definition.

To show that $Y$ is compact, let $\mathcal{A}$ be an open covering of $Y$. The collection $\mathcal{A}$ must contain an open set of type (2), say $Y-C$, since none of the open sets of type (1) contain the point $\infty$. Take all the members of $A$ different from $Y-C$ and intersect them with $X$; they form a collection of open sets of $X$ covering $C$. Because $C$ is compact, finitely many of them cover $C$; the corresponding finite collection of elements of $\mathscr{A}$ will, along with the element $Y-C$, cover all of $Y$.

To show that $Y$ is Hausdorff, let $x$ and $y$ be two points of $Y$. If both of them lie in $X$, there are disjoint sets $U$ and $V$ open in $X$ containing them, respectively. On the other hand, if $x \in X$ and $y=\infty$, we can choose a compact set $C$ in $X$ containing a neighborhood $U$ of $x$. Then $U$ and $Y-C$ are disjoint neighborhoods of $x$ and $\infty$, respectively, in $Y$.

Step 3. Finally, we prove the converse. Suppose a space $Y$ satisfying conditions (1)-(3) exists. Then $X$ is Hausdorff because it is a subspace of the Hausdorff space $Y$. Given $x \in X$, we show $X$ is locally compact at $x$. Choose disjoint open sets $U$ and $V$ of $Y$ containing $x$ and the single point of $Y-X$, respectively. Then the set $C=Y-V$ is closed in $Y$, so it is a compact subspace of $Y$. Since $C$ lies in $X$, it is also compact as a subspace of $X$; it contains the neighborhood $U$ of $x$. \(\square\)

If $X$ itself should happen to be compact, then the space $Y$ of the preceding theorem is not very interesting, for it is obtained from $X$ by adjoining a single isolated point. However, if $X$ is not compact, then the point of $Y-X$ is a limit point of $X$, so that $\bar{X}=Y$.

Definition. If $Y$ is a compact Hausdorff space and $X$ is a proper subspace of $Y$ whose closure equals $Y$, then $Y$ is said to be a compactification of $X$. If $Y-X$ equals a single point, then $Y$ is called the one-point compactification of $X$.

We have shown that $X$ has a one-point compactification $Y$ if and only if $X$ is a locally compact Hausdorff space that is not itself compact. We speak of $Y$ as "the" one-point compactification because $Y$ is uniquely determined up to a homeomorphism.

EXAMPLE 4. The one-point compactification of the real line $\mathbb{R}$ is homeomorphic with the circle, as you may readily check. Similarly, the one-point compactification of $\mathbb{R}^{2}$ is homeomorphic to the sphere $S^{2}$. If $\mathbb{R}^{2}$ is looked at as the space $\mathbb{C}$ of complex numbers, then $\mathbb{C} \cup\{\infty\}$ is called the Riemann sphere, or the extended complex plane.\\
In some ways our definition of local compactness is not very satisfying. Usually one says that a space $X$ satisfies a given property "locally" if every $x \in X$ has "arbitrarily small" neighborhoods having the given property. Our definition of local compactness has nothing to do with "arbitrarily small" neighborhoods, so there is some question whether we should call it local compactness at all.

Here is another formulation of local compactness, one more truly "local" in nature; it is equivalent to our definition when $X$ is Hausdorff.

Theorem 29.2. Let $X$ be a Hausdorff space. Then $X$ is locally compact if and only if given $x$ in $X$, and given a neighborhood $U$ of $x$, there is a neighborhood $V$ of $x$ such that $\bar{V}$ is compact and $\bar{V} \subset U$.\\
Proof. Clearly this new formulation implies local compactness; the set $C=\bar{V}$ is the desired compact set containing a neighborhood of $x$. To prove the converse, suppose $X$ is locally compact; let $x$ be a point of $X$ and let $U$ be a neighborhood of $x$. Take the one-point compactification $Y$ of $X$, and let $C$ be the set $Y-U$. Then $C$ is closed in $Y$, so that $C$ is a compact subspace of $Y$. Apply Lemma 26.4 to choose disjoint open sets $V$ and $W$ containing $x$ and $C$, respectively. Then the closure $\bar{V}$ of $V$ in $Y$ is compact; furthermore, $\bar{V}$ is disjoint from $C$, so that $\bar{V} \subset U$, as desired.

Corollary 29.3. Let $X$ be locally compact Hausdorff; let $A$ be a subspace of $X$. If $A$ is closed in $X$ or open in $X$, then $A$ is locally compact.\\
Proof. Suppose that $A$ is closed in $X$. Given $x \in A$, let $C$ be a compact subspace of $X$ containing the neighborhood $U$ of $x$ in $X$. Then $C \cap A$ is closed in $C$ and thus compact, and it contains the neighborhood $U \cap A$ of $x$ in $A$. (We have not used the Hausdorff condition here.)

Suppose now that $A$ is open in $X$. Given $x \in A$, we apply the preceding theorem to choose a neighborhood $V$ of $x$ in $X$ such that $\bar{V}$ is compact and $\bar{V} \subset A$. Then $C=\bar{V}$ is a compact subspace of $A$ containing the neighborhood $V$ of $x$ in $A$.

Corollary 29.4. A space $X$ is homeomorphic to an open subspace of a compact Hausdorff space if and only if $X$ is locally compact Hausdorff.\\
Proof. This follows from Theorem 29.1 and Corollary 29.3.

\section*{Exercises}
\begin{enumerate}
  \item Show that the rationals $\mathbb{Q}$ are not locally compact.
  \item Let $\left\{X_{\alpha}\right\}$ be an indexed family of nonempty spaces.\\
(a) Show that if $\Pi X_{\alpha}$ is locally compact, then each $X_{\alpha}$ is locally compact and $X_{\alpha}$ is compact for all but finitely many values of $\alpha$.\\
(b) Prove the converse, assuming the Tychonoff theorem.
  \item Let $X$ be a locally compact space. If $f: X \rightarrow Y$ is continuous, does it follow that $f(X)$ is locally compact? What if $f$ is both continuous and open? Justify your answer.
  \item Show that $[0,1]^{\omega}$ is not locally compact in the uniform topology.
  \item If $f: X_{1} \rightarrow X_{2}$ is a homeomorphism of locally compact Hausdorff spaces, show $f$ extends to a homeomorphism of their one-point compactifications.
  \item Show that the one-point compactification of $\mathbb{R}$ is homeomorphic with the circle $S^{1}$.
  \item Show that the one-point compactification of $S_{\Omega}$ is homeomorphic with $\bar{S}_{\Omega}$.
  \item Show that the one-point compactification of $\mathbb{Z}_{+}$is homeomorphic with the subspace $\{0\} \cup\left\{1 / n \mid n \in \mathbb{Z}_{+}\right\}$of $\mathbb{R}$.
  \item Show that if $G$ is a locally compact topological group and $H$ is a subgroup, then $G / H$ is locally compact.
  \item Show that if $X$ is a Hausdorff space that is locally compact at the point $x$, then for each neighborhood $U$ of $x$, there is a neighborhood $V$ of $x$ such that $\bar{V}$ is compact and $\bar{V} \subset U$.\\
*11. Prove the following:\\
(a) Lemma. If $p: X \rightarrow Y$ is a quotient map and if $Z$ is a locally compact Hausdorff space, then the map
\end{enumerate}

$$
\pi=p \times i_{Z}: X \times Z \longrightarrow Y \times Z
$$

is a quotient map.\\[0pt]
[Hint: If $\pi^{-1}(A)$ is open and contains $x \times y$, choose open sets $U_{1}$ and $V$ with $\bar{V}$ compact, such that $x \times y \in U_{1} \times V$ and $U_{1} \times \bar{V} \subset \pi^{-1}(A)$. Given $U_{i} \times \bar{V} \subset \pi^{-1}(A)$, use the tube lemma to choose an open set $U_{i+1}$ containing $p^{-1}\left(p\left(U_{i}\right)\right)$ such that $U_{i+1} \times \bar{V} \subset \pi^{-1}(A)$. Let $U=\bigcup U_{i}$; show that $U \times V$ is a saturated neighborhood of $x \times y$ that is contained in $\pi^{-1}(A)$.]\\
An entirely different proof of this result will be outlined in the exercises of §46.\\
(b) Theorem. Let $p: A \rightarrow B$ and $q: C \rightarrow D$ be quotient maps. If $B$ and $C$ are locally compact Hausdorff spaces, then $p \times q: A \times C \rightarrow B \times D$ is a quotient map.

\section*{*Supplementary Exercises: Nets}
We have already seen that sequences are "adequate" to detect limit points, continuous functions, and compact sets in metrizable spaces. There is a generalization of the notion of sequence, called a net, that will do the same thing for an arbitrary topological space. We give the relevant definitions here, and leave the proofs as exercises. Recall that a relation $\preceq$ on a set $A$ is called a partial order relation if the following conditions hold:\\
(1) $\alpha \preceq \alpha$ for all $\alpha$.\\
(2) If $\alpha \preceq \beta$ and $\beta \preceq \alpha$, then $\alpha=\beta$.\\
(3) If $\alpha \preceq \beta$ and $\beta \preceq \gamma$, then $\alpha \preceq \gamma$.

Now we make the following definition:\\
A directed set $J$ is a set with a partial order $\leq$ such that for each pair $\alpha, \beta$ of elements of $J$, there exists an element $\gamma$ of $J$ having the property that $\alpha \leq \gamma$ and $\beta \preceq \gamma$.

\begin{enumerate}
  \item Show that the following are directed sets:\\
(a) Any simply ordered set, under the relation $\leq$.\\
(b) The collection of all subsets of a set $S$, partially ordered by inclusion (that is, $A \leq B$ if $A \subset B$ ).\\
(c) A collection $\mathcal{A}$ of subsets of $S$ that is closed under finite intersections, partially ordered by reverse inclusion (that is $A \leq B$ if $A \supset B$ ).\\
(d) The collection of all closed subsets of a space $X$, partially ordered by inclusion.
  \item A subset $K$ of $J$ is said to be cofinal in $J$ if for each $\alpha \in J$, there exists $\beta \in K$ such that $\alpha \preceq \beta$. Show that if $J$ is a directed set and $K$ is cofinal in $J$, then $K$ is a directed set.
  \item Let $X$ be a topological space. A net in $X$ is a function $f$ from a directed set $J$ into $X$. If $\alpha \in J$, we usually denote $f(\alpha)$ by $x_{\alpha}$. We denote the net $f$ itself by the symbol $\left(x_{\alpha}\right)_{\alpha \in J}$, or merely by $\left(x_{\alpha}\right)$ if the index set is understood.
\end{enumerate}

The net ( $x_{\alpha}$ ) is said to converge to the point $x$ of $X$ (written $x_{\alpha} \rightarrow x$ ) if for each neighborhood $U$ of $x$, there exists $\alpha \in J$ such that

$$
\alpha \leq \beta \Longrightarrow x_{\beta} \in U .
$$

Show that these definitions reduce to familiar ones when $J=\mathbb{Z}_{+}$.\\
4. Suppose that

$$
\left(x_{\alpha}\right)_{\alpha \in J} \longrightarrow x \text { in } X \quad \text { and } \quad\left(y_{\alpha}\right)_{\alpha \in J} \longrightarrow y \text { in } Y .
$$

Show that $\left(x_{\alpha} \times y_{\alpha}\right) \longrightarrow x \times y$ in $X \times Y$.\\
5. Show that if $X$ is Hausdorff, a net in $X$ converges to at most one point.\\
6. Theorem. Let $A \in X$. Then $x \in \bar{A}$ if and only if there is a net of points of $A$ converging to $x$.\\[0pt]
[Hint: To prove the implication ⇒, take as index set the collection of all neighborhoods of $x$, partially ordered by reverse inclusion.]\\
7. Theorem. Let $f: X \rightarrow Y$. Then $f$ is continuous if and only if for every convergent net ( $x_{\alpha}$ ) in $X$, converging to $x$, say, the net ( $f\left(x_{\alpha}\right)$ ) converges to $f(x)$.\\
8. Let $f: J \rightarrow X$ be a net in $X$; let $f(\alpha)=x_{\alpha}$. If $K$ is a directed set and $g: K \rightarrow J$ is a function such that\\
(i) $i \leq j \Rightarrow g(i) \leq g(j)$,\\
(ii) $g(K)$ is cofinal in $J$,\\
then the composite function $f \circ g: K \rightarrow X$ is called a subnet of $\left(x_{\alpha}\right)$. Show that if the net $\left(x_{\alpha}\right)$ converges to $x$, so does any subnet.\\
9. Let $\left(x_{\alpha}\right)_{\alpha \in J}$ be a net in $X$. We say that $x$ is an accumulation point of the net $\left(x_{\alpha}\right)$ if for each neighborhood $U$ of $x$, the set of those $\alpha$ for which $x_{\alpha} \in U$ is cofinal in $J$.\\
Lemma. The net ( $x_{\alpha}$ ) has the point $x$ as an accumulation point if and only if some subnet of ( $x_{\alpha}$ ) converges to $x$.\\[0pt]
[Hint: To prove the implication ⇒, let $K$ be the set of all pairs ( $\alpha, U$ ) where $\alpha \in J$ and $U$ is a neighborhood of $x$ containing $x_{\alpha}$. Define $(\alpha, U) \preceq(\beta, V)$ if $\alpha \preceq \beta$ and $V \subset U$. Show that $K$ is a directed set and use it to define the subnet.]\\
10. Theorem. $X$ is compact if and only if every net in $X$ has a convergent subnet.\\[0pt]
[Hint: To prove the implication ⇒, let $B_{\alpha}=\left\{x_{\beta} \mid \alpha \leq \beta\right\}$ and show that $\left\{B_{\alpha}\right\}$ has the finite intersection property. To prove $\rightleftharpoons$, let $\mathcal{A}$ be a collection of closed sets having the finite intersection property, and let $\mathscr{B}$ be the collection of all finite intersections of elements of $\mathcal{A}$, partially ordered by reverse inclusion.]\\
11. Corollary. Let $G$ be a topological group; let $A$ and $B$ be subsets of $G$. If $A$ is closed in $G$ and $B$ is compact, then $A \cdot B$ is closed in $G$.\\[0pt]
[Hint: First give a proof using sequences, assuming that $G$ is metrizable.]\\
12. Check that the preceding exercises remain correct if condition (2) is omitted from the definition of directed set. Many mathematicians use the term "directed set" in this more general sense.


\end{document}