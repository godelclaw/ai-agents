% AIrxiv note compiled by Oruži for Zar
\documentclass[11pt]{article}

\usepackage[margin=1in]{geometry}
\usepackage[T1]{fontenc}
\usepackage[utf8]{inputenc}
\usepackage{lmodern}
\usepackage{microtype}

\usepackage{amsmath,amssymb,amsthm,mathtools}
\usepackage{stmaryrd} % for \llbracket \rrbracket
\usepackage{bbm}
\usepackage{enumitem}
\usepackage{hyperref}
\usepackage{cleveref}

\hypersetup{
  colorlinks=true,
  linkcolor=blue,
  citecolor=blue,
  urlcolor=blue
}

% --- Theorem environments
\newtheorem{theorem}{Theorem}
\newtheorem{lemma}{Lemma}
\newtheorem{definition}{Definition}
\newtheorem{remark}{Remark}
\newtheorem{example}{Example}

% --- Macros
\newcommand{\Evidence}{\mathsf{Evidence}}
\newcommand{\State}{\mathsf{State}}
\newcommand{\Query}{\mathsf{Query}}
\newcommand{\WM}{\mathsf{WM}}
\newcommand{\revise}{\mathbin{\oplus}}   % revision (evidence addition)
\newcommand{\tensor}{\mathbin{\otimes}}  % combination in valuation algebra / quantale
\newcommand{\Strength}{\mathsf{strength}}
\newcommand{\ev}{\mathsf{evidence}}
\newcommand{\side}{\Sigma}
\newcommand{\bn}{\mathsf{BN}}
\newcommand{\fg}{\mathsf{FG}}
\newcommand{\KS}{\mathsf{KS}}

\title{\vspace{-1.5em}
World-Model Calculus for Probabilistic Logic Networks:\\
Evidence, Views, and $\Sigma$-Guarded Compiled Inference
\vspace{-0.5em}}
\author{Zar (project direction) \and Oru\v{z}i (compiled notes)}
\date{\today}

\begin{document}
\maketitle

\begin{abstract}
This note distills the core ideas developed while formalizing a \emph{World-Model (WM) calculus}
for Probabilistic Logic Networks (PLN) inside Lean (the \texttt{Mettapedia} repository).

The guiding principle is: \emph{PLN inference is stateful posterior revision plus evidence extraction},
and ``fast PLN rules'' are \emph{compiled, $\Sigma$-guarded rewrites} whose soundness is discharged
relative to a declared \emph{world-model class} (Bayesian network, factor graph, joint evidence, etc.).

We separate:
(i) \textbf{evidence} as a rich algebraic object (often Heyting/quantale-valued),
(ii) \textbf{views} (probabilities, STVs/WTVs, typicalities) as explicit projections from evidence,
and
(iii) \textbf{tractable sublayers} where inference is exact relative to assumptions (e.g.\ d-separation).
We explain how this architecture supports message passing / activation spreading (MORK-style),
how to make approximation risk explicit (ledgers, hypothesis weights, projection operators),
and how to integrate \textbf{quantifiers} via satisfying-set semantics and Goertzel-style weakness.
Finally we outline a proof-calculus agenda: soundness and (fragmented) completeness results
relative to chosen semantics (propositional/FO/Henkin-HO, finite vs quasi-Borel, etc.).
\end{abstract}

\tableofcontents

\section{Motivation and the ``council direction''}

PLN aims to support \emph{probabilistic reasoning in open-ended domains}: uncertain facts,
imperfect independence, and large-scale inference control (message passing, GNN-like spreading,
positive feedback loops). The historical friction is well-known:

\begin{itemize}[leftmargin=2em]
\item \textbf{Local truth-value rules are not globally complete.}
Even if you have local evidence for $A,B,C$ and for links $A\Rightarrow B$, $B\Rightarrow C$,
you cannot in general compute the \emph{exact} evidence for $A\Rightarrow C$
without carrying joint correlation information.
\item \textbf{Independence assumptions are the real price.}
In practice you need to ask: when is a Markov/d-separation assumption justified, what do you do
when it is violated, and how do you quantify drift?
\item \textbf{Operational factor graphs are not automatically semantics.}
A message-passing graph whose factors are ``deduction'' or ``induction'' update formulas is
useful for search, but it is only semantically correct when those factors are \emph{compiled}
from a true world-model factorization (or guarded by explicit assumptions).
\end{itemize}

The emerging design answer is:

\begin{quote}
\textbf{Keep the semantic kernel small and honest.}
World-model states revise under evidence, and queries extract evidence.
Everything else---truth values, probabilities, ``fast rules'', message passing---are
\emph{explicit views and $\Sigma$-guarded tactics} whose validity is proved
\emph{relative to a declared model class}.
\end{quote}

This aligns the ``council'' instincts: Kolmogorov/Cox/Knuth--Skilling give valuation axioms,
Kohlas--Shenoy give algebraic computation, Tao wants hypotheses stated cleanly,
and Meredith/Stay want the categorical/process structure to stay compositional.

\section{The WM kernel: states, revision, and queries}

\subsection{World-model interface}

At the semantic core we posit:
\begin{itemize}[leftmargin=2em]
\item a type of world-model states $W:\State$ (posterior states, knowledge states, factorized models, etc.),
\item a type of queries $q:\Query$ (propositions, links, quantified claims, etc.),
\item an evidence type $\Evidence$, and
\item operations \emph{revision} and \emph{evidence extraction}.
\end{itemize}

Conceptually:
\[
  W \revise \Delta \;\;\text{is revision of state $W$ by new evidence $\Delta$} \qquad
  \ev(W,q)\in \Evidence \;\;\text{is evidence supporting query $q$ in state $W$.}
\]

In Lean this is packaged as a \texttt{WorldModel} typeclass with an \texttt{EvidenceType} for states.
The essential law is \textbf{additivity of evidence extraction under revision}:
\[
  \ev(W_1 \revise W_2, q) = \ev(W_1,q) \revise \ev(W_2,q).
\]
This is the ``semantic spine'' that makes revision and querying behave like a calculus.

\subsection{Judgments as a proof-calculus skeleton}

The WM kernel naturally yields sequent-style judgments:

\begin{itemize}[leftmargin=2em]
\item $\vdash_{\WM} W$ --- \emph{$W$ is a derivable/valid WM state} (a posterior produced from priors and updates),
\item $\vdash_q\, W \Downarrow q \mapsto e$ --- \emph{from state $W$, query $q$ yields evidence $e$},
\item optionally $\vdash_s\, W \Downarrow q \mapsto s$ --- a scalar \emph{strength view} of the query.
\end{itemize}

This is already a \emph{proof calculus}, but it is a \emph{stateful} one:
proof objects are states and evidence terms, not only formulas.
To avoid re-introducing ``hidden semantics'' through convenience truth-values,
we treat STVs/WTVs/probabilities as \textbf{views} computed from evidence.

\subsection{Query equivalence and $\Sigma$-guarded rewrites}

We isolate a key metatheoretic concept:

\begin{definition}[Query equivalence]
Two queries $q_1,q_2\in\Query$ are \emph{evidence-equivalent} if
\[
  \forall W\in\State.\;\; \ev(W,q_1)=\ev(W,q_2).
\]
\end{definition}

A ``PLN inference rule'' is then packaged as a \emph{sound rewrite}:

\begin{definition}[$\Sigma$-guarded rewrite rule]
A rewrite rule consists of:
\begin{itemize}[leftmargin=2em]
\item a side condition $\side$ (a proposition or a structured context),
\item a conclusion query $q$,
\item a derived evidence term $\mathrm{derive}(W)$, and
\item a proof that if $\side$ holds then $\mathrm{derive}(W)=\ev(W,q)$ for all $W$.
\end{itemize}
\end{definition}

In practice, many PLN rules rewrite only the \emph{strength view} of a query:
they prove an identity about $\Strength(\ev(W,q))$ rather than about full evidence.

\section{Evidence vs views: keep approximations explicit}

\subsection{Evidence as an algebraic object}

PLN needs an evidence object that can encode:
\begin{itemize}[leftmargin=2em]
\item positive and negative support,
\item partial truth / paraconsistency,
\item monotone aggregation of evidence, and
\item (in some sublayers) probabilistic conjugacy (e.g.\ Beta/Dirichlet counts).
\end{itemize}

A useful abstraction is that $\Evidence$ forms (at least) a \emph{commutative quantale}:
a complete lattice with a monoidal product $\tensor$ distributing over joins.
In several PLN developments, $\Evidence$ is also (or instead) treated as a Heyting algebra
(so negation is intuitionistic/paraconsistent rather than Boolean).

\subsection{Views as projections}

A \textbf{view} is a function $v:\Evidence\to X$ (often $X=[0,1]$ or $\mathbb{R}_{\ge 0}^\infty$),
chosen for operational convenience:
\[
  \text{probability view}\quad p(e) \in [0,1], \qquad
  \text{STV/WTV view}\quad \mathrm{stv}(e)\in [0,1]\times[0,1], \quad \text{etc.}
\]
Views \emph{must be explicit}, because they necessarily forget information.
This is not ``annoying monadic programming'' for its own sake---it is the enforcement
mechanism that prevents silent unsoundness.

\begin{remark}[The totality gate]
If $\Evidence$ has incomparable elements, there is no faithful order-embedding into $\mathbb{R}$.
So a single real-valued view cannot be globally complete.
This is not a bug; it is a statement about what information you are discarding.
\end{remark}

\section{A key no-go: why completeness cannot be purely local}

A foundational result proved in the Lean development can be phrased as:

\begin{theorem}[No local complete link-deduction rule (informal)]
There is no function
\[
  f:\Evidence^5\to \Evidence
\]
that takes only \emph{local} link/proposition evidence (for $A,B,C$ and $A\Rightarrow B$, $B\Rightarrow C$)
and returns the \emph{exact} evidence for $A\Rightarrow C$ for \emph{all} world-model states.
\end{theorem}

Interpretation: \textbf{correlations live in the joint state}.
So ``fast PLN rules'' must be framed as \emph{admissible rewrites under assumptions}:
they are not globally complete inference rules in the sequent-calculus sense.

This no-go result is the reason the WM calculus is \emph{state-first}:
the state carries whatever correlation structure is needed for truth of queries.

\section{Tractable sublayers: Bayesian networks and factor graphs}

\subsection{World-model classes and $\Sigma$ side conditions}

To scale, we restrict the class of admissible states:
\[
  \mathcal{C} \subseteq \State \quad \text{(e.g.\ DAG-factorizing BNs, Markov random fields, etc.)}.
\]
Inside such a class, we can prove additional theorems (screening-off, d-separation),
and compile them into admissible rewrite rules.

We write $\side$ for explicit side conditions:
\begin{itemize}[leftmargin=2em]
\item structural hypotheses (a particular DAG/factorization),
\item conditional independence facts (from d-separation),
\item positivity/nondegeneracy assumptions (to avoid division by $0$),
\item approximation/error ledger invariants (if using heuristics).
\end{itemize}

\subsection{Variable elimination as the exact query engine}

Given a discrete factor graph (or a BN compiled to a factor graph), we can compute
unnormalized weights for constraints by variable elimination (VE):
\[
  Z(\text{constraints}) \;=\; \sum_{\text{assignments consistent with constraints}} \prod_f \phi_f.
\]
In Lean, a generic VE engine is implemented for factor graphs parameterized by a semiring $K$,
with specializations to $\mathbb{R}_{\ge 0}^\infty$ (ENNReal) for probabilities.

\begin{remark}
A \texttt{CommSemiring} assumption is \emph{stronger} than separate \texttt{AddCommMonoid}+\texttt{Mul},
but it is not ``making it a ring''. Rings add additive inverses; semirings do not.
We use semirings because VE is ``sum of products'' and needs distributivity.
\end{remark}

\subsection{Fast PLN rules as compiled tactics}

A PLN rule (e.g.\ deduction) becomes:

\begin{quote}
\textbf{A $\Sigma$-guarded rewrite of a hard query into smaller queries}, together with a proof
that the \emph{strength view} of the conclusion equals the rule formula applied to
the strength views of the premises.
\end{quote}

Crucially: this is a \emph{strength-level} rewrite unless the model class supplies a law of
evidence flow itself. The evidence for the conclusion is still obtained by querying the WM state,
unless we can prove more.

\section{Knuth--Skilling bridges: valuations and regraduation}

Knuth--Skilling (K\&S) ``foundations of inference'' axiomatize valuation schemes on lattices.
In our setting, K\&S plays two roles:

\begin{enumerate}[leftmargin=2em]
\item \textbf{Probability as a valuation/view.}
Within Boolean world sets, probabilities arise as valuations and can be handled in standard ways.
\item \textbf{Regraduation bridges for computation.}
K\&S representation theorems justify mapping an abstract valuation to a real scale where
combination becomes addition or scaled multiplication, enabling efficient computation.
\end{enumerate}

In the Lean development, a regraduation map
\[
  \Theta : \alpha \to \mathbb{R}
\]
turns an abstract additive valuation algebra into real addition:
\[
  \Theta(x \star y) = \Theta(x) + \Theta(y).
\]
This is explicitly a \textbf{scale change}, not probability normalization.

This matters for PLN because:
\begin{itemize}[leftmargin=2em]
\item it provides a principled bridge between abstract evidence/valuation objects and numerical algorithms,
\item it lets the VE engine operate over K\&S-valued factor graphs after regrading,
\item it keeps ``probability'' as a \emph{view} rather than the foundational type.
\end{itemize}

\section{Factor graphs in two roles: semantics vs control (MORK alignment)}

A key conceptual cleanup:

\begin{enumerate}[leftmargin=2em]
\item \textbf{Semantic factor graphs:} factors \emph{are the model}.
They represent a genuine factorization (BN/MRF), and VE / junction tree is exact
relative to that model class.
\item \textbf{Operational factor graphs:} factors are \emph{rule schemas} or \emph{control edges}.
Message passing is a heuristic that schedules which queries to ask / which rewrites to attempt.
It becomes \emph{semantics} only after compilation: each operational factor is justified by
a semantic factor (or by a proven admissible rewrite under $\Sigma$).
\end{enumerate}

\subsection{Evidence ledgers and positive feedback loops}

Message passing, activation spreading, and positive feedback loops are not disallowed.
They just need accountability:

\begin{itemize}[leftmargin=2em]
\item Track provenance and overlap to avoid double-counting evidence.
\item Track $\Sigma$ hypotheses explicitly (e.g.\ ``this subgraph is Markov'').
\item Maintain a ledger of approximation error / drift estimates.
\end{itemize}

This viewpoint matches the ``geodesic inference control'' intuition:
control can be decentralized, but evidence must not be silently duplicated.

\subsection{Anchor states and recurrence}

Practical systems often revisit certain ``anchor'' situations (daily routines, infrastructure states,
public transit modes). If the agent revisits anchor-like regimes infinitely often, the system can
calibrate its approximation ledgers. This suggests an empirical condition:

\begin{quote}
\emph{We need enough recurrence in the experienced state space to re-estimate which simplifying
assumptions are valid in the domains where we rely on them.}
\end{quote}

When recurrence is weak, the model-class validity itself must be treated as uncertain
(and updated like any other hypothesis).

\section{Gluing logics: projection operators and probabilistic model-class uncertainty}

The ``complete logic + tractable bubbles'' dream is achievable if we resist one temptation:
\emph{never silently treat a simplifying assumption as if it were true}.

\subsection{Model class as a hypothesis}

Let $M$ range over model classes (Markov, Beta--Bernoulli, Gaussian, etc.).
Treat ``the domain is Markov'' as a hypothesis with probability (or evidence) mass.
Then:
\[
  P(q) \;=\; \sum_M P(M)\, P(q \mid M),
\]
where $P(q\mid M)$ can be computed in a tractable ``bubble'' (factor graph, linear model, etc.),
and $P(M)$ is estimated in the general world-model layer.

\subsection{Projections (forgetful maps) as approximation interfaces}

Let $\pi_M : \State \to \State_M$ be a projection from a rich state to a restricted state
appropriate for model class $M$. Soundness questions become:
\begin{itemize}[leftmargin=2em]
\item What queries commute with projection? (exactness of compiled tactics)
\item What inequalities or bounds hold? (safe approximations)
\item How do we revise $P(M)$ when projections fail diagnostic tests?
\end{itemize}

This is the correct place for ``topological gluing'' intuitions:
projections are like restriction maps, and consistency conditions correspond to commuting diagrams.

\subsection{Provenance and overlap: partial revision instead of wrong addition}

Evidence sources can overlap (shared observations, shared causal channels, ``data incest'').
If overlap is unknown, revision should not be a total operation. A principled fix is:

\begin{itemize}[leftmargin=2em]
\item Use a \textbf{partial} commutative monoid (separation algebra):
$W_1 \revise W_2$ is defined only when the fragments are known independent/disjoint.
\item If disjointness is uncertain, either keep fragments separate (delayed fusion),
or fuse conservatively using bounds (credal sets / Fr\'echet bounds / covariance intersection analogues).
\item Track provenance IDs so ``additivity'' is never applied to overlapping sources.
\end{itemize}

This is the clean mathematical home of ``approximation ledgers''.

\section{Quantifiers in the WM calculus}

PLN must support probabilistic reasoning over quantified claims:
\[
  P(\exists x.\,\mathrm{BlackSwan}(x)),\qquad
  P(\forall x.\,\mathrm{Human}(x)\Rightarrow \mathrm{Mortal}(x)).
\]

\subsection{Satisfying sets and weakness: a usable semantics}

A clean PLN-compatible semantics treats a predicate as a \emph{frame-valued satisfying set}:
\[
  \chi_P : U \to \Evidence,
\]
analogous to a subobject classifier in a topos.
Then universal quantification is evaluated via Goertzel's \emph{weakness} on the diagonal relation:
\[
  \forall x.\,P(x)
  \quad\leadsto\quad
  \mathrm{weakness}_\mu\big(\{(u,v)\mid P(u)\wedge P(v)\}\big)
\]
for an explicit weight function $\mu$.
Existential quantification can be defined by De Morgan using Heyting negation:
\[
  \exists x.\,P(x) := \neg \forall x.\,\neg P(x).
\]

Two distinct quantifier views are worth keeping separate:

\begin{itemize}[leftmargin=2em]
\item \textbf{Extensional (classical-ish) quantifiers:} ``all individuals satisfy''.
\item \textbf{Typicality quantifiers:} ``generality'' measured by weakness/diagonal mass.
\end{itemize}

The separation prevents subtle category mistakes: typicality is not classical $\forall$,
but it is exactly what many PLN uses want.

\subsection{How quantifiers live in the WM calculus}

In the WM architecture, quantified statements are simply queries $q$:
the world model state decides how to compute $\ev(W,q)$.
Different world-model classes yield different quantifier semantics:
finite-domain exact, sampled approximate, or quasi-Borel/infinite.

Fast quantified rules are again $\Sigma$-guarded rewrites (e.g.\ Skolemization,
instantiation heuristics), with explicit approximation tracking.

\subsection{Higher-order quantification and higher-order probability}

Quantifying over predicates (HO) is hard in crisp logics because the domain is huge.
But probabilistic semantics changes the game:

\begin{itemize}[leftmargin=2em]
\item \textbf{Higher-order probabilities can be flattened.}
Kyburg argues that ``second-order probabilities'' add no conceptual power:
they can be represented as ordinary joint distributions over expanded spaces.
So hyperpriors/hyperparameters are \emph{first-order} variables in a larger model.
\item \textbf{Henkin semantics as a pragmatic HO target.}
Instead of full set-theoretic HO semantics, quantify only over a chosen collection of predicates.
This yields completeness theorems similar to first-order logic, and matches ``representable concept''
quantification.
\item \textbf{Quasi-Borel / measurable HO as the long game.}
If the WM state is a probabilistic object on a QBS-like domain, HO quantification can be interpreted
as integration over function spaces---still difficult, but at least mathematically well-defined.
\end{itemize}

\section{Temporal/fixpoint reasoning: modal $\mu$-calculus as a PLN bridge (optional)}

Once reasoning becomes stateful (WM revision) and iterative (message passing), temporal/fixpoint
logics become natural. One promising bridge is the modal $\mu$-calculus:

\begin{itemize}[leftmargin=2em]
\item temporal operators (``eventually'', ``always'') can often be presented as least/greatest fixpoints;
\item PLN temporal operators can be embedded into $\mu$-calculus under a suitable semantics;
\item fixpoint structure also aligns with recurrent ``anchor states'' and invariant tracking.
\end{itemize}

In practice: keep temporal/fixpoint reasoning as an additional query language and add rewrite rules
that are sound relative to the chosen transition-model WM class.

\section{Process calculi, linear logic, and graph-structured semantics}

MeTTa is implemented in Rholang (a $\rho$-calculus descendant), and the project involves
experts in higher-order process calculi and categorical semantics.
This suggests a productive alignment:

\begin{itemize}[leftmargin=2em]
\item WM states can be treated as \emph{resources} (linear logic flavor).
\item Revision is a commutative monoidal ``addition'' of evidence/resources.
\item Query answering is a form of \emph{observation} (a morphism out of state).
\item Operational factor graphs become explicit \emph{process networks} that schedule observations
and rewrites, while the WM semantics provides correctness conditions.
\end{itemize}

This is a plausible bridge to OSLF / type-theoretic presentations and graph-structured lambda theories:
quantifiers become binders, and factor-graph compilation becomes a typed elaboration step.

\section{What ``sound and complete'' should mean here}

You want a proof calculus, not only a semantic kernel.
The WM calculus is already a proof calculus skeleton; the missing pieces are \emph{fragment-specific}
soundness/completeness theorems.

A clean framing:

\subsection{Soundness}

For a fragment $\mathcal{L}$ and semantics $\llbracket\cdot\rrbracket$:
\begin{quote}
\textbf{Soundness:} every derivable judgment corresponds to a true semantic statement
about evidence/strength in the target model class.
\end{quote}

In practice:
\begin{itemize}[leftmargin=2em]
\item Kernel soundness: revision/query laws (commutation, additivity).
\item Rewrite soundness: each $\Sigma$-guarded compiled rule preserves the chosen view (e.g.\ strength).
\item Approximate soundness: rules preserve \emph{bounds} or carry explicit error budget.
\end{itemize}

\subsection{Completeness (fragmented, but meaningful)}

Completeness should be stated relative to a declared semantics and fragment:
\begin{itemize}[leftmargin=2em]
\item \textbf{Propositional fragment:} evidence-quantale semantics vs IPL/intermediate logics.
\item \textbf{First-order fragment:} satisfying-set/weakness quantifiers (finite or Henkin FO).
\item \textbf{BN fragment:} completeness relative to BN query answering (VE/junction tree).
\item \textbf{Higher-order fragment:} Henkin completeness; QBS semantics as aspirational.
\end{itemize}

This is not a retreat. It is how serious proof theory works: different fragments have different theorems.

\section{Lean artifact map (for posterity)}

These are the main artifacts referenced by the architecture (names may evolve, but the roles should not):

\begin{itemize}[leftmargin=2em]
\item \texttt{Mettapedia/Logic/PLNWorldModel.lean}: WM kernel (states, revision, query extraction).
\item \texttt{Mettapedia/Logic/PLNWorldModelCalculus.lean}: $\Sigma$-guarded rewrite-rule schema
and explicit strength-view judgments.
\item \texttt{Mettapedia/Logic/PLNJointEvidence.lean}: complete joint-evidence (Dirichlet-over-worlds) WM instance.
\item \texttt{Mettapedia/Logic/PLNJointEvidenceNoGo.lean}: formal no-go theorem for local link-evidence propagation.
\item \texttt{Mettapedia/Logic/EvidenceQuantale.lean}: algebraic evidence layer (quantale/Heyting structure).
\item \texttt{Mettapedia/Logic/PLNBNCompilation.lean}: $\Sigma$-guarded screening-off rewrites for BN sublayer.
\item \texttt{Mettapedia/ProbabilityTheory/BayesianNetworks/FactorGraph.lean}: factor graph semantics.
\item \texttt{Mettapedia/ProbabilityTheory/BayesianNetworks/VariableElimination.lean}: VE query engine (sum-of-products).
\item \texttt{Mettapedia/ProbabilityTheory/KnuthSkilling/Bridges/ValuationAlgebra.lean}: K\&S regraduation bridge.
\item \texttt{Mettapedia/Logic/PLNFirstOrder/QuantifierSemantics.lean}: extensional vs typicality quantifier views.
\end{itemize}

\section{Roadmap: ``final steps'' that actually matter}

The most valuable next theorems (because they make PLN usable and honest):

\begin{enumerate}[leftmargin=2em]
\item \textbf{Independence-to-screening-off bridge.}
Prove that d-separation / conditional independence implies the screening-off equalities required by
PLN deduction rules (including both $B$ and $\neg B$ branches), under explicit positivity.
\item \textbf{VE correctness as a WM-strength rule.}
Show that VE computes the same strength view as querying the BN world model,
so compiled fast rules can target $\vdash_s$ judgments.
\item \textbf{Quantifier rewrite rules with explicit approximation ledgers.}
Hook satisfying-set quantifiers into the rewrite layer, with soundness under stated domain hypotheses
(finite, sampled, exchangeable, etc.).
\item \textbf{Fragment completeness statements.}
Prove completeness for the propositional and (selected) FO/Henkin fragments relative to the semantics
used in code. Treat QBS HO as a separate long-horizon milestone.
\item \textbf{Operational integration (MORK).}
Implement message passing / activation spreading as inference control that proposes $\Sigma$-guarded rewrites,
while the WM layer remains the single source of semantic truth.
\end{enumerate}

\section*{Acknowledgments / provenance}

This note compiles ideas emerging from ongoing collaboration around PLN, MeTTa, and the Lean formalization
in \texttt{Mettapedia}, with conceptual inspiration from:
Knuth--Skilling foundations of inference, valuation algebras, belief propagation / junction trees,
topos semantics of quantifiers, and higher-order process calculi.

\begin{thebibliography}{99}\small

\bibitem{KnuthSkilling}
K.\ H.\ Knuth and J.\ Skilling.
\newblock \emph{Foundations of Inference}.
\newblock (K\&S representation theorems for valuations on lattices.)

\bibitem{KohlasShenoy}
J.\ Kohlas and P.\ Shenoy.
\newblock Computation in valuation algebras.
\newblock (Axioms: combination $\tensor$ and marginalization $\downarrow$; local computation.)

\bibitem{Dechter}
R.\ Dechter.
\newblock Bucket elimination: A unifying framework for probabilistic inference.
\newblock (Variable elimination; complexity via treewidth.)

\bibitem{Pearl}
J.\ Pearl.
\newblock \emph{Probabilistic Reasoning in Intelligent Systems}.
\newblock (Bayesian networks, d-separation.)

\bibitem{Kyburg}
H.\ E.\ Kyburg.
\newblock Higher order probabilities and interval probabilities.
\newblock (Flattening higher-order probabilities into joint distributions.)

\bibitem{MacLaneMoerdijk}
S.\ Mac~Lane and I.\ Moerdijk.
\newblock \emph{Sheaves in Geometry and Logic}.
\newblock (Subobject classifiers, semantics for quantifiers.)

\end{thebibliography}

\end{document}
